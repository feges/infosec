\chapter[Совершенная криптостойкость]{Криптосистемы совершенной криптостойкости}
\selectlanguage{russian}

Рассмотрим модель криптосистемы, в которой Алиса выступает источником сообщений $m \in \group{M}$. Алиса использует некоторую функцию шифрования, результатом вычисления которой является шифротекст $c \in \group{C}$:

	\[c = E_{K_1}\left(m\right).\]

Шифротекст $c$ передаётся по открытому каналу легальному пользователю Бобу, причём по пути он может быть перехвачен нелегальным пользователем (криптоаналитиком) Евой.

Боб, обладая ключом расшифрования $K_2$, расшифровывает сообщение с использованием функции расшифрования:
	\[m' = D_{K_2}\left(c \right).\]

Рассмотрим теперь исходное сообщение, передаваемый шифротекст и ключи шифрования (и расшифрования, если они отличаются) в качестве случайных величин, описывая их свойства с точки зрения теории информации. Далее полагаем, что в криптосистеме ключи шифрования и расшифрования совпадают.

Будет называть криптосистему \textit{корректной}, если она обладает следующими свойствами.
\begin{itemize}
	\item Легальный пользователь имеет возможность однозначно восстановить исходное сообщение, то есть
					\[H \left( M | C, K \right) = 0, \]
					\[m' = m.\]
	\item Выбор ключа шифрования не зависит от исходного сообщения
					\[ I \left( K ; M \right) = 0, \]
					\[ H \left( K | M \right) = H \left( K \right). \]
\end{itemize}

Второе условие является, в некотором виде, условием на возможность отделить ключ шифрования от данных и алгоритма шифрования.

\section[Определения]{Определения совершенной криптостойкости}

Понятие совершенной секретности (или стойкости) введено американским ученым Клодом Шенноном. В конце Второй мировой войны он закончил работу, посвященную теории связи в секретных системах\cite{Shannon:1949:CTS}. Эта работа вошла составной частью в собрание его трудов, вышедшее в русском переводе в 1963 году.~\cite{Shannon:1963} Понятие о стойкости шифров по Шеннону связано с решением задачи криптоанализа по одной криптограмме.

Криптосистемы совершенной стойкости могут применяться как в современных вычислительных сетях так и для шифрования любой бумажной корреспонденции. Основной проблемой применения данных шифров для шифрования больших объемов информации является необходимость распространения ключей объемом, не меньшим чем передаваемые сообщения.

\begin{definition}\label{perfect_by_probabilities}
Будем называть криптосистему \textbf{совершенно криптостойкой}, если апостериорное распределение вероятностей исходного случайного сообщения $m_i \in \group{M}$ при регистрации случайного шифротекста $c_k \in \group{C}$ совпадает с априорным распределением~\cite{Gultyaeva:2010}:

	\[\forall m_j \in \group{M}, c_k \in \group{C}: P \left( m = m_j | c = c_k \right) = P \left( m = m_j \right).\]
\end{definition}

Данное условие можно переформулировать в терминах статистических свойств сообщения, ключа и шифротекста как случайных величин.

\begin{definition}\label{perfect_by_enthropy}
Криптосистема называется совершенно криптостойкой, если условная энтропия сообщения при известном шифротексте равна безусловной:
	\[H \left( M | C \right) = H \left( M \right),\]
	\[I \left( M; C \right) = 0.\]
\end{definition}

Можно показать, что определения \ref{perfect_by_probabilities} и \ref{perfect_by_enthropy} тождественны.

\section[Условие]{Условие совершенной криптостойкости}

Найдем оценку количества информации, которое содержит шифротекст $C$ относительно сообщения $M$
\[ I(M; C) = H(M) - H(M | C). \]
Очевидны следующие соотношения условных и безусловных энтропий \cite{GabPil:2007}:
\[H(K|C)=H(K|C)+H(M|K,C)=H(M,K|C),\]
\[H(M,K|C)=H(M|C)+H(K|M,C)\geq H(M|C),\]
\[H(K)\geq H(K|C)\geq H(M|C).\]
Отсюда получаем:
 \[ I(M; C) = H(M) - H(M | C)\geq H(M)-H(K). \]
Из последнего неравенства следует, что взаимная информация между сообщением и шифротекстом равна нулю, если энтропия ключа не меньше энтропии сообщений. С другой стороны, взаимная информация между сообщением и шифротекстом равна нулю, если они статистически независимы. Таким образом условием совершенной криптостойкости является неравенство
\[ H(M) \leq H(K).\]
%Если утверждение верно, то количество информации в шифротексте относительно открытого текста $I(M; C)$ равно нулю:
%  \[ I(M; C) = H(M) - H(M | C) = 0, \]
%так как для статистически независимых величин условная энтропия равна безусловной энтропии, то есть $H(M) = H(M | C)$.

%Функцию шифрования обозначим $E: \{ M, K \} \rightarrow C$. Процедура шифрования состоит из следующих шагов.
%\begin{itemize}
%    \item Легальный пользователь $A$ выбирает ключ $k \in K$ и секретно сообщает его легальному пользователю $B$ (дополнительная задача -- распределение ключей).
%    \item По открытому сообщению $m \in M$ и выбранному ключу $k$ вычисляют шифрованное сообщение $c = E_k(m) \in C$.
%\end{itemize}

%Основное требование при шифровании состоит в том, чтобы при выбранном ключе $k$ вычисление $c$  было легкой задачей для любого сообщения $m$.

%Функцию расшифрования обозначим $D: \{ C, K \} \rightarrow M$. Процедура расшифрования состоит из следующих шагов.
%\begin{itemize}
%    \item Легальный пользователь $B$ получает от $A$ секретный ключ $k \in K$.
 %   \item $B$ по принятому шифрованному сообщению $c \in C$ и известному ключу $k$ вычисляет открытое сообщение $m = D_k(c) \in M$.
%\end{itemize}

%Основное требование: при выбранном ключе $k$ вычисление $m$ должно быть легкой задачей для любого $c$. С другой стороны, при неизвестном ключе $k$ вычисление открытого сообщения $m$ по известному шифрованному сообщению $c$ должно быть трудной задачей для любого $c$.

%Криптостойкость шифра оценивается числом операций, необходимым для определения: открытого текста $m$ по шифротексту $c$, либо ключа шифрования $k$ по открытому тексту $m$ и шифротексту $c$.

%$M, C, K$ интерпретируются как случайные величины.
%Пусть заданы распределения вероятностей $P_m(M), P_c(C), P_k(K)$. По определению шифрование $C = E_K(M)$ -- детерминированная функция своих аргументов.
%Если при выбранном шифре оказалось, что открытый текст $M$ и шифротекст $C$ -- статистически независимые случайные величины, то считается, что такая система обладает совершенной криптостойкостью.


%\subsection{Длина ключа}

%Пусть сообщения $m\in M$ и ключи $r\in K$ являются независимыми случайными величинами. Это значит, что их совместная вероятность $P_{mk}(M, K)$ равна произведению отдельных вероятностей:
%\[P_{mk}(M, K) = P_m(M) \cdot P_k(K).\]
%Пусть $C = E_K(M)$ -- множество шифрованных текстов, $M = D_K(C)$ -- множество расшифрованных текстов. Можно найти вероятности $P_c(C), P_{mck}(M,C,K)$.

%Используя известные соотношения о безусловной и условной энтропии~\cite{GabPil:2007}, оценим энтропию открытых текстов $M$ с учетом статистической независимости $M$ и $C$:
 %   \[ H(M) = H(M | C) \leq H(MK | C) = H(K | C) + H(M | CK) = \]     \[ = H(K | C) \leq H(K). \]

%Так как энтропия открытого текста при заданном шифротексте и известном ключе равна нулю, то $H(M|CK)=0$. В результате получаем     \[ H(M) \leq H(K). \]

Обозначим через $L(M)$ и $L(K)$ длину сообщений и ключа соответственно. Известно~\cite{GabPil:2007}, что $H(M)\leq L(M)$ и равенство достигается, когда сообщения состоят из статистически независимых и равновероятных символов. Такое же свойство выполняется и для случайных ключей $H(K)\leq L(K)$. Таким образом, достаточным условием совершенной криптостойкости системы можно считать неравенство
 \[ L(M) \leq L(K)\]
при случайном выборе ключа.

%С другой стороны, энтропия открытого текста $H(M)$ характеризует длину последовательности для описания случайной величины $M$ (открытого сообщения), а $H(K)$ характеризует длину последовательности для описания ключа. Следовательно, совершенная криптостойкость возможна только тогда, когда длина ключа не меньше, чем длина шифруемого сообщения, то есть     \[ H(M) \leq H(K). \] Как правило, длина сообщения заранее неизвестна и ограничена большим числом. Выбрать ключ длины не меньшей, чем возможное сообщение не представляется возможным или рациональным, и один и тот же ключ (или его преобразования) используется многократно для шифрования блоков сообщения фиксированной длины. То есть, $H(K) \ll H(M)$.

На самом деле, сообщение может иметь произвольную (заранее не ограниченную) длину. Поэтому генерация и, главным образом, доставка легальным пользователям случайного и достаточного длинного ключа становятся критическими проблемами. Практическим решением этих проблем является многократное использование одного и того же ключа при условии, что его длина гарантирует вычислительную невозможность любой известной атаки на подбор ключа.


\section{Криптосистема Вернама}
\index{криптосистема!Вернама}
Приведем пример системы с совершенной криптостойкостью.

Пусть сообщение представлено двоичной последовательностью длины $N$:
    \[ m = (m_1, m_2, \dots, m_N). \]
Распределение вероятностей сообщений $P_m(m)$ может быть любым. Ключ также представлен двоичной последовательностью $ k = (k_1, k_2, \dots, k_N)$ той же длины, но с равномерным распределением
    \[ P_k(k) = \frac{1}{2^N} \]
для всех ключей.

Шифрование в криптосистеме \textbf{Вернама} осуществляется путем покомпонентного суммирования по модулю алфавита последовательностей открытого текста и ключа:
    \[ C = M \oplus K = (m_1 \oplus k_1, ~ m_2 \oplus k_2, \dots,  m_N \oplus k_N). \]

Легальный пользователь знает ключ и осуществляет расшифрование:
    \[ M =C \oplus K = (m_1 \oplus k_1, ~ m_2 \oplus k_2, \dots, m_N \oplus k_N). \]

Найдем вероятностное распределение $N$-блоков шифротекстов, используя формулу
    \[ P(c = a) = P(m \oplus k = a) = \sum_{m} P(m) P(m \oplus k = a | m) = \]
    \[ = \sum_{m} P(m) P(k \oplus m) = \sum_{m} P(m) \frac{1}{2^N} = \frac{1}{2^N}. \]

Получили подтверждение известного факта: сумма двух случайных величин, одна из которых имеет равномерное распределение, является случайной величиной с равномерным распределением. В нашем случае распределение ключей равномерное, поэтому распределение шифротекстов тоже равномерное.

Запишем совместное распределение открытых текстов и шифротекстов:
    \[ P(m = a, c = b) ~=~ P(m = a) ~ P(c = b | m = a). \]

Найдем условное распределение
    \[ P(c = b | m = a) ~=~ P(m \oplus k = b | m = a) ~= \]
    \[ =~ P(k = b \oplus a | m = a) ~=~ P(k = b \oplus a) ~=~ \frac{1}{2^N}, \]
так как ключ и открытый текст являются независимыми случайными величинами. Итого:
    \[ P(c=b | m=a) = \frac{1}{2^N}. \]

Подстановка правой части этой формулы в формулу для совместного распределения дает
    \[ P(m=a,c=b)=P(m=a)\frac{1}{2^N}, \]
что доказывает независимость шифротекстов и открытых текстов в этой системе. По доказанному выше, количество информации в шифротексте относительно открытого текста равно нулю. Это значит, что рассмотренная криптосистема Вернама обладает совершенной секретностью (криптостойкостью) при условии, что для каждого $N$-блока (сообщения) генерируется случайный (одноразовый) $N$-ключ.

\input{unicity_distance}
