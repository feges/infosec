\documentclass[10pt,a4paper]{book}
%\documentclass[12pt,report,russian]{ncc}
%\usepackage{a4wide}
% Для векторых русских шрифтов в PDF не забудьте установить пакеты cm-super & cm-unicode
\usepackage{cmap}                       % Поддержка поиска русских слов в PDF (pdflatex)
\usepackage[X2, T2A]{fontenc}
%\usepackage[T2, OT1]{fontenc}
\usepackage[utf8]{inputenc}
\usepackage[english,russian]{babel}
\usepackage{indentfirst}                % Красная строка в первом абзаце
%\usepackage{misccorr}
%Может быть установлено 8pt, 9pt, 10pt, 11pt, 12pt, 14pt, 17pt, and 20pt
%\usepackage[12pt]{extsizes}
%\usepackage[mag=1000,a4paper,left=3cm,right=2cm,top=2cm,bottom=2cm,noheadfoot]{geometry}

\usepackage{amsmath} % разрешить \texttt и аналогичные в формулах
\usepackage{amssymb, } % дополнительные математические символы
\usepackage{graphicx} % поддержка изображений

%\usepackage{amsfonts, eucal, bm, color, }

\usepackage{algorithm, algorithmic}     % 'algorithm' environments
\floatname{algorithm}{Алгоритм}
\usepackage{multirow}                   % multirow cells in tables
\usepackage{arydshln}                   % dash lines in tables
\usepackage{subfig, float, wrapfig}     % sub figures
\usepackage{caption}                    % titles for figures
\usepackage{makeidx}                   % index
%\usepackage[xindy]{imakeidx}
\usepackage[totoc=true]{idxlayout}      % балансировка индексов на последней странице, индекс в ToC
\usepackage{enumerate}
\usepackage{fancybox}                   % страница в рамке
%\usepackage{fancyhdr}                  % глава и секция вверху страницы
%\usepackage{layout}
\usepackage[left=1.84cm, right=1.5cm, paperwidth=14cm, top=1.8cm, bottom=2cm, height=19.8cm, paperheight=20cm]{geometry}
\usepackage[parentracker=true,
  backend=biber,
  hyperref=auto,
  language=auto,
  citestyle=gost-numeric,
  defernumbers=true,
  bibstyle=gost-numeric,
  sortlocale=ru_RU
]{biblatex}								% библиография по ГОСТу
\addbibresource{bibliography.bib}

% поддержка гиперссылок; гиперссылки в pdf, должен быть последним загруженным пакетом
\ifx\pdfoutput\undefined
    \usepackage[unicode,dvips]{hyperref}
\else
    \usepackage[pdftex,colorlinks,unicode,bookmarks]{hyperref}
\fi

%\paperwidth=16.8cm \oddsidemargin=0cm \evensidemargin=0cm \hoffset=-0.33cm \textwidth=13.2cm
%\paperheight=24cm \voffset=-0.4cm \topmargin=0cm \headsep=0cm \headheight=0cm \textheight=19.8cm \footskip=0.9cm

% параметры PDF файла
\hypersetup{
    pdftitle={Защита информации},
    pdfauthor={Э. М. Габидулин, А. С. Кшевецкий, А. И. Колыбельников, С. М. Владимиров},
    pdfsubject=учебное пособие,
    pdfkeywords={защита информации, криптография, МФТИ}
}

% добавить точку после номера секции, раздела и т.д.
\makeatletter
\def\@seccntformat#1{\csname the#1\endcsname.\quad}
\def\numberline#1{\hb@xt@\@tempdima{#1\if&#1&\else.\fi\hfil}}
\makeatother

% перенос слов с тире
%\lccode`\-=`\-
%\defaulthyphenchar=127

% изменить подписи к рисункам, таблицам и т.д.
\captionsetup{labelsep=period}          % заменить : на .
\captionsetup{textformat=period}        % Подписи завершать точкой
%\captionsetup[table]{position=above}    % вертикальные отступы подписи таблицы для случая, когда подпись вверху
%\captionsetup[figure]{position=below}   % вертикальные отступы подписи рисунка для случая, когда подпись внизу

%% стиль главы и секции вверху страницы
%\pagestyle{fancy}
%%\renewcommand{\chaptermark}[1]{\markboth{#1}{}}
%\renewcommand{\sectionmark}[1]{\markright{#1}{}}
%
%%\fancyhf{}
%%\fancyfoot[СE,CO]{\thepage}
%%\fancyhead[LE]{\textsc{\nouppercase{\leftmark}}}
%\fancyhead[RO]{\textsc{\nouppercase{\rightmark}}}
%
%\fancypagestyle{plain}{ %
%\fancyhf{}                              % remove everything
%\renewcommand{\headrulewidth}{0pt}      % remove lines as well
%\renewcommand{\footrulewidth}{0pt}}

% запретить выходить за границы страницы
\sloppy

\newtheorem{theorem}{Теорема}[section]
\newtheorem{lemma}[theorem]{Лемма}
\newtheorem{definition}[theorem]{Определение}
\newtheorem{property}[theorem]{Утверждение}
\newtheorem{corollary}[theorem]{Следствие}
%\newtheorem{algorithm}[theorem]{Алгоритм}
\newtheorem{remark}[theorem]{Замечание}
\newcommand{\proof}{\noindent\textsc{Доказательство.\ }}

%\newtheorem{example}{\textsc{\textbf{Пример}}}
\newcommand{\example}{\textsc{\textbf{Пример.}} }
\newcommand{\exampleend}

\newcommand{\set}[1]{\mathbb{#1}}
\newcommand{\group}[1]{\mathbb{#1}}
\newcommand{\E}{\group{E}}
\newcommand{\F}{\group{F}}
\newcommand{\GF}[1]{\group{GF}(#1)}
\newcommand{\Gr}{\group{G}}
\newcommand{\R}{\group{R}}
\newcommand{\Z}{\group{Z}}
\newcommand{\MAC}{\textrm{MAC}}
\newcommand{\HMAC}{\textrm{HMAC}}
\newcommand{\PK}{\textrm{PK}}
\newcommand{\SK}{\textrm{SK}}

%Наконец, существует способ дублировать знаки операций, который мы приведем безо всяких пояснений. Включив
%\newcommand*{\hm}[1]{#1\nobreak\discretionary{}{\hbox{\mathsurround=0pt #1}}{}}
%в преамбулу, можно написать $a\hm+b\hm+c\hm+d$, при этом в формуле a\hm+b\hm+c\hm+d при переносе знак + будет продублирован.

% Дублирование символов бинарных операций ("+", "-", "="), набранных в строчных формулах, при переносе на другую строку:
%%begin{latexonly}
%\renewcommand\ne{\mathchar"3236\mathchar"303D\nobreak
%      \discretionary{}{\usefont
%      {OMS}{cmsy}{m}{n}\char"36\usefont
%      {OT1}{cmr}{m}{n}\char"3D}{}}
%\begingroup
%\catcode`\+\active\gdef+{\mathchar8235\nobreak\discretionary{}%
% {\usefont{OT1}{cmr}{m}{n}\char43}{}}
%\catcode`\-\active\gdef-{\mathchar8704\nobreak\discretionary{}%
% {\usefont{OMS}{cmsy}{m}{n}\char0}{}}
%\catcode`\=\active\gdef={\mathchar12349\nobreak\discretionary{}%
% {\usefont{OT1}{cmr}{m}{n}\char61}{}}
%\endgroup
%\def\cdot{\mathchar8705\nobreak\discretionary{}%
% {\usefont{OMS}{cmsy}{m}{n}\char1}{}}
%\def\times{\mathchar8706\nobreak\discretionary{}%
% {\usefont{OMS}{cmsy}{m}{n}\char2}{}}
%\mathcode`\==32768
%\mathcode`\+=32768
%\mathcode`\-=32768
%%end{latexonly}

\makeindex

\begin{document}
\selectlanguage{russian}

%\layout

% рамка границ страницы http://www.ctan.org/tex-archive/macros/latex/contrib/fancybox/fancybox-doc.pdf
% сделать поиск по fancypage, thisfancypage
%\thisfancypage{}{\fbox}
%\thisfancypage{\fbox}{}
%\fancypage{}{\fbox}         % закомментировать
%\fancypage{\fbox}{\fbox}    % закомментировать
%\fancypage{\setlength{\fboxsep}{32pt}\fbox}{}

\title{Защита информации \\ Учебное пособие}
\author{Габидулин Эрнст Мухамедович \\ Кшевецкий Александр Сергеевич \\ Колыбельников Александр Иванович \\ Владимиров Сергей Михайлович}
\date{
 %   \textbf{\textsc{Черновой вариант. Может содержать ошибки.}} \\
%    \today
}
\maketitle
\setcounter{page}{3}

\newpage
%\thispagestyle{empty}
\setcounter{tocdepth}{2}
\tableofcontents
%\thispagestyle{empty}
\newpage

%\lhead[\leftmark]{}
%\rhead[]{\rightmark}

\selectlanguage{russian}
\chapter*{Предисловие}
\addcontentsline{toc}{chapter}{Предисловие}

Изучение курса <<Защита информации>> необходимо начать с определения понятия \emph{"информация"}. В теоретической информатике \textbf{информация} -- это любые сведения, или цифровые данные, или сообщения, или документы, или файлы, которые могут быть переданы \emph{получателю информации} от \emph{источника информации}. Можно считать, что информация передается по какому-либо каналу связи с помощью некоторого носителя, которым может быть, например, распечатка текста, диск или другое устройство хранения информации, система передачи сигналов по оптическим, проводным или радио линиям связи и т. д.

\textbf{Защита информации} -- это сохранение \emph{целостности}, \emph{конфиденциальности} и \emph{доступности} информации, передаваемой или хранимой в какой-либо форме. Информацию необходимо защищать от разрушения ее целостности и конфиденциальности в результате вмешательства \emph{нелегального пользователя}.

\textbf{Целостность информации}\index{целостность} -- это сохранность информации (в любой ее форме представления) в неизменном (оригинальном) виде. \textbf{Конфиденциальность}\index{конфиденциальность} означает, что информация получена именно тем, кому она предназначалась, то есть \textbf{легальным пользователем}, и никто другой, то есть \emph{нелегальный пользователь}, эту информацию не получил.

\textbf{Доступность}\index{доступность} -- свойство информации быть доступной легальным пользователям в любой момент времени.

Чтобы реализовать защиту информации, используются различные математические методы, технические средства и организационные меры. В частности, источник информации (на передающей стороне) применяет \textbf{шифрование}, а легальный пользователь (на приемной стороне) осуществляет \textbf{расшифрование}\index{расшифрование}. Процесс получения информации нелегальным пользователем называется \textbf{дешифрованием}\index{дешифрование}\footnote{В англоязычной литературе словом <<decryption>> обозначается и расшифрование, и дешифрование.}, а сам нелегальный пользователь -- \textbf{криптоаналитиком}\index{криптоаналитик}.

В настоящем пособии рассмотрены только основные математические методы защиты информации, и среди них основной акцент сделан на криптографическую защиту, которая включает симметричные и несимметричные методы шифрования, формирование секретных ключей, протоколы ограничения доступа и аутентификации сообщений и пользователей. Кроме того, в пособии рассматриваются типовые уязвимости операционных и информационно-вычислительных систем.

\section*{Благодарности}
\addcontentsline{toc}{section}{Благодарности}
Авторы пособия благодарят студентов, аспирантов и сотрудников института, которые помогли с подготовкой, редактированием и поиском ошибок в тексте:

\begin{itemize}
	\item Дмитрий Банков (201-011 гр.)
	\item Даниил Бершацкий (201-012 гр.)
	\item Дмитрий Бородий (201-112 гр.)
	\item Дмитрий Вербицкий (201-119 гр.)
	\item Тагир Гадельшин (201-119 гр.)
	\item Марат Гаджибутаев (201-018 гр.)
	\item Евгений Глушков (201-012 гр.)
	\item Сергей Жестков (201-013 гр.)
	\item Дмитрий Зборовский (201-119 гр.)
	\item Марат Ибрагимов (201-114 гр.)
	\item Александр Иванов (201-011 гр.)
	\item Александр Иванов (201-019 гр.)
	\item Константин Ковальков (201-015 гр.)
	\item Виталий Крепак (201-013 гр.)
	\item Александр Кротов (201-011 гр.)
	\item Станислав Круглик (201-111 гр.)
	\item Надежда Мозолина (201-119 гр.)
	\item Милосердов Олег (201-016 гр.)
	\item Хыу Чунг Нгуен (201-015 гр.)
	\item Артём Никитин (201-012 гр.)
	\item Андрей Пунь (201-013 гр.)
	\item Иван Саюшев (201-112 гр.)
	\item Игорь Сорокин (201-112 гр.)
	\item Буй Зуи Тан (201-112 гр.)
	\item Евгений Юлюгин (201-916 гр.)
\end{itemize}


\chapter{Основные понятия и определения}

\section{Краткая история криптографии}

Вслед за возникновением письменности появилась задача обеспечения секретности и подлинности передаваемых сообщений путем так называемой тайнописи. Поскольку государства возникали почти одновременно с письменностью, дипломатия и военное управление требовали секретности.

Данные о первых способах тайнописи весьма обрывочны. Предполагается, что тайнопись была известна в древнем Египте и Вавилоне. До нашего времени дошли литературные свидетельства того, что секретная письменность использовалась в древней Греции. Наиболее известен метод шифрования, который использовался Гаем Юлием Цезарем (100--44~гг.~до~н.э.).

Первое известное исследование по анализу стойкости методов шифрования было сделано в <<Манускрипте о дешифровании криптографических сообщений>> Абу аль-Кинди (801–-873~гг.~н.~э.). Он показал, что моноалфавитные шифры, в которых каждому символу кодируемого текста ставится в однозначное соответствие какой-то другой символ алфавита, легко поддаются частотному криптоанализу. Абу аль-Кинди был так же знаком с более сложными полиалфавитными шифрами.

В европейских странах полиалфавитные шифры были открыты в эпоху Возрождения. Итальянский архитектор Баттиста Альберти (1404--1472) изобрел полиалфавитный шифр, который впоследствии получил имя дипломата XVI века Блеза де Виженера. В истории развития полиалфавитных шифров до XX века также наиболее известны немецкий аббат XVI века Иоганн Трисемус и английский ученый начала XIX века Чарльз Витстон. Витстон изобрел простой и стойкий способ полиалфавитной замены, называемый шифром Плейфера в честь лорда Плейфера, способствовавшему внедрению шифра. Шифр Плейфера использовался вплоть до Первой мировой войны.

Прообразом современных шифров для электронно-вычислительных машин стали так называемые роторные машины XX века, которые позволяли создавать и реализовывать устойчивые к взлому полиалфавитные шифры. Примером такой машины является немецкая машина <<Enigma>>, разработанная в конце Первой мировой войны. Период активного применения <<Enigma>> пришелся на Вторую мировую войну.

Появление в середине XX столетия первых ЭВМ кардинально изменило ситуацию. Вычислительные способности компьютеров подняли на совершенно новый уровень как возможности реализации шифров, недоступных ранее из-за их высокой сложности, так и возможности криптоаналитиков по их взлому. Следствием этого факта стало разделение шифров по области применения.

В 1976 году появился шифр DES (Data Encryption Standard)\index{шифр!DES}, который был принят как стандарт США. DES широко использовался для шифрования пакетов данных при передаче в компьютерных сетях и системах хранения данных. С 90-х годов параллельно с традиционными шифрами, основой которых была булева алгебра, активно развиваются шифры, основанные на операциях в конечном поле. Широкое распространение персональных компьютеров и быстрый рост объема передаваемых данных в компьютерных сетях привели к замене в 2002 году стандарта DES на более стойкий и быстрый в программной реализации стандарт -- шифр AES (Advanced Encryption Standard)\index{шифр!AES}. Окончательно, DES был выведен из эксплуатации как стандарт в 2005 году.

В беспроводных голосовых сетях передачи данных используются шифры с малой задержкой шифрования и расшифрования на основе посимвольных преобразований -- так называемые \emph{потоковые шифры}.

%Основным их преимуществом является сочетание помехоустойчивого кодирования с криптостойкостью шифра.

Параллельно с разработкой быстрых шифров в 1977 г. появился новый класс криптосистем, так называемые \emph{криптосистемы с открытым ключом}\index{криптосистема!с открытым ключом}. Хотя эти новые криптосистемы намного медленнее (технически сложнее) симметричных, они открыли принципиально новые возможности --  \emph{электронная подпись}, \emph{аутентификация} и \emph{сертификация} составили основу современной защищенной связи в интернете.

В настоящее время типичное использование криптографии в информационных системах состоит в:
\begin{itemize}
\item цифровой аутентификации пользователей с помощью криптосистем с открытым ключом,
\item создании кратковременных сеансовых ключей и
\item применении быстрых шифров в процессах обмена данными.
\end{itemize}


\section{Модель системы передачи с криптозащитой}
\selectlanguage{russian}

Простая модель системы передачи с криптозащитой представлена на рис. \ref{pic:Encrypt}, где введены следующие обозначения:
\begin{itemize}
    \item $A$ -- источник информации;
    \item $B$ -- получатель информации, легальный пользователь;
    \item $X$ -- сообщение до шифрования или \textbf{открытый текст}\index{открытый текст} (plaintext); $\set{M}$ -- множество всех возможных открытых текстов (от слова Message), $X \in \set{M}$;
    \item $K_1$ -- ключ шифрования\index{ключ!шифрования}; $\set{K}_E$ -- множество всех возможных ключей шифрования  (от слов Key и Encryption), $K_1 \in \set{K}_E$;
    \item $Y$ -- шифрованное сообщение (или \textbf{шифротекст}\index{шифротекст}, или \textbf{шифрограмма}\index{шифрограмма}); $\set{C}$ -- множество всех возможных шифротекстов (от термина Cipher text), $Y \in \set{C}$;
    \item $K_2$ -- ключ расшифрования\index{ключ!расшифрования}; $\set{K}_D$  -- множество возможных ключей расшифрования  (от слов Key и Decryption), зависящее от множества $\set{K}_E$, $K_2 \in \set{K}_D$.
\end{itemize}

\begin{figure}[!ht]
	\centering
	\includegraphics[width=1.0\textwidth]{pic/scheme-of-cipher}
	\caption{Передача информации с криптозащитой\label{pic:Encrypt}}
\end{figure}

\textbf{Шифр}\index{шифр} -- это множество обратимых функций отображения $E_{K_1}$\index{функция!шифрования} множества открытых текстов $\set{M}$ на множество шифротекстов $\set{C}$, зависящих от выбранного ключа шифрования $K_1$ из множества $\set{K}_E$:
%обратимое отображение пары из элемента множества открытых текстов $\set{M}$ и элемента множества ключей шифрования $\set{K}_E$ во множество шифротекстов $\set{C}$:
\begin{equation}
    \label{eq:Encryption}
    Y = E_{K_1}(X), ~ X \in \set{M}, ~ K_1 \in \set{K}_E, ~ Y \in \set{C}.
\end{equation}
Можно сказать, что шифрование -- это обратимая функция двух аргументов: сообщения и ключа. Для каждого $K_1$ эта функция должна быть обратимой. Обратимость -- основное условие шифрования, по которому каждому зашифрованному сообщению $Y$ и ключу $K$ соответствует одно исходное сообщение $X$. Легальный пользователь $B$ (на приемной стороне системы связи)  получает сообщение $Y$ и осуществляет процедуру \textbf{расшифрования}\index{расшифрование}.
Следует отличать шифрование от кодирования, так как кодирование -- это процесс сопоставления конкретным сообщениям строго определенной комбинации символов или сигналов, с целью повышения помехоустойчивости передаваемого сигнала.
Расшифрование --  это отображение множества шифротекстов $\set{C}$ во множество открытых текстов $\set{M}$ функцией $D_{K_2}$\index{функция!расшифрования}, зависящей от ключа расшифрования $K_2$ из множества $\set{K}_D$, являющейся обратной к функции $E_{K_1}$.
\begin{equation}
    \label{eq:Decryption}
    D_{K_2}(Y) = X, ~ Y \in \set{C}, ~ K_2 \in \set{K}_D, ~ X \in \set{M}.
\end{equation}

%Система передачи информации с криптозащитой называется \textbf{криптосистемой}\index{криптосистема}.(?????)

%В общем случае функция шифрования сюръективна и псевдослучайна, отображая один открытый текст в разные шифротексты. Если функция шифрования биективна, на практике ее инкапсулируют в другую функцию с целью добиться псевдослучайности шифрования одинаковых открытых текстов в разные шифротексты.

%Методы защиты информации зависят от возможных сценариев передачи. Рассмотрим несколько основных вариантов.
Рассмотрим возможные сценарии вмешательства криптоаналитика и организации защиты информации от его действий.
Пусть  $A$ --  источник и $B$ -- получатель сообщений.

\begin{description}
    \item[Сценарий 1.] Пусть $E$ -- \textbf{пассивный} криптоаналитик\index{криптоаналитик!пассивный}, который может подслушивать передачу, но не может вмешиваться в процесс передачи. Из пассивности криптоаналитика следует, что $Y = \widetilde{Y}$ и \textbf{целостность} информации обеспечена.

Цель защиты --- \textbf{обеспечение конфиденциальности}.

Средства защиты -- шифрование с помощью \emph{симметричных} или \emph{асимметричных } криптосистем.

Дополнительные задачи -- при большом числе пользователей должна быть решена задача \textbf{генерации и доставки секретных ключей} всем пользователям.

    \item[Сценарий 2.] Пусть $E$ -- \textbf{активный} криптоаналитик\index{криптоаналитик!активный}, который может изменять, удалять и вставлять сообщения или их части.

    Цель защиты -- \textbf{обеспечение конфиденциальности} и  \textbf{обеспечение целостности}.

Средства защиты --  шифрование и добавление \emph{имитовставки}\index{имитовставка} (Message Authentication Code -- $\MAC$), позволяющего обнаружить нарушение целостности.

    \item[Сценарий 3.] Пусть $E$ -- активный криптоаналитик, который может изменять, удалять и вставлять сообщения или их части), дополнительно к этому легальные пользователи $A$ и $B$ не доверяют друг другу.

Цель защиты -- \textbf{аутентификация }пользователей и документов.

Средства -- \emph{электронная подпись} и протокол идентификации (аутентификации) пользователей.
\end{description}

%%Возможно вмешательство нелегального пользователя $E$, называемого \textbf{криптоаналитиком}.
%%
%%
%%Если $X = \widetilde{X}$, то вмешательство криптоаналитика  $E$ не изменило передаваемое сообщение, и \textbf{целостность} информации обеспечена. Если криптоаналитик не получил информацию, содержащуюся в сообщении, то обеспечена \textbf{конфиденциальность}.
%%
%%Если в этой системе возможна двусторонняя передача, то есть от $A$ к $B$ и от $B$ к $A$, то говорят о взаимном обмене информацией между легальными пользователями.
%
%Секретность информации в современных шифрах обеспечивается секретным ключом, в то время как сам алгоритм криптосистемы является общеизвестным. Исторический опыт, например, система шифрования A5/1 в GSM, показывает, что секретность алгоритма шифрования \emph{ослабляет} криптостойкость шифра, а не увеличивает, из-за того, что система становится малоизученной.


\section{Классификация криптосистем}

\input{classification_by_symmetry}

\subsection{Шифры замены и перестановки}

Шифры, по способу преобразования открытого текста в шифротекст, разделяются на шифры замены и шифры перестановки.

\input{substitution_ciphers}

\subsubsection{Шифры перестановки}
\selectlanguage{russian}

Шифры \textbf{перестановки} реализуются следующим образом. Берут открытый текст, например буквенный, и разделяют на блоки определенной длины $x_1, x_2, \dots, x_m$. Затем осуществляется перестановка позиций блока (вместе с символами). Перестановки могут быть однократные и многократные. Частный случай перестановки -- сдвиг. Приведем пример:
\begin{center}
    секрет $\xrightarrow{\text{сдвиг}}$ ретсек $\xrightarrow{\text{перестановка}}$ рскете.
\end{center}
Ключ такого шифра указывает изменение порядка номеров позиций блока при шифровании и расшифровании.

Существуют так называемые \textbf{маршрутные перестановки}. Используется какая-либо геометрическая фигура, например прямоугольник. Запись открытого текста ведется по одному \emph{маршруту}, например по строкам, а считывание для шифрования осуществляется по другому маршруту, например по столбцам. Ключ шифра определяет эти маршруты.
В случае, когда рассматривается перестановка блока текста фиксированной длины, перестановку можно рассматривать как замену.

В полиалфавитных шифрах при шифровании открытый текст разбивается на блоки (последовательности) длины $n$, где $n$ -- \textbf{период}. Этот параметр выбирает \emph{криптограф} и держит его в секрете.

Поясним процедуру шифрования полиалфавитным шифром. Запишем шифруемое сообщение в матрицу по столбцам определенной длины. Пусть открытый текст таков: <<Игры различаются по содержанию, характерным особенностям, а также по тому, какое место они занимают в жизни детей>>. Зададим $n=4$ и запишем этот текст в матрицу размера $(4 \times 24)$:

\begin{center} \resizebox{\textwidth}{!}{ \begin{tabular}{|*{24}{c|}}
    \hline
    и&р&и&т&о&е&н&а&т&ы&о&н&я&а&п&м&к&е&о&а&а&ж&и&е \\
    г&а&ч&с&с&р&и&р&е&м&б&о&м&к&о&у&о&с&н&н&ю&и&д&й \\
    р&з&а&я&о&ж&ю&а&р&о&е&с&а&ж&т&к&е&т&и&и&т&з&е& \\
    ы&л&ю&п&д&а&х&к&н&с&н&т&т&е&о&а&м&о&з&м&в&н&т& \\
    \hline
\end{tabular} } \end{center}

Выбираем $4$ различных моноалфавитных шифра.

Первую строку

\begin{center} \resizebox{\textwidth}{!}{ \begin{tabular}{|*{24}{c|}}
    \hline
    и&р&и&т&о&е&н&а&т&ы&о&н&я&а&п&м&к&е&о&а&а&ж&и&е \\
    \hline
\end{tabular} } \end{center}

шифруем, используя первый шифр. Вторую строку

\begin{center} \resizebox{\textwidth}{!}{ \begin{tabular}{|*{24}{c|}}
    \hline
    г&а&ч&с&с&р&и&р&е&м&б&о&м&к&о&у&о&с&н&н&ю&и&д&й \\
    \hline
\end{tabular} } \end{center}

шифруем, используя второй шифр, и~т.д.

Выполняя расшифрование, легальный пользователь знает период. Он записывает принятую шифрограмму по строкам в матрицу с длиной строки равной периоду, к каждому столбцу применяет соответствующий ключ и расшифровывает сообщение, зная соответствующие шифры.

Шифры перестановки можно рассматривать как частный случай шифров замены, если отождествить один блок перестановки с одним символом большого алфавита.


\input{composite_chiphers}

\subsection{Примеры современных криптопримитивов}

Приведем примеры названий некоторых современных криптографических примитивов, из которых строят системы защиты информации:
\begin{itemize}
    \item DES\index{шифр!DES}, AES, ГОСТ 28147-89, Blowfish\index{шифр!Blowfish}, RC5\index{шифр!RC5}, RC6\index{шифр!RC6} -- блоковые симметричные шифры, скорость обработки -- десятки мегабайт в секунду,
    \item A5/1, A5/2, A5/3\index{шифр!A5}, RC4\index{шифр!RC4} -- потоковые симметричные шифры с высокой скоростью, семейство A5 применяется в мобильной связи GSM, RC4 -- в компьютерных сетях для SSL соединения между браузером и вебсервером,
    \item RSA\index{шифр!RSA} -- криптосистема с открытым ключом для шифрования,
    \item RSA\index{электронная подпись!RSA}, DSA\index{электронная подпись!DSA}, ГОСТ Р 34.10-2001\index{электронная подпись!ГОСТ Р 34.10-2001} -- криптосистемы с открытым ключом для электронной подписи,
    \item MD5\index{хэш-функция!MD5}, SHA-1\index{хэш-функция!SHA-1}, SHA-2\index{хэш-функция!SHA-2}, ГОСТ Р 34.11-94\index{хэш-функция!ГОСТ Р 34.11-94} -- криптографические хэш-функции.
\end{itemize}

\section{Методы криптоанализа и типы атак}
\selectlanguage{russian}

Нелегальный пользователь-криптоаналитик получает информацию путем дешифрования. Сложность этой процедуры определяется числом стандартных операций, которые надо выполнить для достижения цели. \textbf{Двоичной сложностью}\index{сложность!двоичная} (или битовой сложностью) алгоритма называется количество двоичных операций, которые необходимо выполнить для его завершения.
% Наиболее сложным является дешифрование полиалфавитных шифров.

Попытка криптоаналитика $E$ получить информацию называется \textbf{атакой}, или криптоатакой\index{атака}. Как правило, легальным пользователям нужно обеспечить защиту информации на протяжении от нескольких дней до 100 лет. Если попытка атаки оказалась удачной для нелегального пользователя $E$ и информация получена или может быть получена в ближайшем будущем, то такое событие называется  \textbf{взломом криптосистемы}\index{взлом криптосистемы} или \textbf{вскрытием криптосистемы}. Метод вскрытия криптосистемы называется \textbf{криптоанализом}\index{криптоанализ}. Криптосистема называется \textbf{криптостойкой}\index{криптостойкость}, если число стандартных операций для ее взлома превышает возможности современных вычислительных средств в течение всего времени ценности информации (до 100 лет).

В общем случае, в криптоанализе под \textbf{взломом} криптосистемы понимается построение алгоритма криптоатаки для получения доступа к информации с количеством операций, меньшим, чем планировалось при создании этой криптосистемы. Взлом криптосистемы -- это не обязательно реально осуществленное извлечение информации, так как количество операций для извлечения информации может быть вычислительно недостижимым как в настоящее время, так и в течение всего времени защиты.
%, но предполагается достижимым в будущем.

Рассмотрим основные сценарии работы криптоаналитика $E$. В первом сценарии криптоаналитик может осуществлять подслушивание и (или) перехват сообщений. Его вмешательство не нарушает целостности информации: $Y=\widetilde{Y}$. Эта роль криптоаналитика называется \textbf{пассивной}. Так как он получает доступ к информации, то здесь нарушается конфиденциальность.

Во втором сценарии роль криптоаналитика \textbf{активная}. Он может подслушивать, перехватывать сообщения и преобразовывать их по своему усмотрению: задерживать, искажать с помощью перестановок пакетов, устраивать обрыв связи, создавать новые сообщения и т.п. Так что в этом случае выполняется условие $Y \neq \widetilde{Y}$. Это значит, что одновременно нарушается целостность и конфиденциальность передаваемой информации.

Приведем примеры пассивных и активных атак.
\begin{itemize}
    \item Атака <<\textbf{человек посередине}>>\index{атака!<<человек посередине>>} (man-in-the-middle) подразумевает криптоаналитика, который разрывает канал связи, встраиваясь между $A$ и $B$, получает сообщения от $A$ и от $B$, а от себя отправляет новые, фальсифицированные сообщения. В результате $A$ и $B$ не замечают, что общаются с $E$, а не друг с другом.
    \item Атака \textbf{воспроизведения}\index{атака!воспроизведения} (replay attack) -- когда криптоаналитик может записывать и в будущем воспроизводить шифротексты, имитируя легального пользователя.
    \item Атака на \textbf{различение} сообщений\index{атака!на различение} означает, что криптоаналитик, наблюдая одинаковые шифротексты, может извлечь информацию об идентичности исходных открытых текстов.
    \item Атака на \textbf{расширение} сообщений\index{атака!на расширение} означает, что криптоаналитик может дополнить шифротекст осмысленной информацией без знания секретного ключа.
    \item \textbf{Фальсификация} шифротекстов\index{атака!фальсификацией} криптоаналитиком без знания секретного ключа.
\end{itemize}

Часто для нахождения секретного ключа криптоатаки строят в предположениях о доступности дополнительной информации. Приведем примеры.
\begin{itemize}
    \item Атака на основе известного открытого текста\index{атака!с известным открытым текстом} (CPA, chosen plaintext attack) предполагает возможность криптоаналитику выбирать открытый текст и получать для него соответствующий шифротекст.
    \item Атака на основе известного шифротекста\index{атака!с известным шифротекстом} (CCA, chosen ciphertext attack) предполагает возможность криптоаналитику выбирать шифротекст и получать для него соответствующий открытый текст.
\end{itemize}

Обязательным требованием к современным криптосистемам является устойчивость ко всем известным типам атак -- пассивным, активным и с дополнительной информацией.


%Приведем примеры возможных вариантов работы активного криптоаналитика.
%\begin{itemize}
%\item Криптоаналитик имеет $m$ шифрованных сообщений $Y_{1},Y_{2},\ldots Y_{m}$ и пытается определить ключ или прочитать открытый текст $X_{1},X_{2},\ldots X_{m}.$
%\item Криптоаналитик имеет несколько пар открытого и шифрованного текстов
%
%$(Y_{1},X_{1}),(Y_{2}X_{2}),\ldots (Y_{m}X_{m})$ и пытается дешифровать остальной текст или определить алгоритм шифрования или определить ключ.
%\item
%\item
%\item
%\end{itemize}

Для защиты информации от активного криптоаналитика и обеспечения целостности дополнительно к шифрованию сообщений применяют имитовставку\index{имитовставка}. Для неё используют обозначение $\MAC$ (message authentication code). Как правило, $\MAC$ строится на основе хэш-функций, которые будут описаны далее.

Существуют ситуации, когда пользователи $A$ и $B$ не доверяют друг другу. Например, $A$ -- банк, $B$ -- получатель денег. $A$ утверждает, что деньги переведены, $B$ утверждает, что не переведены. Решение задачи аутентификации и неотрицаемости состоит в обеспечении \textbf{электронная подписью}\index{электронная подпись} каждого из абонентов. Предварительно надо решить задачу о генерировании и распределении секретных ключей.

В общем случае системы защиты информации должны обеспечивать:
\begin{itemize}
    \item конфиденциальность (защита от наблюдения),
    \item целостность (защита от изменения),
    \item аутентификацию (защита от фальсификации пользователя и сообщений),
    \item доказательство авторства информации (доказательство авторства и защита от его отрицания)
\end{itemize}
как со стороны получателя, так и со стороны отправителя.

Важным критерием для выбора степени защиты является сравнение стоимости реализации взлома для получения информации и экономического эффекта от владения ею. Очевидно, что если стоимость взлома превышает ценность информации, взлом нецелесообразен.

%Сценарии защиты информации
%   Сценарий 1. A -- передающая сторона. B -- принимающая сторона. E -- пассивный
%криптоаналитик, который может подслушивать передачу, но не может вмешиваться
%в процесс передачи. Цель защиты: обеспечение конфиденциальности. Средства
%-- методы шифрования с секретным ключом (симметричные системы шифрования)
%и методы шифрования с открытым ключом (асимметричные системы шифрования).
%Сценарий 2. E -- активный криптоаналитик, который может изменять, удалять и вставлять
%сообщения или их части. Цель защиты -- обеспечение конфиденциальности (не
%всегда) и обеспечение целостности. Средства -- методы шифрования и добавление
%имитовставки\index{имитовставка} (Message Autentication Code -- $\MAC$).
%Сценарий 3. A и B не доверяют друг другу. Цель защиты -- аутентификация пользователя.
%Средства -- электронная подпись.


\section{Минимальные длины ключей}
\selectlanguage{russian}

Оценим минимальную битовую длину ключа для обеспечения криптостойкости, то есть защиты криптосистемы от атаки, полным перебором всех возможных секретных ключей. Сделаем такие предположения:

\begin{itemize}
    \item одно ядро процессора выполняет $R = 10^7 \approx 2^{23}$ шифрований и расшифрований в секунду;
    \item вычислительная сеть состоит из $n = 10^3 \approx 2^{10}$ узлов;
    \item в каждом узле имеется $C = 16 = 2^4$ ядер процессора;
    \item нужно обеспечить защиту данных на $Y = 100$ лет, т.е. на $S \approx 2^{32}$ с;
    \item выполняется закон Мура об удвоении вычислительной производительности на единицу стоимости каждые 2 года, то есть производительность вырастет в $M = 2^{Y/2} \approx 2^{50}$ раз.
\end{itemize}

Число переборов $N$ примерно равно
    \[ N \approx R \cdot n \cdot C \cdot S \cdot M, \]
    \[ N \approx 2^{23} \cdot 2^{10} \cdot 2^{4} \cdot 2^{32} \cdot 2^{50} = 2^{23+10+4+32+50} = 2^{119}. \]

Следовательно, минимально допустимая длина ключа для защиты от атаки перебором на 100 лет составляет порядка
    \[ \log_2 N \approx 119\text{ бит} \] .

Например, в 1997 году предыдущий американский стандарт шифрования DES с 56-битовым секретным ключом впервые был взломан перебором интернет-сетью из 78 000 частных компьютеров, производивших фоновые вычисления по проекту \textsc{DesChal}.


\chapter{Классические шифры}

В главе приведены наиболее известные \emph{классические} шифры, которыми можно было пользоваться до появления роторных машин. К ним относятся такие шифры, как: шифр Цезаря, шифр Плейфера--Витстона, шифр Хилла, шифр Виженера. Они наглядно демонстрируют различные классы шифров.

\section{Моноалфавитные шифры}
\selectlanguage{russian}


Преобразования открытого текста в шифротекст могут быть описаны различными функциями. Если функция преобразования является аддитивной, то и соответствующий шифр называется \textbf{аддитивным}. Если это преобразование является аффинным, то шифр называется \textbf{аффинным}.

\subsection{Шифр Цезаря}

Известным примером простого шифра замены является \textbf{шифр Цезаря}\index{шифр!Цезаря}. Процедура шифрования поясняется с помощью рисунка
%\ref{fig:caesar}
приведенного ниже и состоит в следующем. Записывают все буквы латинского алфавита в стандартном порядке
    \[ A B C D E \dots Z. \]
Делают циклический сдвиг вправо, например на три буквы, и записывают все буквы во втором ряду, начиная с третьей буквы $C$. Буквы первого ряда заменяют соответствующими (как показано стрелкой на рисунке) буквами второго ряда. После такой замены слова не распознаются теми, кто не знает ключа. Ключом $K$ является первый символ сдвинутого алфавита.

\[ \begin{array}{ccccccccc}
    \text{A} & \text{B} & \text{C} & \text{D} & \text{E} & & \text{X} & \text{Y} & \text{Z} \\
    \downarrow & \downarrow & \downarrow & \downarrow & \downarrow & \dots & \downarrow & \downarrow & \downarrow \\
    \text{C} & \text{D} & \text{E} & \text{F} & \text{G} & & \text{Z} & \text{A} & \text{B} \\
\end{array} \]

\example
В русском языке сообщение \textsc{изучайтекриптографию} посредством шифрования с ключом $K = \text{\textsc{г}}$ (сдвиг вправо на 3 символа по алфавиту) преобразуется в \textsc{лкцъгмхзнултхсёугчлб}.
\exampleend

Недостатком любого шифра замены является то, что в шифрованном тексте сохраняются все частоты появления букв открытого текста и корреляционные связи между буквами. Они существуют в каждом языке. Например, в русском языке чаще всего встречаются буквы $A$ и $O$. Для дешифрования криптоаналитик имеет возможность прочитать открытый текст, используя частотный анализ букв шифротекста. Для "взлома" шифра Цезаря достаточно найти одну пару букв -- одну замену.


\subsection{Аддитивный шифр перестановки}

Следующий рисунок
%Рисунок \ref{fig:zamena}
поясняет \textbf{аддитивный шифр} перестановки\index{шифр!аддитивный перестановки} на алфавите. Все 26 букв латинского алфавита нумеруют по порядку от 0 до 25. Затем номер буквы меняют в соответствии с уравнением
    \[ y = x + b \mod 26, \]
где $x$ -- прежний номер, $y$ -- новый номер, $b$ -- заданное целое число, определяющее сдвиг номера и известное только легальным пользователям. Очевидно, что шифр Цезаря является примером аддитивного шифра.
  	
\[ \begin{array}{ccccccccc}
    \text{A} & \text{B} & \text{C} & \text{D} & \text{E} & & \text{X} & \text{Y} & \text{Z} \\
    \downarrow & \downarrow & \downarrow & \downarrow & \downarrow & \dots & \downarrow & \downarrow & \downarrow \\
    2 & 3 & 4 & 5 & 6 & & 25 & 0 & 1 \\
\end{array} \]


\subsection{Аффинный шифр}

Аддитивный шифр является частным случаем \textbf{аффинного шифра}\index{шифр!афинный}. Правило шифрования сообщения имеет вид
    \[ y = a x + b \mod n. \]
Здесь производится умножение номера символа $x$ из алфавита, $x\in \set\{ 0, 1, 2, \dots, N \leq n-1 \}$, на заданное целое число $a$ и сложение с числом $b$ по модулю целого числа $n$. Ключом является $K = (a, b)$.

Расшифрование осуществляется по формуле
    \[ x = (y - b) a^{-1} \mod n. \]

Чтобы обеспечить обратимость в этом шифре, должен существовать единственный обратный элемент $a^{-1}$ по модулю $n$. Для этого должно выполняться условие $\gcd(a,n) = 1$, то есть $a$ и $n$ должны быть взаимно простыми числами ($\gcd$ -- обозначение термина с английского greatest common divisor -- наибольший общий делитель, $\text{НОД}$). Очевидно, что для "взлома" такого шифра достаточно найти две пары букв -- две замены.


\section{Биграммные шифры замены}
\selectlanguage{russian}

Если при шифровании преобразуются по две буквы открытого текста, то такой шифр называется \textbf{биграммным} шифром замены. Первый биграммный шифр был изобретен аббатом Иоганном Трисемусом и опубликован в 1508 году. Другой биграммный шифр изобретен в 1854 году Чарльзом Витстоном. Лорд Лайон Плейфер внедрил этот шифр в государственных службах Великобритании, и шифр был назван шифром Плейфера.

Опишем процедуру шифрования Плейфера~---~Витстона. Заготавливается таблица для английского алфавита (буквы I, J отождествляются), в которую заносятся буквы перемешанного алфавита, например в виде таблицы, представленной ниже. Часто перемешивание алфавита реализуется с помощью начального слова. В нашем примере начальное слово $playfir$. Таблица имеет вид

\begin{center}
    \begin{tabular}{ccccc}
        p & l & a & y & f  \\
        i & r & b & c & d  \\
        e & g & h & k & m  \\
        n & o & q & s & t  \\
        u & v & w & x & z  \\
    \end{tabular}
\end{center}

Буквы открытого текста разбиваются на пары. Правила шифрования каждой пары состоят в следующем.

\begin{itemize}
    \item Если буквы пары не лежат в одной строке или в одном столбце таблицы, то они заменяются буквами, образующими с исходными буквами вершины прямоугольника. Первой букве пары соответствует буква таблицы, находящаяся в том же столбце. Пара букв открытого текста $we$ заменяется двумя буквами таблицы $hu$. Пара букв открытого текста $ew$ заменяется двумя буквами таблицы $uh$.
    \item Если буквы пары открытого текста расположены в одной строке таблицы, то каждая буква заменяется соседней справа буквой таблицы. Например, пара $gk$  заменяется двумя буквами $hm$. Если одна из этих букв -- крайняя правая в таблице, то ее <<правым соседом>> считается крайняя левая в этой строке. Так, пара $to$ заменяется буквами $nq$.
    \item Если буквы пары лежат в одном столбце, то каждая буква заменяется соседней буквой снизу. Например, пара $lo$ заменяется парой $rv$. Если одна из этих букв крайняя нижняя, то ее <<нижним соседом>> считается крайняя верхняя буква в этом столбце таблицы. Например, пара $kx$ заменяется буквами $sy$.
    \item Если буквы в паре одинаковые, то между ними вставляется определенная буква, называемая <<буквой-пустышкой>>. После этого разбиение на пары производится заново.
\end{itemize}

\example
Используем этот шифр и зашифруем сообщение <<Wheatstone was the inventor>> в первой строке таблицы. Получаем после разбиения этой фразы на пары шифрованный текст во второй строке:
\begin{center} \begin{tabular}{|*{12}c|}
    \hline
    wh & ea & ts & to & ne & wa & st & he & in & ve & nt & or \\
    \hline
    aq & ph & nt & nq & un & ab & tn & kg & eu & gu & on & vg \\
    \hline
\end{tabular} \end{center}
\exampleend

Шифр Плейфера--Витстона может быть взломан. Несложно найти ключ, если известны как шифрованный, так и открытый тексты. Если известен только шифрованный текст, то криптоаналитик анализирует соответствие между частотой появления биграмм в шифрованном тексте и известной частотой появления биграмм в языке, на котором написано сообщение. Такой частотный анализ помогает дешифрованию.


\input{hills_cipher}

% \subsection{Омофонные замены}
%
% Омофонными заменами называют криптопримитивы, в основе которых лежит замена групп символов открытого текста $M$ на группу символов $C$ с использованием ключа $K$. Такой метод шифрования вносит неоднозначность между $M$ и $C$, это позволяет защититься от методов частотного криптоанализа.
%  \subsection{шифрокоды}
%  Шифрокоды -- это класс шифров сочетающих в себе свойства кодов и помехозащищенности со свойствами шифра и обеспечения конфиденциальности.

\input{vigeneres_chipher}

\section[Криптоанализ полиалфавитных шифров]{Криптоанализ полиалфавитных \protect\\ шифров}
\selectlanguage{russian}

При дешифровании полиалфавитных шифров криптоаналитику надо сначала определить период, затем преобразовать шифрограмму в матрицу для предполагаемого периода и использовать для каждого столбца методы криптоанализа моноалфавитных шифров. При неудаче надо изменить предполагаемый период.

Известно несколько методов криптоанализа для нахождения периода. Из них наиболее популярными являются метод Касиски, автокорреляционный метод и метод индекса совпадений.


\subsection{Метод Касиски}

Метод Касиски состоит в том, что в шифротексте находят одинаковые сегменты длины не менее трех символов и вычисляют расстояние между начальными символами последовательных сегментов. Далее находят наибольший общий делитель этих расстояний. Считается, что предполагаемый период $n$ является кратным этому значению. Обычно, нахождение периода осуществляется в несколько этапов.
В результате выбирается наиболее правдоподобное значение периода, а затем криптоаналитик переходит к дешифрованию. Приведем пример использования метода Касиски.

\example
Пусть шифруется следующий текст без учета знаков препинания и различия строчных и прописных букв. Пробелы оставлены в тексте для удобства чтения текста, при шифровании пробелы были опущены.

%\begin{center} \begin{minipage}{0.9\textwidth}
    \noindent \textit{Игры различаются по содержанию характерным особенностям а также по тому какое место они занимают в жизни детей их воспитании и обучении Каждый отдельный вид игры имеет многочисленные варианты Дети очень изобретательны Они усложняют и упрощают известные игры придумывают новые правила и детали Например сюжетно ролевые игры создаются самими детьми но при некотором руководстве воспитателя Их основой является самодеятельность Такие игры иногда называют творческими сюжетно ролевыми играми Разновидностью сюжетно ролевой игры являются строительные игры и игры драматизации В практике воспитания нашли свое место и игры с правилами которые создаются для детей взрослыми К ним относятся дидактические подвижные и игры забавы В основе их лежит четко определенное программное содержание дидактические задачи и целенаправленное обучение Для хорошо организованной жизни детей в детском саду необходимо разнообразие игр так как только при этих условиях будет обеспечена детям возможность интересной и содержательной деятельности Многообразие типов видов форм игр неизбежно как неизбежно многообразие жизни которую они отражают как неизбежно многообразие несмотря на внешнюю схожесть игр одного типа модели}
%\end{minipage} \end{center}

Для шифрования выберем период $n=4$ и следующие 4 моноалфавитных шифра замены.

\begin{center} \begin{tabular}{|lcl|}
    \hline
    абвгдежзийклмнопрстуфхцчшщъыьэюя & -- & алфавит \\
    йклмнопрстуфхцчшщъыьэюяабвгдежзи & -- & 1-й шифр \\
    гаэъчфсолиевяьщцурнкздбюышхтпмйж & -- & 2-й шифр \\
    бфзънаужщмятешлюсдчкэргцйьпвхиыо & -- & 3-й шифр \\
    пъерыжсьзтэиуюйфякхалцбмчвншгощд & -- & 4-й шифр \\
    \hline
\end{tabular} \end{center}

Тогда шифрованный текст примет следующий вид (в шифротексте пробелов нет, они вставлены для удобства чтения).

\begin{center} \begin{minipage}{0.9\textwidth}
    \noindent \textit{съсш щгжисюбщыро фч рлыоуупцлы цйубэыфсюдя лкчааюцщдхия б хйеуж шщ чйхк япуща уорчй чьщьйьщуййч еплжюсчахоищцлщдфснбюсл щ йккцжцлщ эйсншт щчыовхюди ззн лъяд лежон еючълмсртжцьвж лгсзйьчш нфчз чюаюе лжйкуахйнаиеьв йцл ккфщуюийч з ьцсйвгых созжъншшо лъяд цсзнкешлгых цщзшо цспллтп с чахйвщ юйцсзхфс кзсахцщ сйффзшо лъяд рльнгыхъж дпхлез нфчгхл шй шущ юоелхчулу щкяйлщнкыэа ечрюзыгчжфж щц чршйлщм длвожыро кйялыожчжфпшйънх хйещж съсш сьлрнг шпртзпзн чечуцжъещус рысоншй щщтжлтез съспхл спрьлесчшйънхщ ъйужыьл ячваечи щрщт оефжыхъж дхщщщховхюдф щрщт щ змув ыщгепылжпялщ е шубэыляж лщдфснбюсж шпбвщ клща уорчй с лъяд р юяйэщийящ эчнлядф дйрчбщыро ыфжнжыфмерулкфтез у ьщу чншйъжчки чщыйечзафдэсф юйнэщсцта з съсш ргфплт з йъьлео лр иосщх афчэч щюяочаиоьшйо цсймубухьлжъщнжщсбюсфнзнгяхсюакула ьйчбмс лгжффшпшубеффшючф лъьюаюсф нии длячыл йщъбюсолейьшйт сщьцл нжыфм е нфчкуще кйчк юощфцччщуч убьцщлъщгжзо лъя ыгя эйе чйфпяй шущоылр аъвлесжр ъьчах чаакшфцжцг нжыже ечоейпьлкып щюыфсжъьлтс рлыоуупыфтгцщм ыожчжфпшйънщуцщъйчаспрла хсцле ллнйл злях лъя цфщькфуюч ебэ цфщькфуючяшймщлъщгжзо сщьцл яйыщсазщшз чнсппгых угяюолжъосшй хьлрчщфяйощжцфдучнсд цгзюоышщзррйпфдхе лъя ччшймщ чзшг ейнфтз}
\end{minipage} \end{center}

Теперь проведем криптоанализ, используя метод Касиски. Предварительно подсчитаем число появлений каждой буквы в шифротексте. Эти данные приведем в таблице, где $i$ в первой строке означает букву алфавита, а $f_{i}$ во второй строке  -- это число появлений этой буквы в шифротексте. Всего в нашем шифротексте имеется $L=1036$ букв.

\begin{center} \resizebox{\textwidth}{!}{ \begin{tabular}{|c||c|c|c|c|c|c|c|c|c|c|c|c|c|c|c|c|}
    \hline
    $i$     & А & Б & В & Г & Д & Е & Ж & З & И & Й & К & Л & М & Н & О & П \\
    $f_{i}$ & 26 & 15 & 11 & 21 & 20 & 36 & 42 & 31 & 13 & 56 & 23 & 70 & 10 & 33 & 36 & 25 \\
    \hline
\end{tabular} } \end{center}

\begin{center} \resizebox{\textwidth}{!}{ \begin{tabular}{|c||c|c|c|c|c|c|c|c|c|c|c|c|c|c|c|c|}
    \hline
    $i$     & Р & С & Т & У & Ф & Х & Ц & Ч & Ш & Щ & Ъ & Ы & Ь & Э & Ю & Я \\
    $f_{i}$ & 28 & 54 & 15 & 36 & 45 & 32 & 31 & 57 & 35 & 72 & 32 & 35 & 27 & 11 & 30 & 28 \\
    \hline
\end{tabular} } \end{center}

В рассматриваемом примере проведенный анализ показал следующее.
\begin{itemize}
    \item Сегмент СЪС встречается в позициях $1, 373, 417, 613$. Соответствующие расстояния равны
        \[ \begin{array}{l}
            373 - 1 = 372 = 4 \cdot 3 \cdot 31, \\
            417 - 373= 44 = 4 \cdot 11, \\
            613 - 417 = 196 = 4 \cdot 49. \\
        \end{array} \]
        Наибольший общий делитель равен $4$. Делаем вывод, что период кратен $4$.
    \item Сегмент ЩГЖ встречается в позициях $5, 781, 941$. Соответствующие расстояния равны
        \[ \begin{array}{l}
            781 - 5 = 776 = 8 \cdot 97, \\
            941 - 781 = 160 = 32 \cdot 5. \\
        \end{array} \]
        Делаем вывод, что период кратен $8$, что не противоречит выводу для предыдущих сегментов (кратность $4$).
    \item Сегмент ЫРО встречается в позициях $13, 349, 557$. Соответствующие расстояния равны
        \[ \begin{array}{l}
            349 - 13 = 336 = 16 \cdot 3 \cdot 7, \\
            557 - 349 = 208 = 16 \cdot 13. \\
        \end{array} \]
        Делаем вывод, что период кратен 4.
\end{itemize}

Предположение о том, что период $n=4$, оказалось правильным.
\exampleend


\subsection{Автокорреляционный метод}

Автокорреляционный метод состоит в том, что исходный шифротекст $C_{1},C_{2},  \ldots, C_{L}$ выписывается в строку, а под ней выписываются строки, полученные сдвигом вправо на $t =1, 2, 3, \ldots$ позиций. Для каждого $t$ подсчитывается число  $n_{t}$ индексов $i \in \left[ {1,L - t} \right]$, таких, что $C_i  = C_{i + t}$.

Вычисляются автокорреляционные коэффициенты
    \[ \gamma_t  = \frac{n_t}{L - t}. \]
Для чисел $t$, кратных периоду, коэффициенты должны быть заметно больше, чем для сдвигов, не кратных периоду.

\example
Для рассматриваемой криптограммы выделим те значения $t$, для которых $\gamma _t  > 0,05.$ Получим ряд чисел:
\begin{center} \begin{minipage}{0.9\textwidth}
    \noindent 4, 12, 16, 24, 28, 32, 36, 40, 44, 48, 52, 56, 64, 68, 72, 76, 80, 84, 88, 92, 96, 104, 108, 112, 116, 124, 128, 132, 140, 148, 152, 156, 160, 164, 168, 172, 176, 180, 184, 188, 192, 196, 200, 204, 208, 216, 220, 224, 228, 252, 256, 260, 264, 268, 272, 276, 280, 284, 288, 292, 296, 300, 304, 308, 312, 316, 320, 324, 328, 344, 348, 356, 364, 368, 372, 376, 380, 384, 388, 396, 400, 404, 408, 412, 420, 424, 432, 436, 440, 448, 452, 456, 460, 462, 468, 472, 476, 480, 484, 496, 500, 508, 512, 516.
\end{minipage} \end{center}
Все эти числа, кроме 462, делятся на 4. Выбираем значение $n=4$, что верно и совпадает со значением, полученным по методу Касиски.
\exampleend


\subsection{Метод индекса совпадений}

При применении метода индекса совпадений подсчитывают число появлений букв в случайной последовательности
    \[ \mathbf{X} = (X_1 ,X_2 ,  \ldots , X_L ) \]
и вычисляют вероятность того, что два случайных элемента этой последовательности совпадают. Эта величина называется индексом совпадений и обозначается $I_{c}(\mathbf{x}),$ где
    \[ I_{c} (\mathbf{x}) = \frac{{\sum\limits_{i = 1}^A {f_i (f_i  - 1)} }} {{L(L - 1)}}, \]
$f_{i}$ -- число появлений буквы $i$  в последовательности $\mathbf{x}$.

Значение этого индекса используется в криптоанализе полиалфавитных шифров для приближенного определения периода по формуле
    \[ m \approx \frac{{k_p  - k_r }} {{I_{c} (\mathbf{x}) - k_r  + \frac{{k_p  - I_{c} (\mathbf{x})}} {L}}}, \]
где
    \[ k_r  = \frac{1}{A}, ~~ k_p  = \sum\limits_{i=1}^A p_i^2, \]
$p_i $ -- частота появления буквы $i$ в естественном языке, $A$ -- число букв в алфавите.
Теоретическое обоснование метода индекса совпадений не является простым. Оно приведено в Приложении \ref{chap:coincide-index} к данному пособию.

\example
В рассматриваемом выше примере приведены значения $f_{i}$. Для русского языка
    \[ A=32, ~ k_{r} = \frac{1}{32} \approx 0.03125, ~ k_{p} \approx 0.0529. \]
Проведя вычисления, получаем $m \approx 3.376$. Так что, полученное по формуле приближенное значение $3.376$ достаточно близко к значению периода $n=4$.
\exampleend

С развитием ЭВМ многие классические полиалфавитные шифры перестали быть устойчивыми к криптоатакам.


\chapter[Совершенная криптостойкость]{Криптосистемы совершенной криптостойкости}
\selectlanguage{russian}

Рассмотрим модель криптосистемы, в которой Алиса выступает источником сообщений $m \in \group{M}$. Алиса использует некоторую функцию шифрования, результатом вычисления которой является шифротекст $c \in \group{C}$:

	\[c = E_{K_1}\left(m\right).\]

Шифротекст $c$ передаётся по открытому каналу легальному пользователю Бобу, причём по пути он может быть перехвачен нелегальным пользователем (криптоаналитиком) Евой.

Боб, обладая ключом расшифрования $K_2$, расшифровывает сообщение с использованием функции расшифрования:
	\[m' = D_{K_2}\left(c \right).\]

Рассмотрим теперь исходное сообщение, передаваемый шифротекст и ключи шифрования (и расшифрования, если они отличаются) в качестве случайных величин, описывая их свойства с точки зрения теории информации. Далее полагаем, что в криптосистеме ключи шифрования и расшифрования совпадают.

Будет называть криптосистему \textit{корректной}, если она обладает следующими свойствами.
\begin{itemize}
	\item Легальный пользователь имеет возможность однозначно восстановить исходное сообщение, то есть
					\[H \left( M | C, K \right) = 0, \]
					\[m' = m.\]
	\item Выбор ключа шифрования не зависит от исходного сообщения
					\[ I \left( K ; M \right) = 0, \]
					\[ H \left( K | M \right) = H \left( K \right). \]
\end{itemize}

Второе условие является, в некотором виде, условием на возможность отделить ключ шифрования от данных и алгоритма шифрования.

\section[Определения]{Определения совершенной криптостойкости}

Понятие совершенной секретности (или стойкости) введено американским ученым Клодом Шенноном. В конце Второй мировой войны он закончил работу, посвященную теории связи в секретных системах\cite{Shannon:1949:CTS}. Эта работа вошла составной частью в собрание его трудов, вышедшее в русском переводе в 1963 году.~\cite{Shannon:1963} Понятие о стойкости шифров по Шеннону связано с решением задачи криптоанализа по одной криптограмме.

Криптосистемы совершенной стойкости могут применяться как в современных вычислительных сетях так и для шифрования любой бумажной корреспонденции. Основной проблемой применения данных шифров для шифрования больших объемов информации является необходимость распространения ключей объемом, не меньшим, чем передаваемые сообщения.

\begin{definition}\label{perfect_by_probabilities}
Будем называть криптосистему \textbf{совершенно криптостойкой}, если апостериорное распределение вероятностей исходного случайного сообщения $m_i \in \group{M}$ при регистрации случайного шифротекста $c_k \in \group{C}$ совпадает с априорным распределением~\cite{Gultyaeva:2010}:

	\[\forall m_j \in \group{M}, c_k \in \group{C}: P \left( m = m_j | c = c_k \right) = P \left( m = m_j \right).\]
\end{definition}

Данное условие можно переформулировать в терминах статистических свойств сообщения, ключа и шифротекста как случайных величин.

\begin{definition}\label{perfect_by_enthropy}
Криптосистема называется совершенно криптостойкой, если условная энтропия сообщения при известном шифротексте равна безусловной:
	\[H \left( M | C \right) = H \left( M \right),\]
	\[I \left( M; C \right) = 0.\]
\end{definition}

Можно показать, что определения \ref{perfect_by_probabilities} и \ref{perfect_by_enthropy} тождественны.

\section[Условие]{Условие совершенной криптостойкости}

Найдем оценку количества информации, которое содержит шифротекст $C$ относительно сообщения $M$
\[ I(M; C) = H(M) - H(M | C). \]
Очевидны следующие соотношения условных и безусловных энтропий \cite{GabPil:2007}:
\[H(K|C)=H(K|C)+H(M|K,C)=H(M,K|C),\]
\[H(M,K|C)=H(M|C)+H(K|M,C)\geq H(M|C),\]
\[H(K)\geq H(K|C)\geq H(M|C).\]
Отсюда получаем:
 \[ I(M; C) = H(M) - H(M | C)\geq H(M)-H(K). \]
Из последнего неравенства следует, что взаимная информация между сообщением и шифротекстом равна нулю, если энтропия ключа не меньше энтропии сообщений. С другой стороны, взаимная информация между сообщением и шифротекстом равна нулю, если они статистически независимы. Таким образом условием совершенной криптостойкости является неравенство
\[ H(M) \leq H(K).\]
%Если утверждение верно, то количество информации в шифротексте относительно открытого текста $I(M; C)$ равно нулю:
%  \[ I(M; C) = H(M) - H(M | C) = 0, \]
%так как для статистически независимых величин условная энтропия равна безусловной энтропии, то есть $H(M) = H(M | C)$.

%Функцию шифрования обозначим $E: \{ M, K \} \rightarrow C$. Процедура шифрования состоит из следующих шагов.
%\begin{itemize}
%    \item Легальный пользователь $A$ выбирает ключ $k \in K$ и секретно сообщает его легальному пользователю $B$ (дополнительная задача -- распределение ключей).
%    \item По открытому сообщению $m \in M$ и выбранному ключу $k$ вычисляют шифрованное сообщение $c = E_k(m) \in C$.
%\end{itemize}

%Основное требование при шифровании состоит в том, чтобы при выбранном ключе $k$ вычисление $c$  было легкой задачей для любого сообщения $m$.

%Функцию расшифрования обозначим $D: \{ C, K \} \rightarrow M$. Процедура расшифрования состоит из следующих шагов.
%\begin{itemize}
%    \item Легальный пользователь $B$ получает от $A$ секретный ключ $k \in K$.
 %   \item $B$ по принятому шифрованному сообщению $c \in C$ и известному ключу $k$ вычисляет открытое сообщение $m = D_k(c) \in M$.
%\end{itemize}

%Основное требование: при выбранном ключе $k$ вычисление $m$ должно быть легкой задачей для любого $c$. С другой стороны, при неизвестном ключе $k$ вычисление открытого сообщения $m$ по известному шифрованному сообщению $c$ должно быть трудной задачей для любого $c$.

%Криптостойкость шифра оценивается числом операций, необходимым для определения: открытого текста $m$ по шифротексту $c$, либо ключа шифрования $k$ по открытому тексту $m$ и шифротексту $c$.

%$M, C, K$ интерпретируются как случайные величины.
%Пусть заданы распределения вероятностей $P_m(M), P_c(C), P_k(K)$. По определению шифрование $C = E_K(M)$ -- детерминированная функция своих аргументов.
%Если при выбранном шифре оказалось, что открытый текст $M$ и шифротекст $C$ -- статистически независимые случайные величины, то считается, что такая система обладает совершенной криптостойкостью.


%\subsection{Длина ключа}

%Пусть сообщения $m\in M$ и ключи $r\in K$ являются независимыми случайными величинами. Это значит, что их совместная вероятность $P_{mk}(M, K)$ равна произведению отдельных вероятностей:
%\[P_{mk}(M, K) = P_m(M) \cdot P_k(K).\]
%Пусть $C = E_K(M)$ -- множество шифрованных текстов, $M = D_K(C)$ -- множество расшифрованных текстов. Можно найти вероятности $P_c(C), P_{mck}(M,C,K)$.

%Используя известные соотношения о безусловной и условной энтропии~\cite{GabPil:2007}, оценим энтропию открытых текстов $M$ с учетом статистической независимости $M$ и $C$:
 %   \[ H(M) = H(M | C) \leq H(MK | C) = H(K | C) + H(M | CK) = \]     \[ = H(K | C) \leq H(K). \]

%Так как энтропия открытого текста при заданном шифротексте и известном ключе равна нулю, то $H(M|CK)=0$. В результате получаем     \[ H(M) \leq H(K). \]

Обозначим $L(M)$ и $L(K)$ длину сообщений и ключа, соответственно. Известно~\cite{GabPil:2007}, что $H(M)\leq L(M)$ и равенство достигается, когда сообщения состоят из статистически независимых и равновероятных символов. Такое же свойство выполняется и для случайных ключей $H(K)\leq L(K)$. Таким образом, достаточным условием совершенной криптостойкости системы можно считать неравенство
 \[ L(M) \leq L(K)\]
при случайном выборе ключа.

%С другой стороны, энтропия открытого текста $H(M)$ характеризует длину последовательности для описания случайной величины $M$ (открытого сообщения), а $H(K)$ характеризует длину последовательности для описания ключа. Следовательно, совершенная криптостойкость возможна только тогда, когда длина ключа не меньше, чем длина шифруемого сообщения, то есть     \[ H(M) \leq H(K). \] Как правило, длина сообщения заранее неизвестна и ограничена большим числом. Выбрать ключ длины не меньшей, чем возможное сообщение не представляется возможным или рациональным, и один и тот же ключ (или его преобразования) используется многократно для шифрования блоков сообщения фиксированной длины. То есть, $H(K) \ll H(M)$.

На самом деле, сообщение может иметь произвольную (заранее не ограниченную) длину. Поэтому генерация и, главным образом, доставка легальным пользователям случайного и достаточного длинного ключа становятся критическими проблемами. Практическим решением этих проблем является многократное использование одного и того же ключа при условии, что его длина гарантирует вычислительную невозможность любой известной атаки на подбор ключа.


\section{Криптосистема Вернама}
\index{криптосистема!Вернама}
Приведем пример системы с совершенной криптостойкостью.

Пусть сообщение представлено двоичной последовательностью длины $N$:
    \[ m = (m_1, m_2, \dots, m_N). \]
Распределение вероятностей сообщений $P_m(m)$ может быть любым. Ключ также представлен двоичной последовательностью $ k = (k_1, k_2, \dots, k_N)$ той же длины, но с равномерным распределением
    \[ P_k(k) = \frac{1}{2^N} \]
для всех ключей.

Шифрование в криптосистеме \textbf{Вернама} осуществляется путем покомпонентного суммирования по модулю алфавита последовательностей открытого текста и ключа:
    \[ C = M \oplus K = (m_1 \oplus k_1, ~ m_2 \oplus k_2, \dots,  m_N \oplus k_N). \]

Легальный пользователь знает ключ и осуществляет расшифрование:
    \[ M =C \oplus K = (m_1 \oplus k_1, ~ m_2 \oplus k_2, \dots, m_N \oplus k_N). \]

Найдем вероятностное распределение $N$-блоков шифротекстов, используя формулу
    \[ P(c = a) = P(m \oplus k = a) = \sum_{m} P(m) P(m \oplus k = a | m) = \]
    \[ = \sum_{m} P(m) P(k \oplus m) = \sum_{m} P(m) \frac{1}{2^N} = \frac{1}{2^N}. \]

Получили подтверждение известного факта: сумма двух случайных величин, одна из которых имеет равномерное распределение, является случайной величиной с равномерным распределением. В нашем случае распределение ключей равномерное, поэтому распределение шифротекстов тоже равномерное.

Запишем совместное распределение открытых текстов и шифротекстов:
    \[ P(m = a, c = b) ~=~ P(m = a) ~ P(c = b | m = a). \]

Найдем условное распределение
    \[ P(c = b | m = a) ~=~ P(m \oplus k = b | m = a) ~= \]
    \[ =~ P(k = b \oplus a | m = a) ~=~ P(k = b \oplus a) ~=~ \frac{1}{2^N}, \]
так как ключ и открытый текст являются независимыми случайными величинами. Итого:
    \[ P(c=b | m=a) = \frac{1}{2^N}. \]

Подстановка правой части этой формулы в формулу для совместного распределения дает
    \[ P(m=a,c=b)=P(m=a)\frac{1}{2^N}, \]
что доказывает независимость шифротекстов и открытых текстов в этой системе. По доказанному выше, количество информации в шифротексте относительно открытого текста равно нулю. Это значит, что рассмотренная криптосистема Вернама обладает совершенной секретностью (криптостойкостью) при условии, что для каждого $N$-блока (сообщения) генерируется случайный (одноразовый) $N$-ключ.

\input{unicity_distance}


\chapter{Блоковые шифры}\label{chapter-block-ciphers}

\section{Введение и классификация}
\selectlanguage{russian}

Блоковые шифры являются основой современной криптографии. Многие криптографические примитивы -- криптографически стойкие генераторы псевдослучайной последовательности (см. главу~\ref{chapter-crypto-random}), криптографические функции хэширования (см. главу~\ref{chapter-hash-functions}) так или иначе основаны на блоковых шифрах. А использование медленной криптографии с открытым ключом было бы невозможно по практическим соображениям без быстрых блоковых шифров.

Блоковые шифры можно рассматривать как функцию преобразования строки фиксированной длины в строку аналогичной длины\footnote{В случае использования недетерминированных алгоритмов, дающих новый результат при каждом шифровании, длина выхода будет больше. Меньше длина выхода быть не может, так как будет невозможно однозначно восстановить произвольное сообщение.} с использованием некоторого ключа, а также соответствующую ей функцию расшифрования:
\[\begin{array}{l}
	C = E_K\left( M \right), \\
	M'= D_K\left( C \right).
\end{array}\]

Данные функции необходимо дополнить требованиями корректности, производительности и надежности. Во-первых, функция расшифрования должна однозначно восстанавливать произвольное исходное сообщение:

\[ \forall k \in \group{K}, m \in \group{M}: D_k \left( E_k\left( m \right) \right) = m. \]

Во-вторых, функции шифрования и расшифрования должны быстро выполняться легальными пользователями (знающими ключ). В-третьих,  должно быть невозможно найти открытый текст сообщения по шифротексту без знания ключа, кроме как полным перебором всех возможных ключей расшифрования. Также, что менее очевидно, надёжная функция блокового шифра не должна давать возможность найти ключ шифрования (расшифрования), даже если злоумышленнику известны пары открытого текста и шифротекста. Последнее свойство защищает от атак на основе известного открытого текста\index{атака!с известным открытым текстом} и на основе известного шифротекста\index{атака!с известным шифротекстом}, а также активно используется при построении криптографических функций хэширования в конструкции Миагучи---Пренеля\index{Миагучи---Пренеля, конструкция}. То есть:
\begin{itemize}
	\item $C = f \left( M, K \right)$ и $M = f \left( C, K \right)$ должны вычисляться быстро (легальные операции);
	\item $M = f \left( C \right)$ и $C = f \left( M \right)$ должны вычисляться не быстрее, чем $\left| \group{K} \right|$ операций расшифрования (шифрования), при условии, что злоумышленник может отличить корректное сообщение (см. выводы к разделу~\ref{section_unicity_distance});
	\item $K = f \left( M, C \right)$ должно вычисляться не быстрее, чем $\left| \group{K} \right|$ операций шифрования;
\end{itemize}

Если размер ключа достаточно большой (от 128 бит и выше), то функцию блокового шифрования, удовлетворяющую указанным выше условиям, можно назвать надёжной.

Блоковые шифры делят на два больших класса по методу построения.
\begin{itemize}
	\item Шифры, основанные на SP-сетях (сети замены-перестановки), основанные на \textit{обратимых} преобразованиях с открытым текстом. При разработке таких шифров криптограф должен следить за тем, чтобы каждая из производимых операций была и криптографически надёжна, и обратима при знании ключа.
	\item Шифры, в той или иной степени построенные на ячейке Фейстеля. В данных шифрах используется конструкция под названием <<ячейка Фейстеля>>, которая по методу построения уже обеспечивает обратимость операции шифрования легальным пользователем при знании ключа. Криптографу при разработке функции шифрования остаётся сосредоточиться на надёжности конструкции.
\end{itemize}

\begin{figure}[!ht]
	\centering
	\includegraphics[width=1\textwidth]{pic/block-cipher}
  \caption{Общая структура блокового шифра. С помощью функции ключевого расписания из ключа $K$ получается набор раундовых ключей $K1, K2, \dots$. Открытый текст $M$ разбивается на блоки $M1, M2, \dots$, каждый из которых проходит несколько раундов шифрования, используя соответствующие раундовые ключи. Результаты последних раундов шифрования каждого из блоков объединяются в шифротекст $C$ с помощью одного из режима сцепления блоков}
  \label{fig:block-cipher}
\end{figure}

Все современные блоковые шифры являются \textit{раундовыми}. То есть блок текста проходит через несколько одинаковых (или похожих) преобразований, называемых \textit{раундами шифрования}. У~функции шифрования также может существовать начальный и завершающий раунды, отличающиеся от остальных (обычно -- отсутствием некоторых преобразований, которые не имеют смысла для <<крайних>> раундов).

Аргументами каждого раунда является результат предыдущего раунда (для самого первого -- часть открытого текста) и \textit{раундовый ключ}\index{ключ!раундовый}. Раундовые ключи получаются из оригинального ключа шифрования с помощью процедуры, получившей название расписание ключей\index{расписание ключей} (или же ключевое расписание\index{ключевое расписание}, англ. \textit{key schedule}). Функция ключевого расписания является важной частью блокового шифра. На потенциальной слабости этой функции основаны такие криптографические атаки, как атака на основе связанных ключей\index{атака!на связанных ключах} и атака скольжения\index{атака!скольжения}.

После прохождения всех раундов шифрования блоки $C1, C2, \dots$ объединяются в шифротекст $C$ с помощью одного из режимов сцепления блоков (см. раздел~\ref{chapter-block-chaining}). Простейшим примером режима сцепления блоков является режим электронной кодовой книги\index{режим!электронной кодовой книги}, когда блоки $C1, C2, \dots$ просто конкатенируются в шифротекст $C$ без дополнительной обработки.

К числовым характеристикам блокового шифра относят:
\begin{itemize}
	\item размер входного и выходного блока;
	\item размер ключа шифрования;
	\item количество раундов.
\end{itemize}

Также надёжные блоковые шифры обладают <<лавинным эффектом>>\index{лавинный эффект} (англ. \textit{avalanche effect}): изменение одного бита в блоке открытого текста или ключа приводит к полному изменению соответствующего блока шифротекста.


\input{lucifer}

\input{Feistel_cipher}

\section{Российский стандарт шифрования ГОСТ 28147-89}
\selectlanguage{russian}

Стандарт шифрования \textbf{ГОСТ 28147-89} \cite{GOST-89} относится к действующим симметричным одноключевым криптографическим алгоритмам. Он зарегистрирован 2 июня 1989 года и введен в действие Постановлением Государственного комитета СССР по стандартам от 02.06.89 № 1409.
%Дата актуализации описания 01 февраля 2008 года, дата актуализации текста 15 марта 2009 года.
Последнее изменение внесено в алгоритм 13 марта 2007 года.
ГОСТ 28147-89 устанавливает единый алгоритм криптографических преобразований для систем обмена информацией в вычислительных сетях и определяет правила шифрования и расшифрования данных, а также выработки имитовставки\index{имитовставка}. Основные параметры шифра таковы: размер блока составляет 64 бита, число раундов $m=32$, имеется 8 ключей по 32 бита каждый, так что общая длина ключа 256 бит. Основа алгоритма -- цепочка ячеек Фейстеля.

\begin{figure}[!ht]
    \centering
    \includegraphics[width=0.6\textwidth]{pic/gost-28147-89}
    \caption{Схема ГОСТ 28147-89\label{fig:gost-28147-89}}
\end{figure}

Структурная схема алгоритма шифрования представлена на рисунке \ref{fig:gost-28147-89} и включает
\begin{itemize}
    \item ключевое запоминающее устройство (КЗУ) на 256 бит, которое состоит из восьми 32-разрядных накопителей $(X_0, X_1, X_2, X_3, X_4, X_5, X_6, X_7)$ и содержит сеансовые ключи шифрования одного раунда;
    \item 32-разрядный сумматор $\boxplus$ по модулю $2^{32}$;
    \item сумматор $\oplus$ по модулю 2;
    \item блок подстановки $(S)$;
    \item регистр циклического сдвига на одиннадцать шагов в сторону старшего разряда  $(R)$.
\end{itemize}

Блок подстановки $(S)$ состоит из 8 узлов замены -- $s$-блоков с памятью на 64 бита каждый. Поступающий на вход блока подстановки 32-разрядный вектор разбивается на восемь последовательных 4-разрядных векторов, каждый из которых преобразуется в 4-разрядный вектор соответствующим узлом замены. Узел замены представляет собой таблицу из шестнадцати строк, содержащих по четыре бита в строке. Входной вектор определяет адрес строки в таблице, заполнение данной строки является выходным вектором. Затем 4-разрядные выходные векторы последовательно объединяются в 32-разрядный вектор.

При перезаписи информации содержимое $i$-го разряда одного накопителя переписывается в $i$-й разряд другого накопителя.

Ключ, определяющий заполнение КЗУ, и таблицы блока подстановки $K$ являются секретными элементами.

Стандарт не накладывает ограничений на степень секретности защищаемой информации.

ГОСТ 28147-89 удобен как для аппаратной, так и для программной реализации.

Алгоритм имеет четыре режима работы. Из них первые три -- режимы шифрования, а  последний -- генерирования имитовставки\index{имитовставка} (другие названия: инициализирующий вектор, синхропосылка):
\begin{itemize}
    \item простой замены;
    \item гаммирования;
    \item гаммирования с обратной связью;
    \item выработки имитовставки\index{имитовставка}.
\end{itemize}


Подробно данные режимы описаны в следующем разделе.


\section{Стандарт шифрования США AES}
\selectlanguage{russian}

До 2001 г. стандартом шифрования данных в США был DES\index{шифр!DES} (аббревиатура от Data Encryption Standard), который был принят в 1980 году. Входной блок открытого текста и выходной блок шифрованного текста DES составляли по 64 бита каждый, длина ключа -- 56 бит (до процедуры расширения). Алгоритм основан на ячейке Фейстеля\index{ячейка Фейстеля} с $s$-блоками и таблицами расширения и перестановки бит. Количество раундов -- 16.

Для повышения криптостойкости и замены стандарта DES был объявлен конкурс на новый стандарт AES (аббревиатура от Advanced Encryption Standard). Победителем конкурса стал шифр Rijndael. Название составлено с использованием первых слогов фамилий его создателей (Rijmen and Daemen). В русскоязычном варианте читается как <<Рэндал>>~\cite{Kiwi:1999}. Шифр был утвержден в качестве стандарта FIPS 197 в ноябре 2001 г. и введён в действие 26 мая 2002 года~\cite{FIPS-PUB-197}.

AES -- это раундовый\index{шифр!раундовый} блоковый\index{шифр!блоковый} шифр с переменной длиной ключа (128, 192 или 256 бит) и фиксированной длиной входного и выходного блоков (128 бит).

\subsection[Состояние, ключ шифрования и число раундов]{Состояние, ключ шифрования и число \protect\\ раундов}

Различные преобразования воздействуют на результат промежуточного шифрования, называемый \textit{состоянием} ($\mathsf{State}$). Состояние представлено $(4 \times 4)$-матрицей из байт.

\textit{Ключ шифрования раунда} ($\mathsf{Key}$) также представляется прямоугольной $(4 \times \mathsf{Nk})$-матрицей из байт $k_{i,j}$, где $\mathsf{Nk}$ равно длине ключа, разделенной на 32, то есть 4, 6 или 8.

Эти представления приведены ниже.
\[
    \mathsf{State} = \left[ \begin{array}{cccc}
        a_{0,0} & a_{0,1} & a_{0,2} & a_{0,3} \\
        a_{1,0} & a_{1,1} & a_{1,2} & a_{1,3} \\
        a_{2,0} & a_{2,1} & a_{2,2} & a_{2,3} \\
        a_{3,0} & a_{3,1} &a_{3,2} & a_{3,3}  \\
    \end{array} \right],
\] \[
    \mathsf{Key} = \left[ \begin{array}{cccc}
        k_{0,0} & k_{0,1} & k_{0,2} & k_{0,3} \\
        k_{1,0} & k_{1,1} & k_{1,2} & k_{1,3} \\
        k_{2,0} & k_{2,1} & k_{2,2} & k_{2,3} \\
        k_{3,0} & k_{3,1} & k_{3,2} & k_{3,3} \\
    \end{array} \right].
\]

Иногда блоки символов интерпретируются как одномерные последовательности из 4-байтных векторов, где каждый вектор является соответствующим столбцом прямоугольной таблицы. В этих случаях таблицы можно рассматривать как наборы из 4, 6 или 8 векторов, нумеруемых в диапазоне $0 \dots 3, 0 \dots 5$ или $0 \dots 7$. Сами 4-байтые векторы называют словами. В тех случаях, когда нужно пометить индивидуальный байт внутри 4-байтого вектора или слова, используется обозначение $(a, b, c, d)$, где $a, b, c, d$ соответствуют байтам в одной из позиций $0, 1, 2, 3$ в столбце, векторе или слове.

\textit{Входные} и \textit{выходные} блоки шифра AES рассматриваются как последовательности 16 байт $(a_0, a_1, \dots, a_{15})$. Преобразование входного блока $(a_0, \dots, a_{15})$ в исходную $(4 \times 4)$ матрицу состояния $\mathsf{State}$ или конечной матрицы состояния в выходную последовательность проводится по правилу (запись по столбцам):
    \[ a_{i,j} = a_{i + 4j}, ~ i = 0 \dots 3, ~ j = 0 \dots 3. \]

Аналогично ключ шифрования может рассматриваться как последовательность байт $(k_0, k_1, \dots, k_{4 \cdot \mathsf{Nk} - 1})$, где $\mathsf{Nk} = 4, 6, 8$. Число байт в этой последовательности равно 16, 24 или 32, а номера этих байт находятся в интервалах $0 \dots 15, ~ 0 \dots 23$ или $0 \dots 31$ соответственно. $(4 \times \mathsf{Nk})$-матрица ключа шифрования $\mathsf{Key}$ задается по правилу:
    \[ k_{i,j} = k_{i + 4j}, ~ i = 0 \dots 3, ~ j = 0 \dots \mathsf{Nk} - 1. \]

Число раундов $\mathsf{Nr}$ зависит от длины ключа. Его значения приведены в таблице ниже.

\begin{center}
    \begin{tabular}{|l|c|c|c|}
    \hline
    Длина ключа, биты           &128 & 192 & 256 \\
    $\mathsf{Nk}$               & 4  & 6   & 8 \\
    Число раундов $\mathsf{Nr}$ & 10 & 12 & 14 \\
    \hline
    \end{tabular}
\end{center}


\subsection{Операции в поле}

При переходе от одного раунда к другому матрицы \textit{состояния} и \textit{ключа шифрования раунда} подвергаются ряду преобразований. Преобразования могут осуществляться над:
\begin{itemize}
    \item отдельными байтами или парами байт (необходимо определить операции сложения и умножения);
    \item столбцами матрицы, которые рассматриваются как 4-мерные векторы с соответствующими байтами в качестве элементов;
    \item строками матрицы.
\end{itemize}

В алгоритме шифрования AES байты рассматриваются как элементы поля $\GF{2^8}$, а вектор-столбцы из четырех байт -- как многочлены третьей степени над полем $\GF{2^8}$. В Приложении \ref{chap:discrete-math} дано подробное описание этих операций.

Хотя определение операций дано через их математическое представление, в реализациях шифра AES активно используются таблицы с заранее вычисленными результатами операций над отдельными байтами, включая взятие обратного элемента и перемножение элементов в поле $\GF{2^8}$ (на что требуется 256 байт и 65 Кбайт памяти соответственно).

\subsection{Операции одного раунда шифрования}

В каждом раунде шифра AES, кроме последнего раунда, производятся следующие 4 операции:
\begin{itemize}
  \item замена байт, $\mathsf{SubBytes}$;
  \item сдвиг строк, $\mathsf{ShiftRows}$;
  \item перемешивание столбцов, $\mathsf{MixColumns}$;
  \item добавление текущего ключа, $\mathsf{AddRoundKey}$.
\end{itemize}

В последнем раунде исключается операция <<перемешивание столбцов>>. В обозначениях, близких к языку С, можно записать программу в следующем виде:
\[
    \begin{array}{l}
        \mathsf{Round(State, RoundKey)} \{ \\
        ~~~~ \mathsf{SubBytes(State)}; \\
        ~~~~ \mathsf{ShiftRows(State)}; \\
        ~~~~ \mathsf{MixColumns(State)}; \\
        ~~~~ \mathsf{AddRoundKey(State, RoundKey)}; \\
        \} \\
    \end{array}
\]
Последний раунд слегка отличается, и его можно записать в следующем виде:
\[
    \begin{array}{l}
        \mathsf{Round(State, RoundKey)} \{ \\
        ~~~~ \mathsf{SubBytes(State)}; \\
        ~~~~ \mathsf{ShiftRows(State)}; \\
        ~~~~ \mathsf{AddRoundKey(State, RoundKey)}; \\
        \} \\
    \end{array}
\]
В этих обозначениях все <<функции>>, а именно: $\mathsf{Round}$, $\mathsf{SubBytes}$, $\mathsf{ShiftRows}$, $\mathsf{MixColumns}$ и $\mathsf{AddRoundKey}$ -- воздействуют на матрицы, определяемые указателем $\mathsf{(State, RoundKey)}$. Сами преобразования описаны в следующих разделах.


\subsubsection{Замена байт $\mathsf{SubBytes}$}

Нелинейная операция <<замена байт>> действует независимо на каждый байт $a_{i,j}$ текущего состояния. Таблица замены (или $s$-блок) является обратимой и формируется последовательным применением двух преобразований.

\begin{enumerate}
    \item Сначала байт $a$ представляется как элемент $a(x)$ поля Галуа $\GF{2^8}$ и заменяется на обратный элемент $a^{-1} \equiv a^{-1}(x)$ в поле. Байт $\mathrm{'00'}$, для которого обратного элемента не существует, переходит сам в себя.
    \item Затем к обратному байту $a^{-1} = (x_0, x_1, x_2, x_3, x_4, x_5, x_6, x_7)$ применяется аффинное преобразование над полем $\GF{2}$ следующего вида:
        \[
            \left[  \begin{array}{c}
                y_{0} \\ y_{1} \\ y_{2} \\ y_{3} \\ y_{4} \\ y_{5} \\ y_{6} \\ y_{7} \\
            \end{array} \right] = \left[ \begin{array}{cccccccc}
                1 & 0 & 0  & 0 & 1 & 1 & 1 & 1 \\
                1 & 1 & 0  & 0 & 0 & 1 & 1 & 1 \\
                1 & 1 & 1  & 0 & 0 & 0 & 1 & 1 \\
                1 & 1 & 1  & 1 & 0 & 0 & 0 & 1 \\
                1 & 1 & 1  & 1 & 1 & 0 & 0 & 0 \\
                0 & 1 & 1  & 1 & 1 & 1 & 0 & 0 \\
                0 & 0 & 1  & 1 & 1 & 1 & 1 & 0 \\
                0 & 0 & 0  & 1 & 1 & 1 & 1 & 1  \
            \end{array} \right] \cdot \left[ \begin{array}{c}
                x_{0} \\ x_{1} \\ x_{2} \\ x_{3} \\ x_{4} \\ x_{5} \\ x_{6} \\ x_{7} \\
            \end{array} \right] + \left[ \begin{array}{c}
                1 \\ 1 \\ 0 \\ 0 \\ 0 \\ 1 \\ 1 \\ 0 \\
            \end{array} \right].
        \]
\end{enumerate}

В полиномиальном представлении это аффинное преобразование имеет вид:
\[Y(z)=(z^4)X(z)(1+z+z^2+z^3+z^4)\mod(1+z^8) + F(z).\]
Применение описанных операций $s$-блока ко всем байтам текущего состояния обозначено
    \[ \mathsf{SubBytes(State)}. \]

Обращение операции $\mathsf{SubBytes(State)}$ также является заменой байт. Сначала выполняется обратное аффинное преобразование, а затем от полученного байта берется обратный.


\subsubsection{Сдвиг строк $\mathsf{ShiftRows}$}

Для выполнения операции <<сдвиг строк>> строки в таблице текущего состояния циклически сдвигаются влево. Величина сдвига различна для различных строк. Строка $0$ не сдвигается вообще. Строка $1$ сдвигается на $C_1=1$ позицию, строка $2$ –- на $C_2=2$ позиции, строка $3$ -– на $C_3=3$ позиции.
%Величины $C1,C2$ и $C3$ зависят от $Nb$. Их значения приведены в табл. \ref{tab:AES-shift-rows}.
%
%\begin{table}[!ht]
%    \centering
%    \begin{tabular}{|c|c|c|c|}
%        \hline
%        Nb & C1 & C2 & C3 \\
%        \hline
%        4  & 1  & 2  & 3  \\
%        \hline
%        6  & 1  & 2  & 3  \\
%        \hline
%        8  & 1  & 3  & 4  \\
%        \hline
%    \end{tabular}
%    \caption{Сдвиг $C$ и длина блока $Nb$.}
%    \label{tab:AES-shift-rows}
%\end{table}


\subsubsection{Перемешивание столбцов $\mathsf{Mix Columns}$}

При выполнении операции <<перемешивание столбцов>> столбцы матрицы текущего состояния рассматриваются как многочлены над полем $\GF{2^8}$ и умножаются по модулю многочлена $y^4 +1$ на фиксированный многочлен $\mathbf{c}(y)$, где
    \[ \mathbf{c}(y) = \mathrm{'03'} y^3 + \mathrm{'01'} y^2 + \mathrm{'01'} y + \mathrm{'02'}. \]
Этот многочлен взаимно прост с многочленом $y^4 + 1$ и, следовательно, обратим. Перемножение удобнее проводить в матричном виде. Если $\mathbf{b}(y) = \mathbf{c}(y) \otimes \mathbf{a}(y)$, то
\[
    \left[ \begin{array}{c}
        b_{0} \\ b_{1} \\ b_{2} \\ b_{3} \\
    \end{array}\right] =  \left[ \begin{array}{cccc}
        \mathrm{'02'} & \mathrm{'03'} & \mathrm{'01'} & \mathrm{'01'} \\
        \mathrm{'01'} & \mathrm{'02'} & \mathrm{'03'} & \mathrm{'01'} \\
        \mathrm{'01'} & \mathrm{'01'} & \mathrm{'02'} & \mathrm{'03'} \\
        \mathrm{'03'} & \mathrm{'01'} & \mathrm{'01'} & \mathrm{'02'} \\
    \end{array} \right] \cdot \left[ \begin{array}{c}
        a_{0} \\ a_{1} \\ a_{2} \\ a_{3} \\
     \end{array} \right].
\]

Обратная операция состоит в умножении на многочлен $\mathbf{d}(y)$, обратный многочлену $\mathbf{c}(y)$ по модулю $y^4 + 1$, то есть
\[
    (\mathrm{'03'} y^{3} + \mathrm{'01'} y^{2} + \mathrm{'01'} y + \mathrm{'02'}) \otimes \mathbf{d}(y) = \mathrm{'01'}.
\]
Этот многочлен равен
\[
    \mathbf{d}(y) = \mathrm{'0B'} y^3 + \mathrm{'0D'} y^2 + \mathrm{'09'} y + \mathrm{'0E'}.
\]


\subsubsection{Добавление ключа раунда $\mathsf{AddRoundKey}$}

Операция <<Добавление ключа раунда>> состоит в том, что к матрице текущего состояния добавляется по модулю $2$ матрица ключа текущего раунда. Обе матрицы должны иметь одинаковые размеры. Матрица ключа раунда вычисляется с помощью процедуры \emph{расширения ключа}, описанной ниже. Операция <<Добавление ключа раунда>> обозначается $\mathsf{AddRoundKey(State, RoundKey)}$.

\[
    \left[ \begin{array}{cccc}
        a_{0,0} & a_{0,1} & a_{0,2} & a_{0,3} \\
        a_{1,0} & a_{1,1} & a_{1,2} & a_{1,3} \\
        a_{2,0} & a_{2,1} & a_{2,2} & a_{2,3} \\
        a_{3,0} & a_{3,1} & a_{3,2} & a_{3,3}
    \end{array} \right]
    \oplus
    \left[ \begin{array}{cccc}
        k_{0,0} & k_{0,1} & k_{0,2} & k_{0,3} \\
        k_{1,0} & k_{1,1} & k_{1,2} & k_{1,3} \\
        k_{2,0} & k_{2,1} & k_{2,2} & k_{2,3} \\
        k_{3,0} & k_{3,1} & k_{3,2} & k_{3,3}
    \end{array} \right] =
\] \[
    = \left[ \begin{array}{cccc}
        b_{0,0} & b_{0,1} & b_{0,2} & b_{0,3} \\
        b_{1,0} & b_{1,1} & b_{1,2} & b_{1,3} \\
        b_{2,0} & b_{2,1} & b_{2,2} & b_{2,3} \\
        b_{3,0} & b_{3,1} & b_{3,2} & b_{3,3}
    \end{array} \right].
\]


\subsection{Процедура расширения ключа}

Матрица ключа текущего раунда получается из исходного ключа шифра с помощью специальной процедуры, состоящей из расширения ключа и выбора раундового ключа. Основные принципы этой процедуры состоят в следующем.
\begin{itemize}
    \item Суммарная длина ключей всех раундов равна длине блока, умноженной на увеличенное на 1 число раундов. Для блока длины 128 бит и 10 раундов общая длина всех ключей раундов равна 1408.
    \item С помощью ключа шифра находят \textit{расширенный ключ}.
    \item Ключи \emph{раунда} выбираются из \emph{расширенного} ключа по правилу: ключ первого раунда состоит из первых 4-х столбцов матрицы расширенного ключа, второй ключ –- из следующих 4-х столбцов и т.д.
\end{itemize}

Расширенный ключ –- это матрица $\mathsf{W}$, состоящая из $4 \cdot (\mathsf{Nr} + 1)$ 4-байтых вектор-столбцов, каждый столбец $i$ обозначается $\mathsf{W}[i]$.

Далее рассматривается только случай, когда ключ шифра состоит из $16$ байт. Первые $\mathsf{Nk} = 4$ столбцов содержат ключ шифра. Остальные столбцы вычисляются рекурсивно из столбцов с меньшими номерами.

Для $\mathsf{Nk} = 4$ имеем 16-байтый ключ
\[
    \mathsf{Key} = (\mathsf{Key}[0], \mathsf{Key}[1], \dots, \mathsf{Key}[15]).
\]
Приведем алгоритм расширения ключа для $\mathsf{Nk} = 4$.
\begin{algorithm}[iht]
    \caption{$\mathsf{KeyExpansion}(\mathsf{Key}, \mathsf{W})$\label{alg:AES-key-exp}}
    \begin{algorithmic}
        \FOR{ $i=0$ \TO $\mathsf{Nk} - 1$}
            \STATE $\mathsf{W}[i] = (\mathsf{Key}[4i], ~ \mathsf{Key}[4i+1], ~ \mathsf{Key}[4i+2], ~ \mathsf{Key}[4i+3])^T$;
        \ENDFOR
        \FOR{ $i = \mathsf{Nk}$ \TO $4 \cdot (\mathsf{Nr} + 1) - 1$}
            \STATE $\mathsf{temp} = \mathsf{W}[i-1]$;
            \IF{ ($i = 0 \mod \mathsf{Nk}$)}
                \STATE $\mathsf{temp} = \mathsf{SubWord}(\mathsf{RotWord}(\mathsf{temp})) ~ \oplus ~ \mathsf{Rcon}[i / \mathsf{Nk}]$;
            \ENDIF
            \STATE $\mathsf{W}[i] = \mathsf{W}[i - \mathsf{Nk}] ~ \oplus ~ \mathsf{temp}$;
        \ENDFOR
    \end{algorithmic}
\end{algorithm}

%\[
%    \begin{array}{l}
%        \mathsf{KeyExpansion}(\mathsf{Key}, \mathsf{W}) \{ \\
%        ~~~~ \mathsf{for ~ (i = 0; ~ i < Nk = 4; ~ i++)} \\
%        ~~~~~~~~ \mathsf{W[i] = (Key[4 \cdot i], ~ Key[4*i+1], ~ Key[4*i+2], ~ Key[4*i+3]);} \\
%        ~~~~ \mathsf{for ~ (i = Nk; ~ i < 4 * (Nr + 1); ~ i++)} ~ \{ \\
%        ~~~~~~~~ \mathsf{temp = W[i-1];} \\
%        ~~~~~~~~ \mathsf{if ~ (i ~ \% ~ Nk ~ == ~ 0)} \\
%        ~~~~~~~~~~~~ \mathsf{temp = SubWord(RotWord(temp))} ~ \oplus ~ \mathsf{Rcon[i / Nk];} \\
%        ~~~~~~~~ \mathsf{W[i] = W[i - Nk]} ~ \oplus ~ \mathsf{temp;} \\
%        ~~~~ \} \\
%        \} \\
%    \end{array}
%\]

Здесь $\mathsf{SubWord}(\mathsf{W})[i]$ обозначает функцию, которая применяет операцию <<замена байт>> (или s-блок) $\mathsf{SubBytes}$ к каждому из 4-х байт столбца $\mathsf{W}[i]$. Функция $\mathsf{RotWord}(\mathsf{W}[i])$  осуществляет циклический сдвиг вверх байт столбца $\mathsf{W}[i]$: если $\mathsf{W}[i] = (a, b, c, d)^T$, то $\mathsf{RotByte}(\mathsf{W}[i]) = (b, c, d, a)^T$. Векторы-константы $\mathsf{Rcon}[i]$ определены ниже.

Как видно из этого описания, первые $\mathsf{Nk} = 4$ столбцов заполняются ключом шифра. Все следующие столбцы $\mathsf{W}[i]$ равны сумме по модулю 2 предыдущего столбца $\mathsf{W}[i-1]$ и столбца $\mathsf{W}[i-4]$. Для столбцов $\mathsf{W}[i]$ с номерами $i$, кратными $\mathsf{Nk} = 4$, к столбцу $\mathsf{W}[i-1]$ применяются операции $\mathsf{RotWord(W)}$ и $\mathsf{SubWord(W)}$, а затем производится суммирование по модулю 2 со столбцом $\mathsf{W}[i-4]$ и константой раунда $\mathsf{Rcon}[i ~/~ 4]$.

%Для $\mathsf{Nk}>6$ имеем
%\[
%\begin{array}{l}
% \mathsf{KeyExpansion\,(byte\,Key\,[4*Nk]\,\, word \,\, W[Nb*(Nr+1)])}\\
%  \{\\
% \quad\quad \mathsf{for\,\,(i=0;\,\, i<Nk;\,\,i++)} \\
%  \qquad \quad\quad\quad \mathsf{W[i]=(Key[4*i];Key[4*i+1];Key[4*i+2];Key[4*i+3]);}\\
%  \quad\quad \mathsf{for \,\,(i=Nk;\,\,i<Nb*(Nr+1);\,\,i++)}\\
%  \quad\quad \{ \\
%  \quad \quad\quad\quad \mathsf{temp=W[i-1]}; \\
%  \quad \quad\quad\quad \mathsf{if\,\,(i\quad\% \quad Nk==0)}\\
%  \qquad \qquad \qquad \quad \mathsf{temp=SubByte(RotByte(temp))\quad\widehat{\,}\quad Rcon[i/Nk]};\\
%\quad \quad\quad\quad \mathsf{else \,\,if\,\,(i\quad\% \quad Nk==4)}\\
% \qquad \qquad \qquad \quad \mathsf{temp=SubByte(temp)};\\
%  \quad \quad\quad\quad \mathsf{W[i]=W[i-Nk] \quad\widehat{\,}\quad temp};\\
%  \quad\quad \} \\
%  \}\, \\
%\end{array}
%\]
%Различие между этими двумя случаями состоит в том, что во втором случае к столбцу $\mathsf{W[i-1]}$ применяются операции
% $\mathsf{RotByte(W)}$ и $\mathsf{SubByte(W)}$, если $\mathsf{i-4}$ кратно $\mathsf{Nk}$.\\

Константы раундов определяются следующим образом:
    \[ \mathsf{Rcon}[i] = (\mathsf{RC}[i], \mathrm{'00'}, \mathrm{'00'}, \mathrm{'00'})^T, \]
где байт $\mathsf{RC}[1]$ равен $\mathsf{RC}[1] = \mathrm{'01'}$, а байты $\mathsf{RC}[i] = \alpha^{i-1}, ~ i = 2, 3, \dots$. Байт $\alpha = \mathrm{'02'}$ –- это примитивный элемент поля $\GF{2^8}$.

\example
Пусть $\mathsf{Nk} = 4$. В этом случае ключ шифра имеет длину 128 бит. Найдем столбцы расширенного ключа. Столбцы $\mathsf{W}[0], \mathsf{W}[1], \mathsf{W}[2], \mathsf{W}[3]$ непосредственно заполняются битами ключа шифра. Номер следующего столбца $\mathsf{W}[4]$ кратен $\mathsf{Nk}$, поэтому
\[
    \mathsf{W}[4] = \mathsf{SubWord}(\mathsf{RotWord}(\mathsf{W}[3])) \oplus \mathsf{W}[0] \oplus
        \left[ \begin{array}{c}
            \mathrm{'01'} \\ \mathrm{'00'} \\ \mathrm{'00'} \\ \mathrm{'00'} \\
        \end{array} \right].
\]
Далее имеем
\[
    \begin{array}{l}
        \mathsf{W}[5] = \mathsf{W}[4] \oplus \mathsf{W}[1], \\
        \mathsf{W}[6] = \mathsf{W}[5] \oplus \mathsf{W}[2], \\
        \mathsf{W}[7] = \mathsf{W}[6] \oplus \mathsf{W}[3].  \\
    \end{array}
\]
Затем
\[
    \mathsf{W}[8] = \mathsf{SubWord}(\mathsf{RotWord}(\mathsf{W}[7])) \oplus \mathsf{W}[4] \oplus
        \left[ \begin{array}{c}
            \alpha \\
            \mathrm{'00'}\\
            \mathrm{'00'}\\
            \mathrm{'00'}\\
        \end{array} \right] ,
\] \[
    \begin{array}{l}
        \mathsf{W}[9] = \mathsf{W}[8] \oplus \mathsf{W}[5], \\
        \mathsf{W}[10] = \mathsf{W}[9] \oplus \mathsf{W}[6], \\
        \mathsf{W}[11] = \mathsf{W}[10] \oplus \mathsf{W}[7] \\
    \end{array}
\]
и т.д.
\exampleend

%\example
%Пусть $\mathsf{Nk=6}.$ В этом случае ключ шифра имеет длину 192 бита. Найдем столбцы расширенного ключа. Столбцы $\mathsf{W[0],W[1],W[2],W[3],W[4],W[5]}$ непосредственно заполняются
%битами ключа шифра. Номер следующего столбца $\mathsf{W[6]}$ кратен $\mathsf{Nk}$, поэтому
%\[
%\begin{array}{ccccccc}
% \mathsf{W[6]} & = & \mathsf{SubByte(RotByte(W[5]))} &\oplus  &  \mathsf{W[0]} & \oplus  & \left[ \begin{array}{c}
% \mathsf{`01'} \\
%  \mathsf{`00'}\\
%  \mathsf{`00'}\\
%  \mathsf{`00'}\\
%\end{array}
%\right]    \\
%\end{array}
%\].
%
%Далее имеем
%\[
%\begin{array}{ccc}
% \mathsf{W[7]=W[6]}\oplus \mathsf{W[1]}; & \mathsf{W[8]=W[7]}\oplus \mathsf{W[2]}; & \mathsf{W[9]=W[8]}\oplus \mathsf{W[3]}; \\
% \mathsf{W[10]=W[9]}\oplus \mathsf{W[4]}; &\mathsf{ W[11]=W[10]}\oplus \mathsf{W[5]}.\\
%\end{array}
%\]
%Затем
%\[
%\begin{array}{ccccccc}
% \mathsf{W[12]} & = & \mathsf{SubByte(RotByte(W[11]))} &\oplus  &  \mathsf{W[6]} & \oplus  & \left[ \begin{array}{c}
% \mathsf{\alpha} \\
%  \mathsf{`00'}\\
%  \mathsf{`00'}\\
%  \mathsf{`00'}\\
%\end{array}
%\right] ,   \\
%\end{array}
%\]
%\[
%\begin{array}{ccc}
% \mathsf{W[13]=W[12]}\oplus \mathsf{W[7]}; & \mathsf{W[14]= W[13]}\oplus \mathsf{W[8]};  & \mathsf{W[15]=W[14]}\oplus \mathsf{W[9]},  \\
%\end{array}
%\]
%и т.д.
%\exampleend
%
%\example
%Пусть $\mathsf{Nk=8}.$ В этом случае ключ шифра имеет длину $256$ бита. Найдем столбцы расширенного ключа. Столбцы
%$\mathsf{W[0],W[1],W[2],W[3],W[4],W[5],W[6],W[7]}$  непосредственно заполняются битами ключа шифра. Номер следующего столбца
%$\mathsf{W[8]}$ кратен $\mathsf{Nk}$, поэтому
%\[
%\begin{array}{ccccccc}
% \mathsf{W[8]} & = & \mathsf{SubByte(RotByte(W[7]))} &\oplus  &  \mathsf{W[0]} & \oplus  & \left[ \begin{array}{c}
% \mathsf{`01'} \\
%  \mathsf{`00'}\\
%  \mathsf{`00'}\\
%  \mathsf{`00'}\\
%\end{array}
%\right]    \\
%\end{array}
%\].
%Далее имеем
%\[
%\begin{array}{ccc}
%\mathsf{ W[7]=W[6]}\oplus \mathsf{W[1]}; & \mathsf{W[8]=W[7]}\oplus \mathsf{W[2]}; & \mathsf{W[9]=W[8]}\oplus \mathsf{W[3]}; \\
%\mathsf{ W[10]=W[9]}\oplus \mathsf{W[4]}; & \mathsf{W[11]=W[10]}\oplus \mathsf{W[5]}.\\
%\end{array}
%\]
%Номер следующего столбца $\mathsf{W[12]}$ равен $12$. Так как $12-4$ кратно $\mathsf{Nk}$, то
%\[
%\begin{array}{ccc}
%\mathsf{ W[12]=SubByte(RotByte(W[11]))}\oplus \mathsf{W[4]}; & \mathsf{W[13]=W[12]}\oplus \mathsf{W[5]}; & \mathsf{W[14]=W[13]}\oplus \mathsf{W[6]}; \\
%\mathsf{ W[15]=W[14]}\oplus \mathsf{W[7]}. &  &\\
%\end{array}
%\]
%Затем
%\[
%\begin{array}{ccccccc}
% \mathsf{W[16]} & = & \mathsf{SubByte(RotByte(W[15]))} &\oplus  &  \mathsf{W[8]} & \oplus  & \left[ \begin{array}{c}
% \mathsf{\alpha} \\
%  \mathsf{`00'}\\
%  \mathsf{`00'}\\
%  \mathsf{`00'}\\
%\end{array}
%\right] ,   \\
%\end{array}
%\]
%\[
%\begin{array}{ccc}
% \mathsf{W[17]=W[16]}\oplus \mathsf{W[9]}; & \mathsf{W[18]=W[17]}\oplus \mathsf{W[10]};  &\mathsf{ W[19]=W[18]}\oplus \mathsf{W[10]}, \\
%\end{array}
%\]
%
%\[
%\begin{array}{ccc}
%\mathsf{ W[20]=SubByte(RotByte(W[19]))}\oplus \mathsf{W[12]}; & \mathsf{W[21]=W[20]}\oplus \mathsf{W[13]}; & \mathsf{W[22]=W[21]}\oplus \mathsf{W[14]}; \\
%\mathsf{ W[23]=W[22]}\oplus \mathsf{W[15]}, &  &\\
%\end{array}
%\]
%и т.д.

Ключ $i$-го раунда состоит из столбцов матрицы расширенного ключа
\[
    \mathsf{RoundKey} = (\mathsf{W}[4(i-1)], \mathsf{W}[4(i-1) + 1], \ldots, \mathsf{W}[4i-1]).
\]
%Если длина блока равна 192 битам $Nb=6$, то ключ 5-го раунда состоит из столбцов $W[24],W[25],W[26],W[27],W[28],W[29].$
%\exampleend

В настоящее время американский стандарт шифрования AES де-факто используется международно в негосударственных системах передачи данных, если позволяет законодательство страны. C 2010 г. процессоры Intel поддерживают специальный набор инструкций для шифра AES.


\input{Block_cipher_modes}

\section{Некоторые свойства блоковых шифров}

\input{feistel_network_reversibility}

\input{Feistel_cipher_without_s_blocks}

\input{Avalanche_effect}

\input{double_and_triple_ciphering}

\input{stream-ciphers}

\input{hash-functions}

\chapter{Криптосистемы с открытым ключом}\label{chapter-public-key}
\selectlanguage{russian}

\textbf{Криптосистемой с открытым ключом} (public-key cryptosystem, PKC) называется криптографическое преобразование, использующее два ключа -- открытый и секретный. Пара из \textbf{закрытого}\index{ключ!закрытый} (private key, secret key, SK)\footnote{В контексте криптосистем с открытым ключом можно ещё встретить использование термина <<секретный ключ>>. Мы не рекомендуем использовать данный термин, чтобы не путать с секретным ключом\index{ключ!секретный}, используемым в симметричных криптосистемах} и \textbf{открытого}\index{ключ!открытый} (public key, PK) ключей создается пользователем, который свой закрытый ключ держит в секрете, а открытый ключ делает общедоступным для всех пользователей. Криптографическое преобразование в одну сторону (шифрование) можно выполнить зная только открытый ключ, а в другую (расшифрование) -- только зная закрытый ключ. Во многих криптосистемах из закрытого ключа теоретически можно вычислить открытый ключ, однако это является сложной вычислительной задачей.

Если прямое преобразование выполняется открытым ключом, а обратное -- закрытым, то криптосистема называется \textbf{схемой шифрования с открытым ключом}. Все пользователи, зная открытый ключ получателя, могут зашифровать для него сообщение, которое может расшифровать только владелец закрытого ключа.

Если прямое преобразование выполняется закрытым ключом, а обратное -- открытым, то криптосистема называется \textbf{схемой электронной подписи (ЭП)}. Владелец закрытого ключа может \emph{подписать} сообщение, а все пользователи, зная открытый ключ, могут проверить, что подпись была создана только владельцем закрытого ключа и никем другим.

Криптосистемы с открытым ключом снижают требования к каналам связи, которые требуются для передачи данных. В симметричных криптосистемах перед началом связи (перед шифрованием сообщения и его передачей) требуется по защищённому каналу связи передать или согласовать секретный ключ шифрования. Злоумышленник не должен иметь возможность ни прослушать данный канал связи, ни подменить передаваемую информацию (ключ). Для надёжной работы криптосистем с открытым ключом необходимо, чтобы злоумышленник не имел возможности подменить открытый ключ легального пользователя. Другими словами, криптосистема с открытым ключом, в случае использования открытых и незащищённых каналов связи, устойчива к пассивному криптоаналитику\index{криптоаналитик!пассивный}, но всё ещё должна предпринимать меры по защите от активного криптоаналитика\index{криптоаналитик!активный}.

Для предотвращения атак <<человек посередине>> (man-in-the-middle attack)\index{атака!<<человек посередине>>} с активным криптоаналитиком\index{криптоаналитик!активный}, который бы подменял открытый ключ получателя во время его передачи будущему отправителю сообщений, используют \textbf{сертификаты открытых ключей}\index{сертификат открытого ключа}. Сертификат представляет собой информацию о соответствии открытого ключа и его владельца, подписанную электронной подписью третьего лица. В корпоративных информационных системах достаточно, если на всю организацию такое лицо, подписывающее сертификаты, будет одно. В этом случае его называют \textbf{доверенным центром сертификации} или \textbf{удостоверяющим центром}. В глобальной сети Интернет для защиты распространения программного обеспечения (например, защиты от подделок в ПО) и проверок сертификатов в протоколах на базе SSL/TLS\index{протокол!SSL/TLS} используется иерархия удостоверяющих центров, рассмотренная в разделе~\ref{section-CAs}. При обмене личными сообщениями и при распространении программного обеспечения с открытым кодом вместо жёсткой иерархии может использоваться \textbf{сеть доверия}\index{сеть доверия}. В сети доверия каждый участник может подписать сертификат любого другого участника. Предполагается, что подписывающий знает лично владельца сертификата и удостоверился о соответствии сертификата владельцу при личной встрече.

Криптосистемы с открытым ключом построены на основе односторонних (однонаправленных) функций c потайным входом. Под \textbf{односторонней} функцией понимают \emph{вычислительную} невозможность вычисления ее обращения: вычисление значения функции $y = f(x)$ при заданном аргументе $x$ является легкой задачей, вычисление аргумента $x$ при заданном значении функции $y$ -- трудной задачей.

Односторонняя функция $y = f(x,K)$ с \textbf{потайным входом}\index{функция!с потайным входом} $K$ определяется как функция, которая легко вычисляется при заданном $x$, и аргумент $x$ которой можно легко вычислить из $y$, если известен <<секретный>> параметр $K$, и вычислить невозможно, если параметр $K$ неизвестен.

Примером подобной функции является возведение в степень по модулю составного числа $n$:
	\[ c = f \left( m \right) = m ^ e \mod n.\]

Для того, чтобы быстро вычислить обратную функцию
	\[ m = f^{-1} \left( c \right) = \sqrt[e]{c} \mod n, \]
её можно представить в виде
	\[ m = y^{d} \mod m,\]
где
	\[ d = e^{-1} \mod \varphi \left( n \right). \]

В последнем выражении $\varphi \left( n \right)$ -- это функция Эйлера\index{функция!Эйлера}. В качестве <<потайной дверцы>> или секрета можно рассматривать или непосредственно само число <<$d$>>, или значение $\varphi \left( b \right)$. Последнее можно быстро найти только в том случае, если известно разложение числа $n$ на простые сомножители. Именно эта функция с потайной дверцей лежит в основе криптосистемы RSA\index{криптосистема!RSA}.

Необходимые математические основы модульной арифметики, групп, полей и простых чисел приведены в Приложении~\ref{chap:discrete-math}.

\section{Криптосистемы RSA}
\selectlanguage{russian}
\index{криптосистема!RSA}

\subsection[Шифрование]{Шифрование RSA}

В 1978 г. Рональд Рив\'{е}ст, Ади Шамир и Леонард Адлеман  (R. Rivest, A. Shamir, L. Adleman) предложили алгоритм, обладающий рядом интересных для криптографии свойств. На его основе была построена первая система шифрования с открытым ключом, получившая название по первым буквам фамилий авторов -- система RSA.

Рассмотрим принцип построения криптосистемы шифрования RSA с открытым ключом.

\begin{enumerate}
    \item \textbf{Создание пары из закрытого и открытого ключей.}
        \begin{enumerate}
            \item Случайно выбрать большие простые различные числа $p$ и $q$, для которых $\log_2 p \simeq \log_2 q > 512$ бит.
            \item Вычислить произведение $n = pq$.
            \item Вычислить функцию Эйлера $\varphi(n) = (p-1)(q-1)$.
            \item Выбрать случайное целое число $e \in [2, \varphi(n)-1]$ взаимно простое с $\varphi(n)$: $~ \gcd(e, \varphi(n)) = 1$. Свойство проверяют с помощью алгоритма Евклида.
            \item Вычислить число $d$ такое, что  $d e= 1 \mod \varphi(n)$. Для вычисления используется расширенный алгоритм Евклида.
            \item Закрытый ключ -- $\SK$, открытый ключ -- $\PK$
                \[ \SK = (d), ~ \PK = (n, e). \]

        \end{enumerate}

Генерация модуля $n = pq$ RSA системы является трудной задачей. Действительно, количество нечетных целых чисел длиной точно 500 бит равно $2^{(500-2)}$. Среди них имеется примерно
$(2^{500})/500 - (2^{499})/499 \approx (2^{500})/1000$ простых 500-разрядных чисел. Вероятность случайного выбора простого числа составляет примерно $1/250 $.
Поиск случайных больших простых чисел $p,q$ состоит в генерации случайного нечетного целого числа и проверке его по критериям простоты. Самый распространенный критерий -- вероятностный тест Миллера~---~Рабина\index{тест!Миллера~---~Рабина}. Все вероятностные тесты либо \emph{точно} определяют, что данное число составное, либо что оно \emph{возможно} простое. При $t$-кратной проверке тестом Миллера--Рабина со всеми положительными ответами <<возможно простое>> существует вероятность ошибки $P < \left( \frac{1}{4} \right)^t$, т.~е. ненулевая вероятность того, что число окажется на самом деле составным. Существуют и многие другие детерминированные и вероятностные тесты на простоту числа.

Криптостойкость RSA системы определяется сложностью разложения на сомножители целого $n$-разрядного числа и отсутствием <<лишних>> делителей.

    \item \textbf{Шифрование на открытом ключе $\PK$.}
        \begin{enumerate}
            \item Сообщение представляют целым числом $m \in [1, n-1]$.
            \item Шифротекст вычисляется как
                \[ c = m^e \mod n. \]
                Шифротекст -- тоже целое число из диапазона $[1, n-1]$.
        \end{enumerate}
    \item \textbf{Расшифрование на закрытом ключе $\SK$.}
        \begin{enumerate}
            \item Владелец закрытого ключа вычисляет
                \[ m = c^d \mod n. \]
            \item Покажем верность расшифрования. Пусть
                \[ ed = 1 + a \varphi(n). \]
                Если $m$ и $n$ взаимно простые, то по теореме Эйлера (по модулю $n$):
                \[ c^d = m ^{ed} = m^1 m^{a\varphi(n)} = m \cdot 1^a = m \mod n. \]

                В общем случае $m$ и $n$ могут иметь общие делители, но расшифрование тоже оказывается верным. Пусть $m = 0 \mod p$. По китайской теореме об остатках:
                \[
                     m = c^d \mod n ~\Leftrightarrow~
                     \left\{ \begin{array}{l}
                        m = c^d \mod p, \\
                        m = c^d \mod q. \\
                     \end{array} \right..
                \]
                Подставляя $c=m^e$, получаем тождество
                \[ \left\{ \begin{array}{l}
                    m^{ed} = 0 = m \mod p, \\
                    m^{ed} = m  \left( m^{q-1} \right)^{a(p-1)} = m \cdot 1^{a(p-1)} = m \mod q. \\
                \end{array} \right. \]
                Следовательно, $m^{ed} = m \mod pq$.
        \end{enumerate}
\end{enumerate}


Что касается вычислительной сложности других операций, то применение алгоритма Евклида для проверки, является ли число $e$  взаимно простым с числами $p-1, q-1$, а также вычисление обратного элемента $d$, считается легкой задачей (задачей с квадратичной сложностью, не более).
Возведение числа в заданную степень $d$ выполняется с помощью последовательного \emph{возведения в квадрат и перемножения}. Пусть
    \[ d = d_0 + d_1 2^1 + d_2 2^2 + \ldots + d_{k-1} 2^{k-1} \]
двоичное представление с коэффициентами $d_{i} \in \{ 0, 1 \}$. Степень $c^d$ вычисляется рекуррентным образом:
  \[ c^d =((... (((c^ {d_{k-1}})^2  (c^{d_{k-2}}))^2)\dots(c^{d_2}))^2 (c^{d_1}))^2 (c^{d_0}).\]

%    \[ c^d = c^ {d_0} \cdot (c^2)^{d_1} \cdot (c^{2^2})^{d_2} \dots  (c^{2^{k-1}})^{d_{k-1}}, \]
Всего выполняется  $k-1$ операций возведения в квадрат и не более $k-1$ умножений, что считается легкой задачей.


\subsubsection{Пример схемы}

%\example
%Схема шифрования RSA.
\begin{enumerate}
    \item Генерирование параметров.
        \begin{enumerate}
            \item Выберем числа $p=13, q=11, n = 143$.
            \item Вычислим $\varphi(n) = (p-1)(q-1) = 12 \cdot 10 = 120$.
            \item Выберем $e=23: ~ \gcd(e, \varphi(n))=1, ~ e \in [2, 119]$.
            \item Найдем $d = e^{-1} \mod \varphi(n) = 23^{-1} \mod 120 = 47$.
            \item Открытый и закрытые ключи:
                \[ \PK = (e:23, n:143), ~ \SK = (d:47). \]
        \end{enumerate}
    \item Шифрование.
        \begin{enumerate}
            \item Пусть сообщение $m = 22 \in [1, n-1]$.
            \item Вычислим шифротекст
                \[ c = m^e = 22^{23} = 55 \mod 143. \]
        \end{enumerate}
    \item Расшифрование.
        \begin{enumerate}
            \item Полученный шифротекст $c = 55$.
            \item Вычислим открытый текст
                \[ m = c^d = 55^{47} = 22 \mod 143. \]
        \end{enumerate}
\end{enumerate}

%Рассмотрим ее основные положения на примере криптосистемы с открытым ключом.
%Приведем общую схему алгоритма RSA.
%$C_i=M_{i}^{E_k}(mod N_j)$
%$N_j=P_{j}Q_{j}$
%$M_i=C_{i}^{D_k}(mod N_j)$
%$E_k\neq D_k$
%Вычислить $E_k$ из $D_k$  при длине блока сообщения  $L_{блока} > L_{дополнения}$ можно только с экспоненциальной сложностью. $E_k D_K=1(mod \varphi(N_j))$
%Данное сравнение не дает единственного решения. Решение данного сравнения и можно свести к следующему уравнению:
%$ax+by=1$
%$E_k D_k=k \varphi(N_j)+1$
%$1\leq E_k D_k <\varphi(N_j)$
%$\varphi(N_j)(-k)+ E_k D_k=1$
%Стандарт ISO X.509 определяет требования по реализации алгоритма RSA, в частности, требования к общесистемным параметрам и ключам, методы распространения сертификатов ключей и ключевых параметров, а также порядок ввода их в действие и многое другое.


\subsection[Электронная подпись]{Электронная подпись RSA}

Предположим, что пользователь $A$ сообщения не шифрует, но хочет посылать свои сообщения в виде открытых текстов с подписью. Для этого надо создать электронную подпись (ЭП). Это можно сделать, используя систему RSA. При этом должны быть выполнены следующие требования:
\begin{itemize}
    \item вычисление подписи от сообщения является вычислительно легкой задачей;
    \item фальсификация подписи при неизвестном закрытом ключе -- вычислительно трудная задача;
    \item подпись должна быть проверяемой открытым ключом.
\end{itemize}

Создание параметров ЭП RSA производится так же, как и для схемы шифрования RSA. Пусть  $A$ имеет закрытый ключ $\SK = (d)$, а получатель (проверяющий) $B$ -- открытый ключ $\PK = (e,n)$ пользователя $A$.

\begin{enumerate}
    \item $A$ вычисляет подпись сообщения $m \in [1,n-1]$ как
        \[ s = m^{d} \mod n \]
        на своем закрытом ключе $\SK$.
    \item $A$ посылает $B$ сообщение в виде $(m, s)$, где $m$ -- открытый текст, $s$ -- подпись.
    \item $B$ принимает сообщение $(m, s)$, возводит $s$ в степень $e$ по модулю $n$ ($e, n$ -- часть открытого ключа). В результате вычислений $B$ получает открытый текст
        \[ \left( m^{d} \mod n \right)^{e} \mod n = m. \]
    \item Сравнивает полученное значение с первой частью сообщения. При полном совпадении подпись принимается.
\end{enumerate}
Недостаток этой системы создания ЭП состоит в том, что подпись $m^{d} \mod n$ имеет большую длину, равную длине открытого сообщения $m$.

Для уменьшения длины подписи применяется другой вариант процедуры: вместо сообщения $m$ отправитель подписывает $h(m)$, где $h(x)$ -- известная криптографическая хэш-функция. Модифицированная процедура состоит в следующем.

\begin{enumerate}
    \item $A$ посылает $B$ сообщение в виде $(m, s)$, где $m$ -- открытый текст,
        \[ s = h(m)^d \mod n \]
        подпись.
    \item $B$ принимает сообщение $(m, s)$, вычисляет хэш $h(m)$ и возводит подпись в степень
        \[ h_1 = s^e \mod n. \]
    \item $B$ сравнивает значения $h(m)$ и $h_1$. При равенстве
        \[ h(m) = h_1 \]
        подпись считается подлинной, при неравенстве -- фальсифицированной.
\end{enumerate}


\subsubsection{Пример схемы}

\begin{enumerate}
    \item Генерирование параметров.
        \begin{enumerate}
            \item Выберем $p=13, q=17, n = 221$.
            \item Вычислим $\varphi(n) = (p-1)(q-1) = 12 \cdot 16 = 192$.
            \item Выберем $e=25: ~ \gcd(e = 25, \varphi(n) = 192) = 1, \\
                e \in [2, \varphi(n) - 1 = 191]$.
            \item Найдем $d = e^{-1} \mod \varphi(n) = 25^{-1} \mod 192 = 169$.
            \item Открытый и закрытые ключи:
                \[ \PK = (e:25, n:221), ~ \SK = (d:169). \]
        \end{enumerate}
    \item Подписание.
        \begin{enumerate}
            \item Пусть хэш сообщения $h(m) = 12 \in [1, n-1]$.
            \item Вычислим ЭП
                \[ s = h^d = 12^{169} = 90 \mod 221. \]
        \end{enumerate}
    \item Проверка подписи.
        \begin{enumerate}
            \item Пусть хэш полученного сообщения $h(m) = 12$, полученная подпись $s = 90$.
            \item Выполним проверку
                \[ h_1 = s^e = 90^{25} = 12 \mod 221, ~~ h_1 = h, \]
                подпись верна.
        \end{enumerate}
\end{enumerate}


\subsection[Рандомизация шифрования и ЭП]{Рандомизация шифрования и \protect\\ подписания RSA}

\textbf{Семантически безопасной}\index{криптосистема!семантически-безопасная} называется криптосистема, для которой вычислительно невозможно извлечь любую информацию из шифротекстов, кроме длины шифротекста. Алгоритм RSA не является семантически безопасным. Одинаковые сообщения шифруются одинаково и, следовательно, применима атака на различение сообщений.

Кроме того, сообщения длиной менее $\frac{k}{3}$ бит, зашифрованные на малой экспоненте $e=3$, \emph{дешифруются} нелегальным пользователем извлечением обычного кубического корня.

В приложениях RSA используется только в сочетании с рандомизацией\index{рандомизация шифрования}. В стандарте PKCS\#1 RSA Laboratories описана схема рандомизации перед шифрованием OAEP-RSA (Optimal Asymmetric Encryption Padding). Примерная схема:
\begin{enumerate}
    \item Выбирается случайное $r$.
    \item Для открытого текста $m$ вычисляется
        \[ x = m \oplus H_1(r), ~ y = r \oplus H_2(x), \]
        где $H_1$ и $H_2$ -- криптографические хэш-функции.
    \item Сообщение $M = x \| y$ далее шифруется RSA.
\end{enumerate}
Восстановление $m$ из $M$ при расшифровании:
    \[ r = y \oplus H_2(x), ~ m = x \oplus H_1(r). \]

В модификации OAEP+ $x$ вычисляется как
    \[ x = (m \oplus H_1(r)) \| H_3(m \| r). \]

В описанной выше схеме ЭП под $m$ понимается хэш открытого текста, вместо шифрования выполняется подписание, вместо расшифрования -- проверка подписи.


\subsection{Выбор параметров и оптимизация}

\subsubsection{Выбор экспоненты $e$}

В случайно выбранной экспоненте $e$ c битовой длиной $k = \lceil \log_2 e \rceil$ половина бит в среднем равна 0, половина -- 1. При возведении в степень $m^e \mod n$ по методу <<возводи в квадрат и перемножай>> получится $k-1$ возведений в квадрат и, в среднем,
 $\frac{1}{2}(k-1)$ умножений.

Если выбрать $e$, содержащую малое число единиц в двоичной записи, то число умножений уменьшится до числа единиц в $e$.

Часто экспонента $e$ выбирается  \emph{малым} \emph{простым} числом и/или содержащим малое число единиц в битовой записи для ускорения шифрования или проверки подписи, например:
\[
    \begin{array}{l}
        3 = [11]_2, \\
        17 = 2^4+1 = [10001]_2, \\
        257 = 2^8+1 = [100000001]_2, \\
        65537 = 2^{16}+1 = [10000000000000001]_2.
    \end{array}
\]

%Время шифрования или проверки подписи для малых экспонент становится $O(k^2)$ вместо $O(k^3)$, то есть в сотни раз быстрее для 1000-битовых чисел.


\subsubsection[Ускорение шифрования]{Ускорение шифрования по китайской \protect\\ теореме об остатках}

Возводя $m$ в степень $e$ отдельно по $\mod p$ и $\mod q$ и применяя китайскую теорему об остатках (Chinese remainder theorem, CRT), можно быстрее выполнить шифрование.

Однако ускорение шифрования в криптосистеме RSA через CRT может привести к уязвимостям в некоторых применениях, например в смарт-картах.

\example
Пусть $c = m^e \mod n$ передается на расшифрование на смарт-карту, где вычисляется
\[ \begin{array}{c}
    m_p = c^d \mod p, \\
    m_q = c^d \mod q, \\
    m = m_p q (q^{-1} \mod p) + m_q p (p^{-1} \mod q) \mod n. \\
\end{array} \]
Криптоаналитик внешним воздействием может вызвать сбой во время вычисления $m_p$ (или $m_q$), в результате получится $m_p'$ и $m'$ вместо $m$. Зная $m_p'$ и $m'$, криптоаналитик находит разложение числа $n$ на множители $p,q$:
    \[ \gcd(m' - m, ~ n) = \gcd( (m_p' - m) q (q^{-1} \mod p), ~ pq) = q. \]
\exampleend


\subsubsection{Длина ключей}

В 2005 году было разложено 663-битовое число вида RSA. Время разложения в эквиваленте составило 75 лет вычислений одного ПК. Самые быстрые алгоритмы факторизации -- субэкспоненциальные\index{задача!факторизации}. Минимальная рекомендуемая длина модуля $n$ -- 1024 бит, но лучше использовать 2048 или 4096 бит.

В приложении показано, что битовая сложность (количество битовых операций) вычисления произвольной степени $a^b \mod n$ является кубической $O(k^3)$, а возведения в квадрат $a^2 \mod n$ и умножения $a b \mod n$ -- квадратичными $O(k^2)$, где $k$ -- битовая длина чисел $a,b,n$.

%Увеличение длины модуля $n$ в 2 раза увеличивает время возведения в степень в $2^3$ раз для большой экспоненты $e$, а для маленькой экспоненты -- в $2^2$ раза.


\input{el-gamal}

\input{GOST_R_34.10-2001.tex}

\section{Длины ключей}
\selectlanguage{russian}

В табл.~\ref{tab:recommended-key-lengths} приведены битовые длины ключей для криптосистем.
%Традиционные рекомендации основаны на аппроксимации существующих алгоритмов для взлома на 10-30 лет вперед.

\begin{table}[!ht]
    \centering
    \caption{Минимальные длины ключей в битах по стандартам России и США\label{tab:recommended-key-lengths}}
    \resizebox{\textwidth}{!}{ \begin{tabular}{|l|c|c|c|c|}
        \hline
        & \multirow{2}{*}{\parbox{1.5cm}{Блоковые шифры, $K$}} & \multicolumn{3}{|c|}{Схема ЭП} \\
        \cline{3-5}
        & & \parbox{1.5cm}{RSA\index{криптосистема!RSA}, $n$} & \parbox{2.3cm}{Эллипт. кривые, порядок точки} & \parbox{3.5cm}{Эль-Гамаль\index{криптосистема!Эль-Гамаля} $\mod p$: модуль / порядок (под)группы} \\
        \hline \hline
        \multicolumn{5}{|c|}{Взломано} \\
        \hline
        Биты & 56 & 663 & 109 & 503  \\
        Конкурс & \textsc{DesChal} & RSA-200 & ECC2K-108 &  \\
        Год & 1997 & 2005 & 2000 &  \\
        \hline \hline
        \multicolumn{5}{|c|}{Стандарт России} \\
        \hline
        Биты & 256 &  & 255 & \\
        ГОСТ & 28147—89 & --- & 34.10-2001 & --- \\
        Год & 1989 & & 2001 & \\
%       \hline
%       \multicolumn{2}{|l|}{\parbox{4cm}{Россия: нелицензируемая деятельность}} & \multicolumn{4}{c|}{40} \\
        \hline \hline
        \multicolumn{5}{|c|}{Стандарт США} \\
        \hline
        Биты & 128-256 & 1024-3072 & 151-480 & 1024-3072/160-256 \\
        FIPS № & 197 & draft 186-3 & draft 186-3 & draft 186-3 \\
        Год & 2001 & 2006 & 2006 & 2006 \\
%       \hline
%       \multicolumn{2}{|l|}{\parbox{4cm}{США: экспортные ограничения до 2001 г.}} & 56 & 512 & 112 & 512/112 \\
%       \hline \hline
%       \multicolumn{2}{|l|}{Традиционные} & 80 & 1024 & 160 & 1024/160 \\
%       \cline{3-6}
%       \multicolumn{2}{|l|}{рекомендации} & 112 & 2048 & 224 & 2048/224 \\
%       \hline
%       \multicolumn{2}{|l|}{\parbox{4cm}{Рекомендация Lenstra, Verheul для 2010 г.}} & 78 & 1369 & 146-160 & 1369/138 \\
        \hline
    \end{tabular} }
\end{table}
%}\end{center}


\subsection*{Скорость вычисления ЭП}

Сравним производительность схем ЭП, чтобы продемонстрировать преимущества ЭП вида Эль-Гамаля\index{криптосистема!Эль-Гамаля} перед RSA\index{криптосистема!RSA} для больших ключей. В приложении показано, что в модульной арифметике по модулю числа $n$ с битовой длиной $k \simeq \log_2 n$ операции имеют битовую сложность:
\[ \begin{array}{lcl}
    a^b \mod n & - & O(k^3), \\
    ab \mod n, ~ a^{-1} \mod n & - & O(k^2), \\
    a+b \mod n & - & O(k). \\
\end{array} \]

Так как все описанные схемы ЭП используют возведение в степень по модулю, то битовая сложность -- $O(k^3)$. Оценки количества целочисленных $t$-разрядных умножений при вычислении ЭП имеют вид:
\begin{enumerate}
    \item RSA\index{электронная подпись!RSA}:
        \[ (2 \log_2 n) \cdot \left( \frac{\log_2 n}{t} \right)^2; \]
    \item DSA\index{электронная подпись!DSA} (Digital Signature Algorithm, стандарт США~\cite{FIPS-PUB-186-4}), вычисляемая по принципу Эль-Гамаля\index{криптосистема!Эль-Гамаля} по модулю $p$ и с порядком циклической подгруппы $q$:
        \[ (2 \log_2 q) \cdot \left( \frac{\log_2 p}{t} \right)^2; \]
    \item ГОСТ Р 34.10-2001\index{электронная подпись!ГОСТ Р 34.10-2001} (стандарт России~\cite{GOST-2001}) и ECDSA\index{электронная подпись!ECDSA} (Elliptic Curve Digital Signature Algorithm, стандарт США~\cite{FIPS-PUB-186-4}), вычисляемые по принципу Эль-Гамаля\index{криптосистема!Эль-Гамаля} в группе точек эллиптической кривой по модулю $p$:
        \[ (2 \log_2 p) \cdot 4 \cdot \left( \frac{\log_2 p}{t} \right)^2; \]
\end{enumerate}

В табл.~\ref{tab:signature-rate} приведены оценки скорости вычисления ЭП (оценки числа умножений 64-битовых слов).

\begin{table}[!ht]
    \centering
    \caption{Оценочное число 64-битовых умножений для вычисления ЭП\label{tab:signature-rate}}
    \begin{tabular}{|c|l|c|}
        \hline
        ЭП & Оценочное число 64-битовых умножений \\
        \hline \hline
        RSA\index{электронная подпись!RSA} 1024 & $(2 \cdot 1024) \cdot \left( \frac{1024}{64} \right)^2 \approx$ 500 000 \\
        RSA\index{электронная подпись!RSA} 2048 & 4 000 000 \\
        RSA\index{электронная подпись!RSA} 3072 & 14 000 000 \\
        RSA\index{электронная подпись!RSA} 4096 & 34 000 000 \\
        \hline \hline
        DSA\index{электронная подпись!DSA} 1024/160 & $(2 \cdot 160) \cdot \left( \frac{1024}{64} \right)^2 \approx$ 82 000 \\
        DSA\index{электронная подпись!DSA} 3072/256 & 1 200 000 \\
        \hline \hline
        ECDSA\index{электронная подпись!ECDSA} 160 & $(2 \cdot 160) \cdot 4 \cdot \left( \frac{160}{64} \right)^2 \approx$ 8 000 \\
        ECDSA\index{электронная подпись!ECDSA} 512 & 260 000 \\
        \hline \hline
        ГОСТ Р 34.10-2001\index{электронная подпись!ГОСТ Р 34.10-2001} & $(2 \cdot 256) \cdot 4 \cdot \left( \frac{256}{64} \right)^2 \approx$ 33 000 \\
        \hline
    \end{tabular}
\end{table}


\section{Инфраструктура открытых ключей}\label{chapter-public-key-infrastructure}

\subsection{Иерархия удостоверяющих центров}\label{section-CAs}
\selectlanguage{russian}

Проблему аутентификации и распределения сеансовых симметричных ключей шифрования в Интернете, а также в больших локальных и виртуальных сетях, решают с помощью построения иерархии открытых ключей криптосистем с открытым ключом.

\begin{enumerate}
    \item Существует удостоверяющий центр (УЦ) верхнего уровня\index{Удостоверяющий центр!верхнего уровня}, корневой УЦ\index{Удостоверяющий центр!корневой} (Root Certification Authority, $CA$)\index{Certification Authority!Root}, обладающий парой из секретного и открытого ключей. Открытый ключ УЦ верхнего уровня распространяется среди всех пользователей, причем все пользователи \emph{доверяют УЦ}. Это означает, что:
        \begin{itemize}
            \item УЦ -- <<хороший>>, обеспечивает надежное хранение секретного ключа, не пытается фальсифицировать и скомпрометировать свои ключи,
            \item имеющийся у пользователей открытый ключ УЦ действительно принадлежит УЦ.
        \end{itemize}
        В массовых информационных и интернет-системах открытые ключи многих корневых УЦ встроены в дистрибутивы и пакеты обновлений ПО. Доверие пользователей неявно проявляется в их уверенности в том, что открытые ключи корневых УЦ, включенные в ПО, не фальсифицированы и не скомпрометированы. \emph{Де-факто пользователи доверяют а) распространителям ПО и обновлений, б) корневому УЦ.}\index{доверие}

        Назначение УЦ верхнего уровня -- проверка принадлежности и подписание открытых ключей других удостоверяющих центров второго уровня, а также организаций и сервисов. УЦ подписывает своим секретным ключом следующее сообщение:
        \begin{itemize}
            \item название и URI УЦ нижележащего уровня или организации/сервиса,
            \item значение сгенерированного открытого ключа и название алгоритма соответствующей криптосистемы с открытым ключом,
            \item время выдачи и срок действия открытого ключа.
        \end{itemize}

    \item УЦ второго уровня (Certificate Authority, CA) имеют свои пары открытых и секретных ключей, сгенерированных и подписанных корневым УЦ. Причем перед подписанием корневой УЦ убеждается в <<надежности>> УЦ второго уровня, производит юридические проверки. Корневой УЦ не имеет доступа к секретным ключам УЦ второго уровня.

        Пользователи, имея в своей базе открытых ключей доверенные открытые ключи корневого УЦ, могут проверить ЭП открытых ключей УЦ 2-го уровня и убедиться, что предъявленный открытый ключ действительно принадлежит данному УЦ. Таким образом:
        \begin{itemize}
            \item Пользователи полностью доверяют корневому УЦ и его открытому ключу, который у них хранится. Пользователи верят, что корневой УЦ не подписывает небезопасные ключи и гарантирует, что подписанные им ключи действительно принадлежат УЦ 2-го уровня.
            \item Проверив ЭП открытого ключа УЦ 2-го уровня с помощью доверенного открытого ключа УЦ 1-го уровня, пользователь верит, что открытый ключ УЦ 2-го уровня действительно принадлежит данному УЦ и не был скомпрометирован.
        \end{itemize}

        Аутентификация в протоколе защищенного интернет-соединения SSL/TLS\index{протокол!SSL/TLS} достигается в результате проверки пользователями совпадения URI-адреса сервера из ЭП с фактическим адресом.

        УЦ второго уровня в свою очередь тоже подписывает открытые ключи УЦ третьего уровня, а также организаций. И так далее по уровням.

    \item В результате построена \emph{иерархия} подписанных открытых ключей.

    \item Открытый ключ с идентификационной информацией (название организации, URI-адрес веб-ресурса, дата выдачи, срок действия и др.) и подписью УЦ вышележащего уровня, заверяющей ключ и идентифицирующие реквизиты, называется \textbf{сертификатом открытого ключа},\index{сертификат открытого ключа} на который существует международный стандарт X.509, последняя версия 3. В сертификате указывается его область применения: подписание других сертификатов, аутентификация для веб, аутентификация для электронной почты и т.д.
\end{enumerate}


\begin{figure}[!ht]
	\centering
	\includegraphics[width=0.8\textwidth]{pic/X509-hierarchy}
	\caption{Иерархия сертификатов\label{fig:x509-hierarchy}}
\end{figure}

На рис. \ref{fig:x509-hierarchy} приведены пример иерархии сертификатов и путь подписания сертификата X.509 интернет-сервиса Google Mail.

Система распределения, хранения и управления сертификатами открытых ключей называется \textbf{инфраструктурой открытых ключей}\index{инфраструктура открытых ключей} (Public Key Infrastructure, PKI)\index{PKI}. PKI применяется для аутентификации в системах SSL/TLS\index{протокол!SSL/TLS}, IPsec\index{протокол!IPsec}, PGP\index{PGP} и т.д. Помимо процедур выдачи и распределения открытых ключей PKI также определяет процедуру отзыва скомпрометированных или устаревших сертификатов.


\input{x509}


\input{key_distribution_protocols}

\chapter{Разделение секрета}

\section{Пороговые схемы}

Идея \textbf{пороговой} $(n, N)$-схемы\index{разделение секрета!пороговое} разделения общего секрета среди $N$ пользователей состоит в следующем.
%описывается так:
Доверенная сторона хочет распределить некий секрет $K_0$ между $N$ пользователями таким образом, что:
%. Поставлены следующие условия:
\begin{itemize}
    \item любые $m, ~ n \le m \le N$ легальных пользователей могут получить секрет (или доступ к секрету), если предъявят свои секретные ключи;
    \item любые $m, ~ m < n$, легальных пользователей не могут получить секрет и не могут определить (вычислить) этот секрет, пытаясь решить трудную в вычислительном смысле задачу.
\end{itemize}

Далее рассмотрены два случая: $(n, N)$-схема Шамира и простая $(N,N)$-схема.

\input{shamirs_secret_sharing}

\input{xor_secret_sharing}

\section[Распределение секрета по коалициям]{Распределение секрета по \protect\\ коалициям}

\subsection{Схема для нескольких коалиций}

Предположим, что имеется $N$ легальных пользователей
    \[ \{ U_1, U_2, \dots, U_N \}, \]
которым нужно сообщить (открыть, получить доступ к) общий секрет $K$.

Секрет может быть доступен только определенным коалициям\index{распределение секрета!по коалициям}, например
\[ \begin{array}{l}
    C_1 = \{ U_1, U_2 \}, \\
    C_2 = \{ U_1, U_3, U_4 \}, \\
    C_3 = \{ U_2, U_3 \}, \\
    \dots
\end{array} \]
При этом ни одна из коалиций $C_i, ~ i = 1, 2, \dots$ не должна быть подмножеством другой коалиции.


\subsubsection{Пример 1}

Имеется 4 участника
    \[ \{ U_1, U_2, U_3, U_4 \}, \]
которые образуют 3 коалиции
\[ \begin{array}{l}
    C_1 = \{ U_1, U_2 \}, \\
    C_2 = \{ U_1, U_3 \}, \\
    C_3 = \{ U_2, U_3, U_4 \}. \\
\end{array} \]
Распределение частичных секретов между ними представлено в виде табл. \ref{tab:secret-share-coalition-1}, в которой введены следующие обозначения: $a_1, b_1, c_2, c_3$ -- случайные числа из кольца $\Z_M$. В строках таблицы содержатся частичные секреты каждого из пользователей, в столбцах таблицы показаны частичные секреты, соответствующие каждой из коалиций.

\begin{table}[!ht]
    \centering
    \caption{Распределение секрета по определенным коалициям\label{tab:secret-share-coalition-1}}
    \begin{tabular}{|c||c|c|c|}
        \hline
              & $C_1 = \{ U_1, U_2 \}$ & $C_2 = \{U_1, U_3 \}$ & $C_3 = \{ U_2, U_3, U_4 \}$ \\
        \hline \hline
        $U_1$ & $a_1$     & $b_1$     & -- \\
        $U_2$ & $K - a_1$ & --        & $c_2$ \\
        $U_3$ & --        & $K - b_1$ & $c_3$  \\
        $U_4$ & --        & --        & $K - c_2 - c_3$ \\
        \hline
    \end{tabular}
\end{table}

Как видно из приведенных данных, суммирование по модулю $M$ чисел, приведенных в каждом из столбцов таблицы, открывает секрет $K$.


\subsubsection{Пример 2}

%\section{Схема разделения секрета на монотонных булевых функциях}
%\example
В системе распределения секрета доверенный
%с использованием монотонных булевых функций
центр использует кольцо $\Z_m$ целых чисел по модулю $m$. Требуется разделить секрет $K$ между $5$ пользователями
    \[ \{ U_1, U_2, U_3, U_4, U_5 \} \]
так, чтобы восстановить секрет могли только коалиции
\[ \begin{array}{lll}
    C_1 = \{ U_1, U_2 \},      & & C_2 = \{ U_1, U_3 \}, \\
    C_3 = \{ U_2, U_3, U_4 \}, & & C_4 = \{ U_2, U_3, U_5 \}, \\
    C_5 = \{ U_3, U_4, U_5 \}, & & C_6 = \{ U_1, U_2, U_3 \}. \\
\end{array} \]

Заданное множество коалиций с доступом не является минимальным, так как одни коалиции входят в другие:
    \[ C_1 \subset C_6, ~ C_2 \subset C_6. \]
Исключая $C_6$, получим минимальное множество коалиций с доступом к секрету -- ни одна из оставшихся коалиций не входит в другую $C_i \nsubseteq C_j$ для $i \neq j$. Пользователям выдаются тени по минимальному множеству коалиций с доступом. В строках таблицы \ref{tab:secret-share-coalition-2} содержатся частичные секреты каждого из пользователей, в столбцах таблицы показаны частичные секреты, соответствующие каждой из коалиций.

\begin{table}[!ht]
    \centering
    \caption{Распределение секрета по определенным коалициям\label{tab:secret-share-coalition-2}}
    \begin{tabular}{|c||c|c|c|c|c|}
        \hline
              & $C_1$     & $C_2$     & $C_3$           & $C_4$           & $C_5$  \\
        \hline \hline
        $U_1$ & $a_1$     & $b_1$     & --              & --              & -- \\
        $U_2$ & $K - a_1$ & --        & $c_2$           & $d_2$           & --\\
        $U_3$ & --        & $K - b_1$ & $c_3$           & $d_3$           & $e_3$ \\
        $U_4$ & --        & --        & $K - c_2 - c_3$ & --              & $e_4$ \\
        $U_5$ & --        & --        & --              & $K - d_2 - d_3$ & $K - e_3 - e_4$ \\
        \hline
    \end{tabular}
\end{table}

Тени выбираются случайно из кольца $\mathbb{\Z}_m$. В результате у пользователей будут тени:
%\exampleend

\input{brickells_scheme}

\chapter{Примеры систем защиты}

\input{kerberos}

\input{pgp}

\input{tls}

\input{ipsec}

\section[Защита персональных данных в мобильной связи]{Защита персональных данных в \protect\\ мобильной связи}

\input{gsm2}

\input{gsm3}

%\section{Беспроводная сеть Wi-Fi}
%\subsection{WPA-PSK2, 802.11n, Radix?}
%\subsection{Wimax 802.16(?)}

\chapter{Аутентификация пользователя}


\section{Многофакторная аутентификация}

Для защищенных приложений применяется \textbf{многофакторная} аутентификация одновременно по факторам различной природы:
\begin{enumerate}
    \item Свойство, которым обладает субъект. Например, биометрия, природные уникальные отличия: лицо, радужная оболочка глаз, папиллярные узоры, последовательность ДНК.
    \item Знание -- информация, которую знает субъект. Например, пароль, PIN-код.
    \item Владение -- вещь, которой обладает субъект. Например, электронная или магнитная карта, флеш-память.
%    \item Факторы присвоения. Например, номер машины, RFID-метка.
\end{enumerate}

В обычных массовых приложениях из-за удобства использования применяется аутентификация только по \textbf{паролю}\index{пароль}, который является общим секретом пользователя и информационной системы. Биометрическая аутентификация по отпечаткам пальцев применяется существенно реже. Как правило, аутентификация по отпечаткам пальцев является дополнительным, а не вторым обязательным фактором (тоже из-за удобства ее использования).

%Так же явно или неявно используется аутентификация по факторам:
%\begin{enumerate}
%    \item Социальная сеть. Доверие к индивидууму в личном общении или интернет на основании общих связей.
%    \item Географическое положение. Например, для проверки оплаты товаров по кредитной карте.
%    \item Время. Доступ к сервисам или местам только в определенное время.
%    \item И др.
%\end{enumerate}


\section[Энтропия и криптостойкость паролей]{Энтропия и криптостойкость \protect\\ паролей}

Стандартный набор символов паролей, которые можно набрать на клавиатуре, используя английские буквы и небуквенные символы, состоит из $D=94$ символов. При длине пароля $L$ символов и предположении равновероятного использования символов энтропия паролей равна
    \[ H = L \log_2 D. \]

Клод Шеннон, исследуя энтропию символов английского текста, изучал вероятность успешного предсказания людьми следующего символа по первым нескольким символам слов или текста. В результате Шеннон получил оценку энтропии первого символа $s_1$ текста порядка $H(s_1) \approx 4{,}6$--$4{,}7$ бит/символ и оценки энтропий последующих символов, постепенно уменьшающиеся до $H(s_9) \approx 1{,}5$ бит/символ для 9-го символа. Энтропия для длинных текстов литературных произведений получила оценку $H(s_\infty) \approx 0{,}4$ бит/символ.

Статистические исследования баз паролей показывают, что наиболее часто используются буквы <<a>>, <<e>>, <<o>>, <<r>> и цифра <<1>>.

NIST использует следующие рекомендации для оценки энтропии паролей\index{энтропия!пароля}, создаваемых людьми.
\begin{enumerate}
    \item Энтропия первого символа $H(s_1) = 4$ бит/символ.
    \item Энтропия со 2-го по 8-й символы $H(s_{2 \leq i \leq 8}) = 2$ бит/символ.
    \item Энтропия с 9-го по 20-й символы $H(s_{9 \leq i \leq 20}) = 1{,}5$ бит/символ.
    \item Энтропия с 21-го символа $H(s_{i \geq 21}) = 1$ бит/символ.
    \item Проверка композиции на использование символов разных регистров и небуквенных символов добавляет до 6 бит энтропии пароля.
    \item Словарная проверка на слова и часто используемые пароли добавляет до 6 бит энтропии для коротких паролей. Для 20-символьных и более длинных паролей прибавка к энтропии 0 бит.
\end{enumerate}

Для оценки энтропии пароля нужно сложить энтропии символов $H(s_i)$ и сделать дополнительные надбавки, если пароль удовлетворяет тестам на композицию и отсутствие в словаре.

\begin{table}[!ht]
    \centering
    \caption{Оценка NIST предполагаемой энтропии паролей\label{tab:password-entropy}}
    \resizebox{\textwidth}{!}{ \begin{tabular}{|c||c|c|c||c|}
        \hline
        \multirow{2}{*}{\parbox{1.5cm}{Длина пароля, символы}} & \multicolumn{3}{|c||}{\parbox{6cm}{Энтропия паролей пользователей по критериям NIST}} & \multirow{2}{*}{\parbox{2.5cm}{Энтропия случайных равновероятных паролей}} \\
        \cline{2-4}
        & \parbox{1.5cm}{Без проверок} & \parbox{2cm}{Словарная проверка} & \parbox{2.5cm}{Словарная и композиционная проверка} & \\
        \hline
        4  & 10 & 14 & 16 & 26.3 \\
        6  & 14 & 20 & 23 & 39.5 \\
        8  & 18 & 24 & 30 & 52.7 \\
        10 & 21 & 26 & 32 & 65.9 \\
        12 & 24 & 28 & 34 & 79.0 \\
        16 & 30 & 32 & 38 & 105.4 \\
        20 & 36 & 36 & 42 & 131.7 \\
        24 & 40 & 40 & 46 & 158.0 \\
        30 & 46 & 46 & 52 & 197.2 \\
        40 & 56 & 56 & 62 & 263.4 \\
        \hline
    \end{tabular} }
\end{table}

В табл. \ref{tab:password-entropy} приведена оценка NIST на величину энтропии пользовательских паролей в зависимости от их длины и сравнение с энтропией случайных паролей с равномерным распределением символов из набора в $D=94$ символов клавиатуры. Вероятное число попыток для подбора пароля составляет $O(2^H)$. Из таблицы видно, что по критериям NIST энтропия реальных паролей в 2--4 раза меньше энтропии случайных паролей с равномерным распределением символов.

\example
Оценим общее количество существующих паролей. Население Земли -- 7 млрд. Предположим, что все население использует компьютеры, Интернет, и у каждого человека по 10 паролей. Общее количество существующих паролей -- $7 \cdot 10^{10} \approx 2^{36}$.
%Следовательно, \emph{реальная энтропия паролей не превышает 36 бит}.

Имея доступ к наиболее массовым интернет-сервисам с количеством пользователей десятки и сотни миллионов, в которых пароли часто хранятся в открытом виде из-за необходимости обновления ПО и, в частности, выполнения аутентификации, мы 1) имеем базу паролей, покрывающую существенную часть пользователей, 2) можем статистически построить правила генерирования паролей. Даже если пароль хранится в защищенном виде, то при вводе пароль, как правило, в открытом виде пересылается по Интернету, и все преобразования пароля для аутентификации осуществляет интернет-сервис, а не веб-браузер. Следовательно, интернет-сервис имеет доступ к исходному паролю.
\exampleend

В 2002 г. был подобран ключ для 64-битового блокового шифра RC5 сетью \texttt{distributed.net} персональных компьютеров, выполнявших вычисления в фоновом режиме. Суммарное время вычислений всех компьютеров -- 1757 дней, было проверено 83\% пространства всех ключей. Это означает, что пароли с оценочной энтропией менее 64 бит, то есть \emph{все пароли} до 40 символов по критериям NIST, могут быть подобраны в настоящее время. Конечно, с оговорками на то, что 1) нет ограничений на количество и скорость попыток аутентификаций, 2) алгоритм генерирования вероятных паролей эффективен.

Строго говоря, использование даже 40-символьного пароля для аутентификации или в качестве ключа блокового шифрования является небезопасным.


\subsubsection{Число паролей}

Приведем различные оценки числа паролей, создаваемых людьми.

Пароли, создаваемые людьми, основаны на словах или закономерностях естественного языка. В английском языке всего около $1\ 000\ 000 \approx 2^{20}$ слов, включая термины.

%http://www.springerlink.com/content/bh216312577r6w64/fulltext.pdf
%http://www.antimoon.com/forum/2004/4797.htm

Используемые слоги английского языка имеют вид V, CV, VC, CVV, VCC, CVC, CCV, CVCC, CVCCC, CCVCC, CCCVCC, где C -- согласная (consonant), V -- гласная (vowel). 70\% слогов имеют структуру VC или CVC. Общее число слогов $S = 8000 - 12000$. Средняя длина слога -- 3 буквы.

Предполагая равновероятное распределение всех слогов английского языка, для числа паролей из $r$ слогов получим верхнюю оценку
    \[ N_1 = S^r = 2^{13 r} \approx 2^{4.3 L_1}. \]
Средняя длина паролей составит
    \[ L_1 \approx 3 r. \]

Теперь предположим, что пароли могут состоять только из 2--3 буквенных слогов вида CV, VC, CVV, VCC, CVC, CCV с равновероятным распределением символов. Подсчитаем число паролей $N_2$, которые могут быть построены из $r$ таких слогов. В английском алфавите $n_v = 10, n_c = 16, n = n_v + n_c = 26$. Верхняя оценка числа $r$-слоговых паролей:
    \[ N_2 = (n_c n_v + n_v n_c + n_c n_v n_v + n_v n_c n_c + n_c n_v n_c + n_c n_c n_v)^r \approx \]
        \[ \approx \left( n_c n_v(3 n_c + n_v) \right)^r, \]
    \[ N_2 \approx \left( \frac{n^3}{2} \right)^r \approx 2^{13 r} \approx 2^{4.3 L_2}. \]
Средняя длина паролей:
    \[ L_2 = \frac{n_c n_v(2 + 2 + 3 n_v + 3 n_c + 3 n_c + 3 n_c)}{n_c n_v (1 + 1 + n_v + n_c + n_c + n_c)} \cdot r \approx 3 r. \]

Как видно, получились одинаковые оценки числа и длины паролей.

Подсчитаем верхние оценки числа паролей из $L$ символов, предполагая равномерное распределение символов из алфавита в $D$ символов: a) $D_1 = 26$ строчных буквы, б) все $D_2 = 94$ печатных символа клавиатуры (латиница и небуквенные символы):
    \[ N_3 = D_1^L \approx 2^{4.7 L}, \]
    \[ N_4 = D_2^L \approx 2^{6.6 L}. \]

\begin{table}[!ht]
    \centering
    \caption{Различные верхние оценки числа паролей\label{tab:password-number}}
    \resizebox{\textwidth}{!}{ \begin{tabular}{|c||c|c|c|}
        \hline
        \multirow{2}{*}{\parbox{1.5cm}{Длина пароля}} & \multicolumn{3}{|c|}{Число паролей} \\
        \cline{2-4}
            & \parbox{3cm}{На основе слоговой композиции} &
            \parbox{3cm}{Алфавит $D=26$ символов} &
            \parbox{3cm}{Алфавит $D=94$ символа} \\
        \hline \hline
        6  & $2^{26}$ & $2^{28}$ & $2^{39}$ \\
        9  & $2^{39}$ & $2^{42}$ & $2^{59}$ \\
        12 & $2^{52}$ & $2^{56}$ & $2^{79}$ \\
        15 & $2^{65}$ & $2^{71}$ & $2^{98}$ \\
        \hline
        21 & $2^{91}$ & $2^{99}$ & $2^{137}$ \\
        \hline
        39 & $2^{169}$ & $2^{183}$ & $2^{256}$ \\
        \hline
    \end{tabular} }
\end{table}

Из таблицы \ref{tab:password-number} видно, что при доступном объеме вычислений в $2^{60 \ldots 70}$ операций, пароли вплоть до 15 символов, построенные на словах, слогах, изменениях слов, вставках цифр, небольшом изменении регистров и других простейших модификациях, могут быть найдены перебором на кластере (или ПК) в настоящее время.

Для достижения криптостойкости паролей, сравнимой со 128- или 256-битовым секретным ключом, требуется выбирать пароль из 20 и 40 символов соответственно, что, как правило, не реализуется из-за сложности запоминания и ввода без ошибок.


%Подсчитаем число паролей $N_1$, которые могут могут построены из $r$ ~ 2-3 буквенных слогов: $cv, vc, ccv, cvc, vcc$, где $c$ -- согласная, $v$ -- гласная. В английском алфавите $n_v = 10, n_c = 16, n = n_v + n_c = 26$. Число паролей
%    \[ N_1 = \left( n_v n_c (1 + 1 + n_c + n_c + n_c) \right)^r \approx 3^r n_v^r n_c^{2r}. \]
%Средняя длина паролей
%    \[ L = r \left( \frac{2 + 2 + 3 n_c + 3 n_c + 3 n_c}{1 + 1 + n_c + n_c + n_c} \right) \approx 3r. \]
%
%%Учтем, что $b \leq r$ символов могут быть заглавными: $N_1 \rightarrow N_2 < N_1 \binom{L}{b} \left( \frac{n}{n_v} \right)^b$. Вставим $d$ цифр в случайные места: $N_2 \rightarrow N_3 = N_2 (10 (1 + L))^d \approx N_2 (10 L)^d$.
%%
%%Общее число паролей
%%    \[ N = N_3 = 3^r 10^r 16^{2r} \binom{3r}{b} 2.6^b \left(10 \cdot 3 r \right)^d. \]
%%
%%\begin{table}[!ht]
%%    \centering
%%    \small
%%    \begin{tabular}{|c|c|c|c|c||cr|}
%%        \hline
%%        \parbox{1.3cm}{Слогов, $r$} & \parbox{1.8cm}{Заглавных букв, $b$} & \parbox{1.5cm}{Вставок цифр, $d$} & \parbox{2.8cm}{Средняя длина пароля, $L+d$} & \parbox{3cm}{Верхняя оценка числа паролей $N$} & \multicolumn{2}{|c|}{\parbox{3.2cm}{Число всех паролей}} \\
%%        \hline
%%        $2$ & $0$ & $0$ & $6$ & $2^{26}$ & $2^{36}$ & a-z \\
%%        $2$ & $2$ & $0$ & $6$ & $2^{32}$ & $2^{48}$ & A-Z, a-z \\
%%        $2$ & $2$ & $2$ & $8$ & $2^{45}$ & $2^{48}$ & A-Z, a-z, 0-9 \\
%%        \hline
%%        $3$ & $0$ & $0$ & $9$ & $2^{39}$ & $2^{54}$ & a-z \\
%%        $3$ & $3$ & $0$ & $9$ & $2^{49}$ & $2^{54}$ & A-Z, a-z \\
%%        $3$ & $3$ & $2$ & $11$ & $2^{63}$ & $2^{65}$ & A-Z, a-z, 0-9 \\
%%        \hline
%%        $4$ & $0$ & $0$ & $12$ & $2^{52}$ & $2^{93}$ & a-z \\
%%        $4$ & $3$ & $0$ & $12$ & $2^{64}$ & $2^{186}$ & A-Z, a-z \\
%%        $4$ & $3$ & $2$ & $14$ & $2^{78}$ & $2^{222}$ & A-Z, a-z, 0-9 \\
%%        \hline
%%    \end{tabular}
%%    \caption{Сравнение верхней оценки числа паролей, построенных на слогах, со всем доступным множеством паролей.}
%%    \label{tab:password-number}
%%\end{table}
%
%Учтем, что $b$ символов в пароле могут быть взяты не из 26-символьного алфавита строчных букв, а из всего алфавита в $D=94$ печатных символа клавиатуры (латиница и небуквенные символы):
%\[
%    \begin{array}{ll}
%    b=1 & N_1 \rightarrow N_2 = \frac{n_v}{n_v+n_c} 3^r n_v^{r-1} n_c^{2r} \cdot L. \]
%
%    \[ N_1 \rightarrow N_2 < N_1 \binom{L}{b} \left( \frac{D}{n_v} \right)^b. \]
%
%
%
%Общее число паролей
%    \[ N < 3^r n_v^r n_c^{2r} \binom{L}{b} \left( \frac{D}{n_v} \right)^b = 3^r 10^r 16^{2r} \binom{3r}{b} \left( \frac{94}{10} \right)^b. \]
%
%\begin{table}[!ht]
%    \centering
%    \small
%    \begin{tabular}{|c|c|c|c||cr|}
%        \hline
%        \parbox{1.5cm}{Слогов, $r$} & \parbox{3cm}{Средняя длина пароля, $L$} & \parbox{3cm}{Символов из всего алфавита, $b$} & \parbox{3cm}{Верхняя оценка числа паролей $N$} & \multicolumn{2}{|c|}{\parbox{3.2cm}{Число всех паролей, $D^L$}} \\
%        \hline
%        \multirow{3}{*}{2} & \multirow{3}{*}{6} & $0$ & $2^{26}$ & $2^{28}$ & a-z \\
%        & & $1$ & $2^{32}$ & $2^{34}$ & A-Z, a-z \\
%        & & $3$ & $2^{40}$ & $2^{39}$ & Весь алфавит \\
%        \hline
%        \multirow{3}{*}{3} & \multirow{3}{*}{9} & $0$ & $2^{39}$ & $2^{42}$ & a-z \\
%        & & $2$ & $2^{50}$ & $2^{51}$ & A-Z, a-z \\
%        & & $4$ & $2^{59}$ & $2^{59}$ & Весь алфавит \\
%        \hline
%        \multirow{3}{*}{4} & \multirow{3}{*}{12} & $0$ & $2^{52}$ & $2^{56}$ & a-z \\
%        & & $3$ & $2^{69}$ & $2^{68}$ & A-Z, a-z \\
%        & & $6$ & $2^{81}$ & $2^{77}$ & Весь алфавит \\
%        \hline
%    \end{tabular}
%    \caption{Сравнение верхней оценки числа паролей, построенных на слогах, со всем доступным множеством паролей в алфавите из $D$ символов.}
%    \label{tab:password-number}
%\end{table}
%
%Из таблицы \ref{tab:password-number} видно, что при доступном объеме вычислений в $2^{60 \ldots 70}$ операций, пароли вплоть до 12 символов, построенные на словах, слогах, изменениях слов, вставках цифр, небольшого изменения регистров и другой простейшей обфускации, могут быть найдены перебором на кластере (или ПК) в настоящее время.


\subsubsection{Атака для подбора паролей и ключей шифрования}

В схемах аутентификации по паролю иногда используется хэширование и хранение хэша пароля на сервере. В таких случаях применима словарная атака или атака с применением заранее вычисленных таблиц для ускорения поиска.

Для нахождения пароля, прообраза хэш-функции, или для нахождения ключа блокового шифрования по атаке с выбранным шифротекстом (для одного и того же известного открытого текста и соответствующего шифротекста) может быть применен метод перебора с балансом между памятью и временем вычислений. Самый быстрый метод радужных таблиц (rainbow tables)\index{радужные таблицы}, 2003 г., заранее вычисляет следующие цепочки и хранит результат в памяти.

Для нахождения пароля, прообраза хэш-функции $H$, цепочка строится как
    \[ M_0 \xrightarrow{H(M_0)} h_0 \xrightarrow{R_0(h_0)} M_1 \ldots M_t \xrightarrow{H(M_t)} h_t \xrightarrow{R_t(h_t)} M_{t+1}, \]
где $R_i(h)$ -- функция редуцирования, преобразования хэша в пароль для следующего хэширования.

Для нахождения ключа блокового шифрования для одного и того же известного открытого текста $M$ таблица строится как
    \[ K_0 \xrightarrow{E_{K_0}(M)} c_0 \xrightarrow{R_0(c_0)} K_1 \ldots K_t \xrightarrow{E_{K_t}(M)} c_t \xrightarrow{R_t(c_t)} K_{t+1}, \]
где $R_i(c)$ -- функция редуцирования, преобразования шифротекста в новый ключ.

Функция редуцирования $R_i$ зависит от номера итерации, чтобы избежать дублирующиеся подцепочки, которые возникают в случае коллизий между значениями в разных цепочках в разных итерациях, если $R$ постоянна. Rainbow-таблица размера $(m \times 2)$ состоит из строк $(M_{0,j}, M_{t+1,j})$ или $(K_{0,j}, K_{t+1,j})$, вычисленных для разных значений стартовых паролей $M_{0,j}$ или $K_{0,j}$ соответственно.

Опишем атаку на примере нахождения прообраза $\overline{M}$ хэша $\overline{h} = H(\overline{M})$. На первой итерации исходный хэш $\overline{h}$ редуцируется в сообщение $\overline{h} \xrightarrow{R_t(\overline{h})} \overline{M}_{t+1} $ и сравнивается со всеми значениями последнего столбца $M_{t+1,j}$ таблицы. Если нет совпадения, переходим ко второй итерации. Хэш $\overline{h}$ дважды редуцируется в сообщение $\overline{h} \xrightarrow{R_{t-1}(\overline{h})} \overline{M}_t \xrightarrow{H(\overline{M}_t)} \overline{h}_t \xrightarrow{R_t(\overline{h}_t)} \overline{M}_{t+1}$ и сравнивается со всеми значениями последнего столбца $M_{t+1,j}$ таблицы. Если не совпало, то переходим к третьей итерации и т.д. Если для $r$-кратного редуцирования сообщение $\overline{M}_{t+1}$ содержится в таблице во втором столбце, то из совпавшей строки берется $M_{0,j}$, и вся цепочка пробегается заново для поиска искомого сообщения $\overline{M}: ~ \overline{h} = H(\overline{M})$.

Найдем вероятность нахождения пароля в таблице. Пусть мощность множества всех паролей $N$. Изначально в столбце $M_{0,j}$ содержится $m_0 = m$ различных паролей. Предполагая случайное отображение с пересечениями паролей $M_{0,j} \rightarrow M_{1,j}$, в $M_{1,j}$ будет $m_1$ различных паролей. Согласно задаче о размещении,
\[
    m_{i+1} = N \left( 1 - \left( 1 - \frac{1}{N} \right)^{m_i} \right) \approx N \left( 1 - e^{-\frac{m_i}{N}} \right).
\]
Вероятность нахождения пароля
\[
    P = 1 - \prod \limits_{i=1}^t \left( 1 - \frac{m_i}{N} \right).
\]

Чем больше таблица из $m$ строк, тем больше шансов найти пароль или ключ, выполнив в наихудшем случае   $O \left( m \frac{t(t+1)}{2} \right)$ операций.

Примеры применения атаки на хэш-функциях $\textrm{MD5}$\index{хэш-функция!MD5}, $\textrm{LM} \sim \textrm{DES}_{\textrm{Password}} (\textrm{const})$ приведены в табл. \ref{tab:rainbow-tables}.

\begin{table}[!ht]
    \centering
    \caption{Атаки на радужных таблицах на \emph{одном} ПК\label{tab:rainbow-tables}}
    \resizebox{\textwidth}{!}{ \begin{tabular}{|c|c|c|c|c|c|c|}
        \hline
        \multirow{2}{*}{\parbox{1.0cm}{Длина, биты}} & \multicolumn{3}{|c|}{Пароль или ключ} &
            \multicolumn{3}{|c|}{Радужная таблица} \\
        \cline{2-7}
        & \parbox{1.2cm}{Длина, симв.} & \parbox{1cm}{Множество} & \parbox{1cm}{Мощность} &
            Объем & \parbox{1.5cm}{Время вычисления таблиц} & \parbox{1.3cm}{Время поиска} \\
        \hline \hline
        \multicolumn{7}{|c|}{Хэш LM} \\
        \hline
        \multirow{3}{*}{$2 \times 56$} & \multirow{3}{*}{14} &
            A--Z & $2^{33}$ & 610 MB &  & 6 с \\
        & & A--Z, 0-9 & $2^{36}$ & 3 GB &  & 15 с \\
        & & все & $2^{43}$ & 64 GB & \parbox{1.5cm}{несколько лет} & 7 мин \\
        \hline \hline
        \multicolumn{7}{|c|}{Хэш MD5} \\
        \hline
        128 & 8 & a-z, 0-9 & $2^{41}$ & 36 GiB & - & 4 мин \\
        \hline
    \end{tabular} }
\end{table}


\section{Аутентификация по паролю}

Из-за малой энтропии пользовательских паролей во всех системах регистрации и аутентификации пользователей применяется специальная политика безопасности. Типичные минимальные требования:
\begin{enumerate}
    \item Длина пароля от 8 символов. Использование разных регистров и небуквенных символов в паролях. Запрет паролей из словаря слов или часто используемых паролей. Запрет паролей в виде дат, номеров машин и других номеров.
    \item Ограниченное время действия пароля. Обязательная смена пароля по истечении срока действия.
    \item Блокирование возможности аутентификации после нескольких неудачных попыток. Ограниченное число актов аутентификаций в единицу времени. Временная задержка перед выдачей результата аутентификации.
\end{enumerate}

Дополнительные рекомендации (требования) пользователям:
\begin{enumerate}
    \item Не использовать одинаковые или похожие пароли для разных систем. Например, электронная почта, вход в ОС, электронная платежная система, форумы, социальные сети. Пароль часто передается в открытом виде по сети. Пароль доступен администратору системы, возможны утечки конфиденциальной информации с серверов. Стараться выбирать случайные стойкие пароли.
    \item Не записывать пароли. Никому не сообщать пароль, даже администратору. Не передавать пароли по почте, телефону, Интернету и т.д.
    \item Не использовать одну и ту же учетную запись для разных пользователей даже в виде исключения.
    \item Всегда блокировать компьютер, когда пользователь отлучается от него даже на короткое время.
\end{enumerate}

\input{os_passwords}

\input{http_auth}

\chapter{Программные уязвимости}

\input{security_models}

\section{Контроль доступа в ОС}
\selectlanguage{russian}

\subsection{Windows}
%http://www.gentlesecurity.com/blog/andr/cracking_windows_access_control.pdf
%http://msdn.microsoft.com/en-us/library/bb250462(VS.85).aspx#upm_ovwim
%http://msdn.microsoft.com/en-us/library/bb625963.aspx
%http://msdn.microsoft.com/en-us/library/bb625964.aspx

Операционные системы Windows до Windows Vista использовали только дискреционную модель безопасности. Владелец файла имел возможность изменить права доступа или разрешить доступ другому пользователю.

Начиная с Windows Vista, в дополнение к стандартной дискреционной модели субъекты и объекты стали обладать мандатным уровнем доступа, устанавливаемым администратором (или по умолчанию системой для новых созданных объектов) и имеющим приоритет над стандартным дискреционным доступом, который может менять владелец.

В Vista мандатный уровень доступа предназначен в большей степени для обеспечения \emph{целостности} и устойчивости системы, чем для обеспечения секретности.

Уровень доступа объекта (integrity level в терминологии Windows) помечается шестнадцатеричным числом в диапазоне от \texttt{0} до \texttt{0x4000}, большее число означает более высокий уровень доступа. В Vista определены 5 базовых уровней:
\begin{itemize}
    \item ненадежный (Untrusted, \texttt{0x0000}),
    \item низкий (Low Integrity, \texttt{0x1000}),
    \item средний (Medium Integrity, \texttt{0x2000}),
    \item высокий (High Integrity, \texttt{0x3000}) и
    \item системный (System Integrity, \texttt{0x4000}).
\end{itemize}

Дополнительно объекты имеют три атрибута, которые, если они установлены, запрещают доступ субъектов с более низким уровнем доступа к ним: cубъекты с более низким уровнем доступа не могут
\begin{itemize}
    \item читать (no read-up),
    \item изменять (no write-up),
    \item исполнять (no execute-up)
\end{itemize}
объекты с более высоким уровнем доступа. Для всех объектов по умолчанию установлен атрибут запрета записи объектов с более высоким уровнем доступа, чем имеет субъект (no write-up).

Субъекты имеют два атрибута:
\begin{itemize}
    \item запрет записи объектов с более высоким уровнем доступа, чем у субъекта (no write-up, эквивалентно аналогичному атрибуту объекта),
    \item установка уровня доступа созданного процесса-потомка как минимума от уровня доступа родительского процесса (субъекта) и исполняемого файла (объекта файловой системы).
\end{itemize}
Оба атрибута установлены по умолчанию.

Все пользовательские данные и процессы по умолчанию имеют средний уровень доступа, а системные файлы -- системный. Например, если в Internet Explorer, который в защищенном (protected) режиме запускается с низким уровнем доступа, обнаружится уязвимость, злоумышленник не будет иметь возможности изменить системные данные на диске, даже если браузер запущен администратором.

Уровень доступа процесса соответствует уровню доступа пользователя (процесса), который запустил процесс. Например, пользователи LocalSystem, LocalService, NetworkService получают системный уровень, администраторы -- высокий, обычные пользователи системы -- средний, остальные (everyone) -- низкий.

По каким-то причинам, вероятно, в целях совместимости с ранее разработанными программами и/или для упрощения разработки и настройки новых сторонних программ других производителей, субъекты с системным, высоким и средним уровнями доступа создают объекты или владеют объектами со \emph{средним} уровнем доступа. И только субъекты с низким уровнем доступа создают объекты с низким уровнем доступа. Это означает, что системный процесс может владеть файлом или создать файл со средним уровнем доступа, и другой процесс с более низким уровнем доступа, например средним, может получить доступ к файлу, в т.ч. и на запись. Это нарушает принцип запрета записи в объекты, созданные субъектами с более высоким уровнем доступа.


\subsection{Linux}

Стандартная ОС Unix обеспечивает дискреционную модель контроля доступа на следующей основе.
\begin{itemize}
    \item Каждый субъект (процесс) и объект (файл) имеют владельца, пользователя и группу, которые могут изменять доступ к данному объекту для себя и других пользователей и групп.
    \item Каждый объект (файл) имеет атрибуты доступа на чтение (r), запись (w) и исполнение (x) для трех типов пользователей: владельца-пользователя (u), владельца-группы (g), остальных пользователей (o) -- (u:rwx, g:rwx, o:rwx).
    \item Субъект может входить в несколько групп.
\end{itemize}

В 2000 г. Агентство Национальной Безопасности США (NSA) выпустило набор изменений SELinux с открытым исходным кодом к ядру ОС Linux версии 2.4. Начиная с версии ядра 2.6, SELinux входит как часть стандартного ядра. SELinux реализует комбинацию ролевой, мандатной и дискреционной моделей контроля доступа, которые могут быть изменены только администратором системы (и/или администратором безопасности). По сути, SELinux каждому субъекту приписывает одну или несколько ролей, и для каждой роли указано, к объектам с какими атрибутами они могут иметь доступ и какого вида.

Основная проблема ролевых систем контроля доступа -- очень большой список описания ролей и атрибутов объектов, что увеличивает сложность системы и приводит к регулярным ошибкам в таблицах описания контроля доступа.


\section{Виды программных уязвимостей}

\textbf{Вирусом} называется самовоспроизводящаяся часть кода (подпрограмма)\index{вирус}, которая встраивается в носители (другие программы) для своего исполнения и распространения. Вирус не может исполняться и передаваться без своего носителя.

\textbf{Червем} называется самовоспроизводящаяся отдельная (под)программа\index{червь}, которая может исполняться и распространяться самостоятельно, не используя программу-носитель.

Первой вехой в изучении компьютерных вирусов можно назвать 1949 год, когда Джон фон Нейман прочёл курс лекций в Университете Иллинойса под названием <<Теория самовоспроизводящихся машин>> (изданы в 1966~\cite{Neumann:1966}, переведены на русский язык издательством <<Мир>> в 1971 году~\cite{Neumann:1971}), в котором ввёл понятие самовоспроизводящихся механических машин. Первым сетевым вирусом считается вирус Creeper 1971 г., распространявшийся в сети ARPANET, предшественнике Интернета. Для его уничтожения был создан первый антивирус, Reaper, который находил и уничтожал Creeper.

Первый червь для Интернета, червь Морриса 1988 г., уже использовал \emph{смешанные} атаки для заражения UNIX машин~\cite{EichinRochlis:1988}\cite{Spafford:1989}. Сначала программа получала доступ к удалённому запуску команд, эксплуатируя уязвимости в сервисах \texttt{sendmail}, \texttt{finger} (с использованием атаки переполнением буфера) или \texttt{rsh}. Далее с помощью механизма подбора паролей червь получал доступ к локальным аккаунтам пользователей:
\begin{itemize}
    \item получение доступа к учётным записям с очевидными паролями:
		\begin{itemize}
			\item без пароля вообще;
			\item имя аккаунта в качестве пароля;
			\item имя аккаунта в качестве пароля, повторенное дважды;
			\item использование <<ника>> (англ. <<nickname>>);
			\item фамилия (англ. <<lastname>>);
			\item фамилия, записанная задом наперёд;
		\end{itemize}
		\item перебор паролей на основе встроенного словаря из 432 слов;
		\item перебор паролей на основе системного словаря \texttt{/usr/dict/words}.
\end{itemize}

\textbf{Программной уязвимостью}\index{программная уязвимость} называется свойство программы, позволяющее нарушить ее работу. Программные уязвимости могут приводить к отказу в обслуживании (Denial of Service, DoS-атака)\index{атака!отказ в обслуживании}, утечке и изменению данных, появлению и распространению вирусов и червей.

Одной из распространенных атак для заражения персональных компьютеров является переполнение буфера в стеке. В интернет-сервисах наиболее распространенной программной уязвимостью в настоящее время является межсайтовый скриптинг (Cross-Site Scripting, XSS-атака)\index{атака!XSS}.

Наиболее распространенные программные уязвимости можно разделить на классы:
\begin{enumerate}
    \item Переполнение буфера -- копирование в буфер данных большего размера, чем длина выделенного буфера. Буфером может быть контейнер текстовой строки, массив, динамически выделяемая память и т.д. Переполнение становится возможным вследствие либо отсутствия контроля над длиной копируемых данных, либо из-за ошибок в коде. Типичная ошибка -- разница в 1 байт между размерами буфера и данных при сравнении.
    \item Некорректная обработка (парсинг) данных, введенных пользователем, является причиной большинства программных уязвимостей в веб-приложениях. Под обработкой понимаются:
        \begin{enumerate}
            \item проверка на допустимые значения и тип (числовые поля не должны содержать строки и т.д.);
            \item фильтрация и экранирование специальных символов, имеющих значения в скриптовых языках или для декодирования из одной текстовой кодировки в другую. Примеры символов: \texttt{\textbackslash},  \texttt{\%}, \texttt{<}, \texttt{>}, \texttt{"},  \texttt{'};
            \item фильтрация ключевых слов языков разметки и скриптов. Примеры: \texttt{script}, \texttt{JavaScript};
            \item декодирование различными кодировками при парсинге. Распространенный способ обхода системы контроля парсинга данных состоит в однократном или множественном последовательном кодировании текстовых данных в шестнадцатеричные кодировки \texttt{\%NN} ASCII и UTF-8. Например, браузер или веб-приложения производят $n$ -- кратные последовательные декодирования, в то время как система контроля делает $k$-кратное декодирование, $0 \leq k < n$, и, следовательно, пропускает закодированные запрещенные символы и слова.
        \end{enumerate}
    \item Некорректное использование синтаксиса функций. Например, \texttt{printf(s)} может привести к уязвимости записи в указанный адрес памяти. Если злоумышленник вместо обычной текстовой строки введет в качестве \texttt{s = "текст некоторой длины\%n"}, то функция \texttt{printf()}, ожидающая первым аргументом строку формата \texttt{printf(fmt, \dots)}, обнаружив \texttt{\%n}, возьмет значения из ячеек памяти, следующих перед текстовой строкой (устройство стека функции описано далее), и запишет в адрес памяти, равный считанному значению, количество выведенных символов на печать функцией \texttt{printf()}.
\end{enumerate}


\section[Переполнение буфера в стеке с исполнением кода]{Переполнение буфера в стеке с \protect\\ исполнением кода}
\selectlanguage{russian}

В качестве примера переполнения буфера опишем самую распространенную атаку, направленную на исполнение кода злоумышленника.

В 64-битовой x86\_64 архитектуре основное пространство виртуальной памяти процесса из 16 эксабайтов ($2^{64}$ байт) свободно, и только малая часть занята (выделена). Виртуальная память выделяется процессу операционной системой блоками по 4 Кб, называемыми страницами памяти. Выделенные страницы соответствуют страницам физической оперативной памяти или страницам файлов.

Пример выделенной виртуальной памяти процесса представлен в табл.~\ref{tab:virtual-memory}. Локальные переменные функций хранятся в области памяти, называемой стеком.
\begin{table}[!ht]
    \centering
    \caption{Пример структуры виртуальной памяти процесса\label{tab:virtual-memory}}
    \resizebox{\textwidth}{!}{ \begin{tabular}{r|c|}
        \multicolumn{2}{c}{Адрес ~~~~~~~~~~~~~~ Использование} \\
        \cline{2-2}
        \texttt{0x00000000 00000000} & \\
        & \\
        \cdashline{2-2}
        \texttt{0x00000000 0040063F} & \multirow{2}{*}{\parbox{6cm}{Исполняемый код, динамические библиотеки}} \\
        & \\
        \cdashline{2-2}
        & \\
        & \\
        & \\
        \cdashline{2-2}
        \texttt{0x00000000 0143E010} & \multirow{2}{*}{Динамическая память} \\
        & \\
        \cdashline{2-2}
        & \\
        & \\
        & \\
        \cdashline{2-2}
        \texttt{0x00007FFF A425DF26} & \multirow{2}{*}{Переменные среды} \\
        & \\
        \cdashline{2-2}
        & \\
        & \\
        & \\
        \cdashline{2-2}
        \texttt{0x00007FFF FFFFEB60} & \multirow{2}{*}{Стек функций} \\
        & \\
        \cdashline{2-2}
        & \\
        & \\
        \texttt{0xFFFFFFFF FFFFFFFF} & \\
        \cline{2-2}
    \end{tabular} }
\end{table}

Приведем пример переполнения буфера в стеке\index{стек}, которое дает возможность исполнить код, для 64-разрядной ОС Linux. Ниже приводится листинг исходной программы, которая печатает расстояние Хэмминга между векторами $b1 = \text{\texttt{0x01234567}}$ и $b2 = \text{\texttt{0x89ABCDEF}}$.

\begin{verbatim}
#include <stdio.h>
#include <string.h>

int hamming_distance(unsigned a1, unsigned a2, char *text,
                     size_t textsize) {
  char buf[32];
  unsigned distance = 0;
  unsigned diff = a1 ^ a2;
  while (diff) {
    if (diff & 1) distance++;
    diff >>= 1;
  }
  memcpy(buf, text, textsize);
  printf("%s: %i\n", buf, distance);
  return distance;
}

int main() {
  char text[68] = "Hamming";
  unsigned b1 = 0x01234567;
  unsigned b2 = 0x89ABCDEF;
  return hamming_distance(b1, b2, text, 8);
}
\end{verbatim}

Вывод программы при запуске:
\begin{verbatim}
$ ./hamming
Hamming: 8
\end{verbatim}

При вызове вложенных функций вызывающая функция выделяет стековый кадр для вызываемой функции в сторону уменьшения адресов. Стековый кадр в порядке уменьшения адресов состоит из следующих частей:
\begin{enumerate}
    \item Аргументы вызова функции, расположенные в порядке уменьшения адреса (за исключением тех, которые передаются в регистрах процессора).
    \item Сохраненный регистр процессора \texttt{rip} внешней функции, также называемый адресом возврата. Регистр процессора \texttt{rip} содержит адрес следующей инструкции для исполнения. При входе во вложенную функцию адрес инструкции текущей функции запоминается в стеке, в регистре записывается новое значение адреса первой инструкции из вложенной функции, а по завершении функции регистр восстанавливается из стека, и, таким образом, исполнение возвращается назад.
    \item Сохраненный регистр процессора \texttt{rbp} внешней функции. Регистр процессора \texttt{rbp} содержит адрес сохраненного регистра \texttt{rbp} в стековом кадре вызывающей функции. Процессор обращается к локальным переменным функций по смещению относительно регистра \texttt{rbp}. При вызове вложенной функции регистр сохраняется в стеке, в регистр записывается текущее значение адреса стека (\texttt{rsp}), а по завершении функции регистр восстанавливается.
    \item Локальные переменные, как правило расположенные в порядке уменьшения адреса при объявлении новой переменной (порядок может быть изменен в результате оптимизаций и использования механизмов защиты, таких как Stack Smashing Protection в компиляторе GCC).
\end{enumerate}

Адрес начала стека, а также, возможно, адреса локальных массивов и переменных выровнены на границу параграфа в 16 байт, из-за чего в стеке могут образоваться неиспользуемые байты.

Если в программе есть ошибка, которая может привести к переполнению выделенного буфера в стеке при копировании, есть возможность записать вместо сохраненного значения регистра -- \texttt{rip} новое. В результате по завершении данной функции исполнение начнется с указанного адреса. Если есть возможность записать в переполняемый буфер исполняемый код, а затем на место сохраненного регистра \texttt{rip} адрес на этот код, то получим исполнение заданного кода в стеке функции.

На рис.~\ref{fig:stack-overflow} приведены исходный стек и стек с переполненным буфером, из-за которого записалось новое сохраненное значение \texttt{rip}.

\begin{figure}[!ht]
	\centering
	\includegraphics[width=0.95\textwidth]{pic/stack-overflow}
	\caption{Исходный стек и стек с переполнением буфера\label{fig:stack-overflow}}
\end{figure}


Изменим программу для демонстрации, поместив в копируемую строку исполняемый код для вызова \texttt{/bin/sh}.
{ \small
\begin{verbatim}
...
int main() {
  char text[68] =
    // 28 байт исполняемого кода
    "\x90" "\x90" "\x90"                // nop; nop; nop
    "\x48\x31" "\xD2"                   // xor %rdx, %rdx
    "\x48\x31" "\xF6"                   // xor %rsi, %rsi
    "\x48\xBF" "\xDC\xEA\xFF\xFF"
    "\xFF\x7F\x00\x00"                  // mov $0x7fffffffeadc,
                                        //   %rdi
    "\x48\xC7\xC0" "\x3B\x00\x00\x00"   // mov $0x3b, %rax
    "\x0F\x05"                          // syscall
    // 8 байт строки /bin/sh
    "\x2F\x62\x69\x6E\x2F\x73\x68\x00"  // "/bin/sh\0"
    // 12 байт заполнения и 16 байт новых
    // значений сохраненных регистров
    "\x00\x00\x00\x00"                  // не занятые байты
    "\x00\x00\x00\x00"                  // unsigned distance
    "\x00\x00\x00\x00"                  // unsigned diff
    "\x50\xEB\xFF\xFF"                  // регистр
    "\xFF\x7F\x00\x00"                  //   rbp=0x7fffffffeb50
    "\xC0\xEA\xFF\xFF"                  // регистр
    "\xFF\x7F\x00\x00"                  //   rip=0x7fffffffeac0
    ;
  ...
  return hamming_distance(b1, b2, text, 68);
  ...
}
\end{verbatim} }

Код эквивалентен вызову функции \texttt{execve(``/bin/sh'', 0 0)} через системный вызов функции ядра Linux для запуска оболочки среды \texttt{/bin/sh}. При системном вызове нужно записать в регистр \texttt{rax} номер системной функции, а в другие регистры процессора -- аргументы. Данный системный вызов с номером \texttt{0x3b} требует в качестве аргументов регистры \texttt{rdi} с адресом строки исполняемой программы, \texttt{rsi} и \texttt{rdx} с адресами строк параметров запускаемой программы и переменных среды. В примере в \texttt{rdi} записывается адрес \texttt{0x7fffffffeadc}, который указывает на строку \texttt{``/bin/sh''} в стеке после копирования. Регистры \texttt{rdx} и \texttt{rsi} обнуляются.

На рис.~\ref{fig:stack-overflow} приведен стек с переполненным буфером, в результате которого записалось новое сохраненное значение \texttt{rip}, указывающее на заданный код в стеке.

Начальные инструкции \texttt{nop} с кодом \texttt{0x90} означают пустые операции. Часто точные значения адреса и структуры стека неизвестны, поэтому злоумышленник угадывает предполагаемый адрес стека. В начале исполняемого кода создается массив из операций \texttt{nop} с надеждой, что предполагаемое значение стека, то есть требуемый адрес rip, попадет на эти операции, повысив шансы угадывания. Стандартная атака на переполнение буфера с исполнением кода также подразумевает последовательный перебор предполагаемых адресов для нахождения правильного адреса для \texttt{rip}.

В результате переполнения буфера в примере по завершении функции \texttt{hamming\_distance()} начнет исполняться инструкция с адреса строки \texttt{buf}, то есть заданный код.


\subsection{Защита}

Самый лучший способ защиты от атак переполнения буфера -- создание программного кода со слежением за размером данных и длиной буфера. Однако ошибки все равно происходят.

Существует три стандартных способа защиты от исполнения кода в стеке в архитектуре x86.

\begin{enumerate}
    \item Все 64-разрядные x86 процессоры включают поддержку NX-бита (non-execution)\index{NX-бит}. В таблице виртуальной памяти, выделенной процессу, каждая страница маркирована битом, называемым NX-битом и указывающим на то, может ли данная страница памяти содержать исполняемый код или нет. Преобразование адресов из виртуальных в адреса физической памяти выполняется процессором на основании таблицы виртуальной памяти процесса. Процессор, считывая в том числе значение NX-бита, запрещает исполнение кода из страниц данных и вызывает критическую ошибку сегментирования (segmentation fault).

        Последние версии ядер ОС поддерживают маркирование страниц выделяемой памяти. Маркирование производится исходя из того содержит страница памяти исполняемый код программы или нет. Приведенный выше пример исполнения кода в стеке не будет работать в 64-битовой ОС Linux последних версий при стандартных настройках.
        %Тем не менее, есть программы, динамически формирующие код во время выполнения, для которых NX-бит не используется

    \item Второй стандартный способ -- вставка проверочных символов (называемых canaries, guards) после массивов и в конце стека и их проверка перед выходом из функции. Если произошло переполнение буфера, программа аварийно завершится.

    \item Третий способ -- рандомизация адресного пространства (Address Space Layout Randomization, ASLR), то есть случайное расположение стека, кода и т.д. В настоящее время используется в большинстве современных операционных систем (OpenBSD, Linux, Windows). Это приводит к маловероятному угадыванию адресов и значительно усложняет использование уязвимости.
\end{enumerate}


\subsection{Другие атаки с переполнением буфера}

Почти любую возможность для переполнения буфера в стеке или динамической памяти можно использовать для получения критической ошибки в программе из-за обращения к адресам виртуальной памяти, страницы которых не были выделены процессу. Следовательно, можно проводить атаки отказа в обслуживании (Denial of Service (DoS) атаки).

Переполнение буфера в динамической памяти в случае хранения в ней адресов для вызова функций может привести к подмене адресов и исполнению другого кода.

В описанных DoS-атаках NX-бит не защищает систему.


\input{xss}

\section[SQL-инъекции с исполнением кода веб-сервером]{SQL-инъекции с исполнением кода \protect\\ базой данных интернет-сервиса}
\selectlanguage{russian}

Второй классической уязвимостью веб-приложений являются SQL-инъекции, когда пользователь имеет возможность поменять смысл запроса к базе данных веб-сервера. Запрос делается в виде текстовой строки на скриптовом языке SQL. Например, выражение
\begin{verbatim}
s = "SELECT * FROM Users WHERE Name = '" + username + "';"
\end{verbatim}
предназначено для получения информации о пользователе \texttt{username}. Однако, если пользователь вместо имени введет строку вида
\begin{center} \begin{verbatim}
john';  DELETE * FROM Users;  SELECT * FROM Users WHERE
  Name = 'john,
\end{verbatim} \end{center}
то выражение превратится в три SQL-операции:
%{\color{red} Проверить в каких системах будет работать. В JDBC PrepareStatement требует одно выражение, а ExecutableStatement использовать нельзя, так-как он не возвращает значения.}
\begin{verbatim}
-- запрос о пользователе john
SELECT * FROM Users WHERE Name = 'john';
-- удаление всех пользователей
DELETE FROM Users;
-- запрос о пользователе john
SELECT * FROM Users WHERE Name = 'john';
\end{verbatim}
При выполнении этого SQL-запроса к базе данных все записи пользователей будут удалены.

Уязвимости в SQL-выражениях являются частным случаем уязвимостей, связанных с использованием сложных систем с разными языками управления данными, и, следовательно, с разными системами экранирования специальных символов и контроля над типом данных. Когда веб-сервер принимает от клиента данные, закодированные, обычно, с помощью <<application/x-www-form-urlencoded>>~\cite{html4:1999}, специальные символы (пробелы, неалфавитные символы и т.д.) корректно экранируются браузером и восстанавливаются непосредственно веб-сервером либо стандартными программными библиотеками. Аналогично, когда SQL-сервер передаёт или принимает данные от клиентской библиотеки, внутренним протоколом общения с SQL-сервером происходит кодировка текста, который является частью пользовательских данных. Однако на стыке контекстов -- в тот момент, когда программа, выполняющаяся на веб-сервере, уже приняла данные от пользователя по HTTP-протоколу\index{протокол!HTTP} и собирается передать их SQL-серверу в качестве составной части SQL-команды -- перед программистом стоит сложная задача учёта, в худшем случае, трёх контекстов и кодировок: входного контекста протокола общения с клиентом (HTTP), контекста языка программирования (с соответствующим оформлением и экранированием специальных символов в текстовых константах) и контекста языка управления данными SQL-сервера.

Ситуация усложняется тем, что программист может являться специалистом в языке программирования, но может быть не знаком с особенностями языка SQL или, что чаще, конкретным диалектом языка SQL, используемым СУБД.

Метод защиты также заключается в \emph{разделении} кода и данных. Для защиты от приведенных атак на базу данных следует использовать параметрические запросы к базе данных с \emph{фиксированным} SQL-выражением. Например, в JDBC~\cite{jdbc:2006}:
\begin{verbatim}
PreparedStatement p = conn.prepareStatement(
    "SELECT * FROM Users WHERE Name=?");
p.setString(1, username);
\end{verbatim}

Таким образом задача корректного оформления текстовых данных для передачи на SQL-сервер перекладывается на драйвер общения с СУБД, в котором эта задача, обычно, решена корректно авторами драйвера, хорошо знающими особенности протокола и языка управления данными сервера.


%\chapter{Послесловие}
%Это должно быть что-то в виде заключения, объяснения, почему именно эти темы выбраны, насколько актуален материал с теоретической и практической точки зрения.


\appendix

\chapter{Математическое приложение}\label{chap:discrete-math}

\section{Общие определения}

Выражением $\mod n$ обозначается вычисление остатка от деления произвольного целого числа на целое число $n$. В полиномиальной арифметике эта операция означает вычисление остатка от деления многочленов.
%далее будем обозначать целые числа или операции с целыми числами, взятыми \textbf{по модулю}\index{модуль} целого числа $n$ (остаток от целочисленного деления). Примеры выражений:
    \[ a\mod n, \]
    \[ (a + b) c\mod n. \]
Равенство
    \[ a = b \mod n \]
означает, что выражения $a$ и $b$ равны (говорят также <<сравнимы>>, <<эквивалентны>>) по модулю $n$.

Множество
    \[ \{ 0, 1, 2, 3,  \dots,  n-1 \mod n\} \]
состоит из $n$ элементов, где каждый элемент $i$ представляет все целые числа, сравнимые с $i$ по модулю $n$.

Наибольший общий делитель (НОД) двух чисел $a,b$ обозначается $\gcd(a,b)$ (greatest common divisor).

Два числа $a,b$ называются взаимно простыми, если они не имеют общих делителей, $\gcd(a,b) = 1$.

Выражение $a \mid b$ означает, что $a$ делит $b$.

\input{birthdays_paradox}

\section{Группы}
\selectlanguage{russian}

\subsection{Свойства групп}

\textbf{Группой}\index{группа} называется множество $\Gr$, на котором задана бинарная операция <<$\cdot$>>, удовлетворяющая следующим аксиомам:
\begin{enumerate}
    \item замкнутость
        \[ \forall a,b \in \Gr: a \cdot b = c \in \Gr; \]
    \item ассоциативность
        \[ \forall a,b,c \in \Gr: (a \cdot b) \cdot c = a \cdot (b \cdot c); \]
    \item существование единичного элемента
        \[ \exists ~ e \in \Gr: e\cdot a = a \cdot e = a; \]
    \item существование обратного элемента
        \[ \forall a \in \Gr ~ \exists ~ b \in \Gr: a \cdot b = b \cdot a = e. \]
\end{enumerate}
Если
    \[ \forall a,b \in \Gr: a \cdot b = b \cdot a, \]
то группа коммутативная.

Если операция в группе задана как умножение $\cdot$, то группа называется \textbf{мультипликативной}, $e = 1$, обратный элемент -- $a^{-1}$, возведение в степень $k$ -- $a^k$.

Если операция задана как сложение $+$, то группа называется \textbf{аддитивной}, $e = 0$, обратный элемент $-a$, сложение $k$ раз -- $ka$.

Подмножество группы, удовлетворяющее аксиомам группы, называется \textbf{подгруппой}\index{подгруппа}.

\textbf{Порядком} $|\Gr|$ \textbf{группы}\index{порядок группы} $\Gr$ называется число элементов в группе. Пусть группа мультипликативная. Для любого элемента $a \in \Gr$ выполняется $a^{|\Gr|} = 1$.

\textbf{Порядком элемента} $a$ называется минимальное число
    \[ ord(a): a^{ord(a)} = 1 \]
 Порядок элемента делит порядок группы:
    \[ ord(a) \mid \left|\Gr\right|. \]


\subsection{Циклические группы}

\textbf{Генератором} $g \in \Gr$ называется элемент, \emph{порождающий} всю группу\index{генератор группы}
    \[ \Gr = \{g, g^2, g^3,  \ldots,  g^{|\Gr|} = 1\}. \]
Группа, в которой существует генератор, называется \textbf{циклической}\index{группа!циклическая}.

Если конечная группа нециклическая, то в ней существуют циклические подгруппы, порожденные всеми элементами. Любой элемент $a$ группы порождает либо циклическую \emph{подгруппу}
    \[ \{ a, a^2, a^3,  \dots,  a^{ord(a)} = 1 \} \]
порядка $ord(a)$, если порядок элемента $ord(a) < |\Gr|$, либо \emph{всю} группу
    \[ \Gr = \{ a, a^2, a^3,  \dots,  a^{|\Gr|} = 1 \}, \]
если порядок элемента равен порядку группы $ord(a) = |\Gr|$. Порядок любой подгруппы, как и порядок элемента, делит порядок всей группы.

Представим циклическую группу через генератор $g$ как
    \[ \Gr = \{g, g^2,  \ldots,  g^{|\Gr|} = 1\} \]
и каждый элемент $g^i$  возведем в степени $1, 2,  \ldots,  |\Gr|$. Тогда
\begin{itemize}
    \item элементы $g^i$, для которых число $i$ взаимно просто с $|\Gr|$, породят снова всю группу
            \[ \Gr = \{ g^i, g^{2i}, g^{3i},  \dots,  g^{|\Gr| i} = 1 \}, \]
        так как степени $\{i, 2i, 3i, \dots |\Gr| i \}$ по модулю $|\Gr|$ образуют перестановку чисел $\{1, 2, 3, \dots, |\Gr|\}$; следовательно, $g^i$ -- тоже генератор, число таких чисел $i$ по определению функции Эйлера $\varphi(|\Gr|)$ ($\varphi(n)$ -- количество взаимно простых с $n$ целых чисел в диапазоне $[1,n-1]$);
    \item элементы $g^i$, для которых $i$ имеют общие делители
            \[ d_i = \gcd(i, |\Gr|) \neq 1 \]
        c $|\Gr|$, породят подгруппы
            \[ \{ g^i, g^{2i}, g^{3i},  \dots,  g^{\frac{i}{d_i} |\Gr|} = 1\}, \]
        так как степень последнего элемента кратна $|\Gr|$; следовательно, такие $g^i$ образуют циклические подгруппы порядка $d_i$.
\end{itemize}
%TODO Гашков, Болотов, Часовских "Эллиптическая криптография" или "Методы элл. кри-ии"

Из предыдущего утверждения следует, что число генераторов в циклической группе равно
    \[ \varphi(|\Gr|). \]

Для проверки, является ли элемент генератором всей группы, требуется знать разложение порядка группы $|\Gr|$ на множители. Далее элемент возводится в степени, равные всем делителям порядка группы, и сравнивается с единичным элементом $e$. Если ни одна из степеней не равна $e$, то этот элемент является примитивным элементом, или генератором группы. В противном случае элемент будет генератором какой-либо подгруппы.

Задача разложения числа на множители является трудной для вычисления. На сложности ее решения, например, основана криптосистема RSA\index{криптосистема!RSA}. Поэтому при создании больших групп желательно заранее знать разложение порядка группы на множители для возможности выбора генератора.


\subsection{Группа $\Z_p^*$}

\textbf{Группой $\Z_p^*$} называется группа\index{группа!$\Z_p^*$}
    \[ \Z_p^* = \{1, 2,  \dots,  p-1 \mod p\}, \]
где $p$ -- простое число, операция в группе -- умножение $\ast$ по $\mod p$.

Группа $\Z_p^*$ -- \textbf{циклическая}, порядок
    \[ |\Z_p^*| = \varphi(p) = p - 1. \]
Число генераторов в группе --
    \[ \varphi(|\Z_p^*|) = \varphi(p-1). \]

Из того, что $\Z_p^*$ -- группа, для простого $p$ и любого $a \in [2, p-1] \mod p$ следует \textbf{малая теорема Ферма}\index{теорема!Ферма малая}:
    \[ a^{p-1} = 1 \mod p. \]
На малой теореме Ферма основаны многие тесты проверки числа на простоту.

\example
Рассмотрим группу $\Z_{19}^*$. Порядок группы -- 18. Делители: 2, 3, 6, 9. Является ли 12 генератором?
\[ \begin{array}{l}
    12^2 = -8 \mod 19, \\
    12^3 = -1 \mod 19, \\
    12^6 = 1 \mod 19, \\
\end{array} \]
12 -- генератор подгруппы 6 порядка. Является ли 13 генератором?
\[ \begin{array}{l}
    13^2 = -2 \mod 19, \\
    13^3 = -7 \mod 19, \\
    13^6 = -8 \mod 19, \\
    13^9 = -1 \mod 19, \\
    13^{18} = 1 \mod 19, \\
\end{array} \]
13 -- генератор всей группы.
\exampleend

\example
В таб. \ref{tab:Zp-sample} приведен пример группы $\Z_{13}^*$. Число генераторов -- $\varphi(12) = 4$. Подгруппы --
    \[ \Gr^{(1)}, \Gr^{(2)}, \Gr^{(3)}, \Gr^{(4)}, \Gr^{(6)}, \]
верхний индекс обозначает порядок подгруппы.

\begin{table}[!ht]
    \centering
    \caption {Генераторы и циклические подгруппы группы $\Gr=\Z_{13}^*$\label{tab:Zp-sample}}
    \resizebox{\textwidth}{!}{ \begin{tabular}{|c|p{0.66\textwidth}|c|}
        \hline
        Элемент & Порождаемая группа или подгруппа & Порядок \\
        \hline
        1 & $\Gr^{(1)} = \{ 1 \}$ & 1 \\
        2 & $\Gr = \{ 2, 4,  8 = -5, -10 = 3, 6, 12 = -1, -2, -4, -8 = 5, 10 = -3, -6, -12 = 1 \}$ & 12, ген. \\
        3 & $\Gr^{(3)} = \{ 3, 9 = -4, -12 = 1 \}$ & 3 \\
        4 & $\Gr^{(6)} = \{ 4, 16 = 3, 12 = -1, -4, -3, -12 = 1 \}$ & 6 \\
        5 & $\Gr^{(4)} = \{ 5, 25 = -1, -5, 1 \}$ & 4 \\
        6 & $\Gr = \{6, 36 = -3, -5, -4, 2, -1, -6, 3, 5, 4, -2, -12 = 1 \}$ & 12, ген. \\
        7 = -6 & $\Gr = \{ -6, 36 = -3, 5, -4, -2, -1, 6, 3, -5, 4, 2, -12 = 1 \}$ & 12, ген. \\
        8 = -5 & $\Gr^{(4)} = \{ -5, 25 = -1, 5, 1 \}$ & 4 \\
        9 = -4 & $\Gr^{(3)} = \{ -4, 16 = 3, -12 = 1 \}$ & 3 \\
        10 = -3 & $\Gr^{(6)} = \{ -3, 9 = -4, 12 = -1, 3, -9 = 4, -12 = 1 \}$ & 6 \\
        11 = -2 & $\Gr = \{ -2, 4, 5, 3, -6, -1, 2, -4, -5, -3, 6, -12 = 1 \}$ & 12, ген. \\
        12 = -1 & $\Gr^{(2)} = \{ -1, 1 \}$ & 2 \\
        \hline
    \end{tabular} }
\end{table}
\exampleend


\subsection{Группа $\Z_n^*$}

\textbf{Функция Эйлера}\index{функция!Эйлера} $\varphi(n)$ определяется как количество чисел, взаимно простых с $n$ , в интервале от 1 до $n-1$.

Если $n=p$ -- простое число, то
    \[ \varphi(p) = p - 1, \]
    \[ \varphi(p^k) = p^k - p^{k-1} = p^{k-1}(p - 1). \]
Если $n$ -- составное число и
    \[ n = \prod \limits_{i} p_i^{k_i} \]
разложено на простые множители $p_i$, то
    \[ \varphi(n) = \prod \limits_{i} \varphi(p_i^{k_i}) =  \prod \limits_{i} p_i^{k_i - 1}(p_i - 1). \]

\textbf{Группой $\Z_n^*$} называется группа\index{группа!$\Z_n^*$}
    \[ \Z_n^* = \left\{ \forall a \in \left\{ 1, 2,  \dots,  n-1 \mod n \right\} : \gcd(a,n) = 1 \right\}, \]
с операцией умножения $\ast$ по $\mod n$.

Порядок группы --
    \[ |\Z_n^*| = \varphi(n). \]
Группа $\Z_p^*$ -- частный случай группы $\Z_n^*$.

Если $n$ \emph{составное} (не простое) число, то группа $\Z_n^*$ \textbf{нециклическая}.

Из того, что $\Z_n^*$ -- группа, для любых $a \neq 0, n > 1: \gcd(a,n) = 1$ следует \textbf{теорема Эйлера}\index{теорема!Эйлера}:
    \[ a^{\varphi(n)} = 1 \mod n. \]

При возведении в степень, если $\gcd(a,n) = 1$, выполняется
    \[ a^b = a^{b \mod \varphi(n)} \mod n. \]

\example
В табл. \ref{tab:Zn-sample} приведена нециклическая группа $\Z_{21}^*$ и ее циклические подгруппы
    \[ \Gr^{(1)}, \Gr_1^{(2)}, \Gr_2^{(2)}, \Gr_3^{(2)}, \Gr_1^{(3)}, \Gr_1^{(6)}, \Gr_2^{(6)}, \Gr_3^{(6)}, \]
верхний индекс обозначает порядок подгруппы, нижний индекс нумерует различные подгруппы одного порядка.

\begin{table}[!ht]
    \centering
    \caption{Циклические подгруппы нециклической группы $\Z_{21}^*$\label{tab:Zn-sample}}
    \begin{tabular}{|c|l|c|}
        \hline
        Элемент & Порождаемая циклическая подгруппа & Порядок \\
        \hline
        1  & $\Gr^{(1)} = \{ 1 \}$ & 1 \\
        2  & $\Gr_1^{(6)} = \{ 2, 4, 8, 16, 11, 1 \}$ & 6 \\
        4  & $\Gr_1^{(3)} = \{ 4, 16, 1 \}$ & 3 \\
        5  & $\Gr_2^{(6)} = \{ 5, 4, 20, 16, 17, 1 \}$ & 6 \\
        8  & $\Gr_1^{(2)} = \{ 8, 1 \}$ & 2 \\
        10 & $\Gr_3^{(6)} = \{ 10, 16, 13, 4, 19, 1 \}$ & 6 \\
        11 & $\Gr_1^{(6)} = \{ 11, 16, 8, 4, 2, 1 \}$ & 6 \\
        13 & $\Gr_2^{(2)} = \{ 13, 1 \}$ & 2 \\
        16 & $\Gr_1^{(3)} = \{ 16, 4, 1 \}$ & 3 \\
        17 & $\Gr_2^{(6)} = \{ 17, 16, 20, 4, 5, 1 \}$ & 6 \\
        19 & $\Gr_3^{(6)} = \{ 19, 4, 13, 16, 10, 1 \}$ & 6 \\
        20 & $\Gr_3^{(2)} = \{ 20, 1 \}$ & 2 \\
        \hline
    \end{tabular}
\end{table}
\exampleend

\subsection{Конечные поля}

\textbf{Полем} называется множество $\F$, для которого\index{поле}:
\begin{itemize}
    \item заданы две бинарные операции, условно называемые операциями умножения <<$\cdot$>> и сложения <<$+$>>;
    \item выполняются аксиомы группы для операции <<сложения>>: \\
        1. замкнутость:
		\[\forall a, b \in \F: a + b \in \F;\]
        2. ассоциативность:
		\[\forall a, b, c \in \F: (a+b)+c = a+(b+c);\]
        3. существование нейтрального элемента по сложению (часто обозначаемого как <<0>>):
		\[\exists 0 \in \F: \forall a \in \F: a + 0 = 0 + a = a; \]
        4. существование обратного элемента:
		\[\forall a \in \F: \exists -a: a + (-a) = 0; \]
    \item выполняются аксиомы группы для операции <<умножения>>, за одним исключением: \\
        1. замкнутость:
		\[\forall a, b \in \F: a \cdot b \in \F; \]
        2. ассоциативность:
		\[\forall a, b, c \in \F: (a \cdot b) \cdot c = a \cdot (b \cdot c);\]
        3. существование нейтрального элемента по умножению (часто обозначаемого как <<1>>):
		\[\exists 1 \in \F: \forall a \in \F: a \cdot 1 = 1 \cdot a = a;\]
        3. существование обратного элемента по умножению для всех элементов множества, кроме нейтрального элемента по сложению:
		\[\forall a \in {\F \backslash 0}: \exists a^{-1}: a \cdot a^{-1} = a^{-1} \cdot a = 1;\]
    \item операции <<сложения>> и <<умножения>> коммутативны
        \[ \begin{array}{l}
            \forall a, b \in \F: a + b = b + a, \\
            \forall a, b \in \F: a \cdot b = b \cdot a; \\
        \end{array} \]
    \item выполняется свойство дистрибутивности
        \[ \forall a, b, c \in \F: a \cdot (b + c) = (a \cdot b) + (a \cdot c). \]
\end{itemize}

Примеры \emph{бесконечных} полей (с бесконечным числом элементов) -- поле рациональных чисел $\group{Q}$, поле вещественных чисел $\group{R}$, поле комплексных чисел $\group{C}$ с обычными операциями сложения и умножения.

В криптографии рассматриваются \emph{конечные} поля (с конечным числом элементов), называемые также \textbf{полями Галуа}.

Число элементов в любом конечном поле равно $p^n$, где $p$ -- простое число и $n$ -- натуральное число. Обозначения поля Галуа: $\GF{p}, \GF{p^n}, \F_p, \F_{p^n}$ (аббревиатура от Galois field). Все поля Галуа $\GF{p^n}$ изоморфны друг другу (существует взаимно однозначное отображение между полями, сохраняющее действие всех операций). Другими словами, существует только одно поле Галуа $\GF{p^n}$ для фиксированных $p, n$.

Приведем примеры конечных полей.

Двоичное поле $\GF{2}$ состоит из двух элементов. Однако задать его можно разными способами:
\begin{itemize}
	\item Как множество из двух чисел <<0>> и <<1>> с определёнными на нём операциями <<сложение>> и <<умножение>> как сложение и умножение чисел по модулю 2. Нейтральным элементом по сложению будет <<0>>, по умножению -- <<1>>:
\[\begin{array}{ll}
	0 + 0 = 0,	& 	0 \cdot 0 = 0, \\
	0 + 1 = 1,	& 	0 \cdot 1 = 0, \\
	1 + 0 = 1,	& 	1 \cdot 0 = 0, \\
	1 + 1 = 0,	& 	1 \cdot 1 = 1. \\
\end{array}\]
	\item Как множество из двух логических объектов <<ЛОЖЬ>> ($F$) и <<ИСТИНА>> ($T$) с определёнными на нём операциями <<сложение>> и <<умножение>> как булевые операции <<исключающее или>> и <<и>> соответственно. Нейтральным элементом по сложению будет <<ЛОЖЬ>>, по умножению -- <<ИСТИНА>>:
\[\begin{array}{ll}
	F + F = F,	& 	F \cdot F = F, \\
	F + T = T,	& 	F \cdot T = F, \\
	T + F = T,	& 	T \cdot F = F, \\
	T + T = F,	& 	T \cdot T = T. \\
\end{array}\]
	\item Как множество из двух логических объектов <<ЛОЖЬ>> ($F$) и <<ИСТИНА>> ($T$) с определёнными на нём операциями <<сложение>> и <<умножение>> как булевые операции <<эквивалентность>> и <<или>> соответственно. Нейтральным элементом по сложению будет <<ИСТИНА>>, по умножению -- <<ЛОЖЬ>>:
\[\begin{array}{ll}
	F + F = T,	& 	F \cdot F = F, \\
	F + T = F,	& 	F \cdot T = T, \\
	T + F = F,	& 	T \cdot F = T, \\
	T + T = T,	& 	T \cdot T = T. \\
\end{array}\]
	\item Как множество из двух чисел <<0>> и <<1>> с определёнными на нём операциями <<сложение>> и <<умножение>>, заданными в табличном представлении. Нейтральным элементом по сложению будет <<1>>, по умножению -- <<0>>:
\[\begin{array}{ll}
	0 + 0 = 1,	& 	0 \cdot 0 = 0, \\
	0 + 1 = 0,	& 	0 \cdot 1 = 1, \\
	1 + 0 = 0,	& 	1 \cdot 0 = 1, \\
	1 + 1 = 1,	& 	1 \cdot 1 = 1. \\
\end{array}\]
\end{itemize}

Все перечисленные выше варианты множеств изоморфны друг другу. Поэтому в дальнейшем под конечным полем $\GF{p}$, где $p$ -- простое число, будем понимать поле, заданное как множество целых чисел от $0$ до $p-1$ включительно, на котором операции <<сложение>> и <<умножение>> заданы как операции сложения и умножения чисел по модулю числа $p$. Например, поле $\GF{7}$ будем считать состоящим из 7-и чисел $\{0, 1, 2, 3, 4, 5, 6\}$ с операциями умножения $(\cdot \mod 7)$ и сложения $(+ \mod 7)$ по модулю.

Конечное поле $\GF{p^n}, n > 1$ строится \textbf{расширением} \emph{базового} поля $\GF{p}$. Элемент поля представляется как многочлен степени $n-1$ (или меньше) с коэффициентами из базового поля $\GF{p}$:
    \[ \alpha = \sum\limits_{i=0}^{n-1} a_i x^i, ~ a_i \in \GF{p}. \]

Операция сложения элементов в таком поле традиционно задаётся как операция сложения коэффициентов при одинаковых степенях в базовом поле $\GF{p}$. Операция умножения -- как умножение многочленов со сложением и умножением коэффициентов в базовом поле $\GF{p}$ и дальнейшим приведением результата по модулю некоторого заданного (для поля) неприводимого\footnote{Многочлен называется \textbf{неприводимым}\index{многочлен!неприводимый}, если он не раскладывается на множители, и \textbf{приводимым}\index{многочлен!приводимый}, если раскладывается.} многочлена $m(x)$. Количество элементов в поле равно $p^n$.

Многочлен $g(x)$ называется \textbf{примитивным элементом}\index{многочлен!примитивный} (генератором) поля, если его степени порождают все ненулевые элементы, т.~е. $\GF{p^n} \setminus \{0\}$, заданное неприводимым многочленом $m(x)$, за исключением нуля:
    \[ \GF{p^n} \setminus \{0\} = \{ g(x), g^2(x), g^3(x), \dots, g^{p^n-1}(x) = 1 \mod m(x) \}. \]

Неприводимый многочлен $\mod m(x)$ называется  \textbf{примитивным}\index{многочлен!примитивный}, если $g(x)=x$.

\example
В табл. \ref{tab:irreducible-gf2} приведены примеры многочленов \emph{над полем} $\GF{2}$.
\begin{table}[!ht]
    \centering
    \caption{Пример многочленов над полем $\GF{2}$\label{tab:irreducible-gf2}}
    \begin{tabular}{|c|c|c|}
        \hline
        Многочлен & \parbox{2.5cm}{Упрощенная запись} & Разложение \\
        \hline
        $'1' x + '0'$ & $x$ & неприводимый \\
        $'1' x + '1'$ & $x+1$ & неприводимый \\
        $'1' x^2 + '0' x + '0'$ & $x^2$ & $x \cdot x$ \\
        $'1' x^2 + '0'x + '1'$ & $x^2 + 1$ & $(x+1) \cdot (x+1)$ \\
        $'1' x^2 + '1' x + '0'$ & $x^2 + x$ & $x \cdot (x+1)$ \\
        $'1' x^2 + '1' x + '1'$ & $x^2 + x + 1$ & неприводимый \\
        $'1' x^3 + '0' x^2 + '0' x + '1'$ & $x^3 + 1$ & $(x+1) \cdot (x^2+x+1)$ \\
        \hline
    \end{tabular}
\end{table}
\exampleend


\section{Конечные поля и операции в алгоритме AES}
\selectlanguage{russian}

В алгоритме блочного шифрования AES преобразования над байтами и битами осуществляются специальными математическими операциями. Биты и байты понимаются как элементы поля.

\subsection{Определение поля Галуа}

%Группой называется множество $\Gr$, в котором задана операция $\cdot$ между двумя элементами группы и удовлетворяются аксиомы:
%\begin{enumerate}
%    \item Замкнутость -- $\forall a,b \in \Gr: a \cdot b = c \in \Gr$.
%    \item Ассоциативность -- $\forall a,b,c \in \Gr: (a \cdot b) \cdot c = a \cdot (b \cdot c)$.
%    \item Существование единичного элемента -- $\exists ~ e \in \Gr: e\cdot a = a \cdot e = a$.
%    \item Существование обратного элемента -- $\forall a \in \Gr ~ \exists ~ b \in \Gr: a \cdot b = b \cdot a = e$.
%\end{enumerate}

\textbf{Полем} называется множество $\F$, для которого\index{поле}:
\begin{itemize}
    \item заданы операции умножения <<$\cdot$>> и сложения <<$+$>>;
    \item выполняются аксиомы группы по сложению <<$+$>> для всего множества $\F$: \\
        1. для $a,b,c \in \F$ верно $a+b \in \F$, \\
        2. $(a+b)+c = a+(b+c)$, \\
        3. существует нулевой элемент -- 0 (ноль), $a+0=0+a=a$, \\
        4. существует единственный обратный элемент $-a$: \\
        \indent \indent \indent ~~~~~~ $a + (-a) = (-a) + a = 0$;
    \item выполняются аксиомы группы по умножению <<$\cdot$>> для множества $\{ \F \backslash 0 \}$, за исключением нуля: \\
        1. для $a,b,c \in \{ \F \backslash 0 \}$ верно \\
        \indent \indent \indent ~~~~~~ $a \cdot b \in \{ \F \backslash 0 \}$, \\
        \indent \indent \indent ~~~~~~ $(a \cdot b) \cdot c = a \cdot (b \cdot c)$, \\
        2. существует единичный элемент -- 1 (единица), ~ $a \cdot 1 = 1 \cdot a = a$, \\
        3. существует единственный обратный элемент $a^{-1}:$ \\
        \indent \indent \indent ~~~~~~ $a \cdot a^{-1} = a^{-1} \cdot a = 1$, \\
        к нулю 0 не существует обратного элемента и $a \cdot 0 = 0$;
%    \item Удовлетворяющее аксиомам группы по сложению и умножению, обратный элемент по умножению существует ко всем элементам кроме 0.
    \item операции сложения и умножения коммутативны
        \[ \begin{array}{l}
            a + b = b + a, \\
            a \cdot b = b \cdot a; \\
        \end{array} \]
    \item выполняется свойство дистрибутивности
        \[ a \cdot (b + c) = (a \cdot b) + (a \cdot c). \]
\end{itemize}

Примеры \emph{бесконечных} полей (с бесконечным числом элементов) -- поле рациональных чисел $\group{Q}$, поле вещественных чисел $\group{R}$, поле комплексных чисел $\group{C}$ с обычными операциями сложения и умножения.

В криптографии рассматриваются \emph{конечные} поля (с конечным числом элементов), называемые также \textbf{полями Галуа}.

Число элементов в любом конечном поле равно $p^n$, где $p$ -- простое число и $n$ -- натуральное число. Обозначения поля Галуа: $\GF{p}, \GF{p^n}, \F_p, \F_{p^n}$ (аббревиатура от Galois field). Все поля Галуа $\GF{p^n}$ изоморфны друг другу (существует взаимно однозначное отображение между полями, сохраняющее действие всех операций). Другими словами, существует только одно поле Галуа $\GF{p^n}$ для фиксированных $p, n$.

Приведем примеры конечных полей.

Двоичное поле $\GF{2}$ состоит из двух элементов:
    \[ \GF{2} = \{0, 1\} \]
с операцией $(\cdot)$ обычного умножения и сложения  $\oplus$ по модулю 2 (исключающее ИЛИ, XOR).

Поле
    \[ \GF{3} = \{0, 1, 2\} \]
состоит из 3-х элементов с операциями умножения $(\cdot \mod 3)$ и сложения $(+ \mod 3)$ по модулю.

Двоичное поле $\GF{2^n}$ строится \textbf{расширением} \emph{базового} поля $\GF{2}$. Элемент поля задается многочленом степени $n-1$ (или меньше) с коэффициентами из базового поля $\GF{2}$:
    \[ \alpha = \sum\limits_{i=0}^{n-1} a_i x^i, ~ a_i \in \GF{2}. \]
Сложение многочленов -- сложение коэффициентов при одинаковых степенях $x^i$ в поле $\GF{2}$, т.~е. по $\text{XOR}$. Умножение многочленов в поле -- обычное умножение многочленов со сложением и умножением коэффициентов в поле $\GF{2}$ и дальнейшим приведением результата по модулю неприводимого многочлена $m(x)$ над полем $\GF{2}$.

Многочлен над базовым полем $\GF{p}$ называется \textbf{неприводимым}\index{многочлен!неприводимый}, если он не раскладывается на множители, и \textbf{приводимым}\index{многочлен!приводимый}, если раскладывается на множители.

Говорят, что неприводимый над базовым полем $\GF{p}$ многочлен $m(x)$ степени $n$ задает поле $\GF{p^n}$, если операция умножения в поле определена по модулю $m(x)$ (сложение определяется базовым полем, умножение -- многочленом $m(x)$). Количество элементов в поле определяется степенью $m(x)$) и равно $p^n$. Элементы поля есть остатки от деления на $m(x)$ и имеют степень не выше $n-1$.

Многочлен $g(x)$ называется \textbf{примитивным элементом}\index{многочлен!примитивный} (генератором) поля, если его степени порождают все ненулевые элементы, т.~е. $\GF{p^n} \setminus \{0\}$, заданное неприводимым многочленом $m(x)$, за исключением нуля:
    \[ \GF{p^n} \setminus \{0\} = \{ g(x), g^2(x), g^3(x), \dots, g^{p^n-1}(x) = 1 \mod m(x) \}. \]

Неприводимый многочлен $\mod m(x)$ называется  \textbf{примитивным}\index{многочлен!примитивный}, если $g(x)=x$.

\example
В табл. \ref{tab:irreducible-gf2} приведены примеры многочленов \emph{над полем} $\GF{2}$.
\begin{table}[!ht]
    \centering
    \caption{Пример многочленов над полем $\GF{2}$\label{tab:irreducible-gf2}}
    \begin{tabular}{|c|c|c|}
        \hline
        Многочлен & \parbox{2.5cm}{Упрощенная запись} & Разложение \\
        \hline
        $'1' x + '0'$ & $x$ & неприводимый \\
        $'1' x + '1'$ & $x+1$ & неприводимый \\
        $'1' x^2 + '0' x + '0'$ & $x^2$ & $x \cdot x$ \\
        $'1' x^2 + '0'x + '1'$ & $x^2 + 1$ & $(x+1) \cdot (x+1)$ \\
        $'1' x^2 + '1' x + '0'$ & $x^2 + x$ & $x \cdot (x+1)$ \\
        $'1' x^2 + '1' x + '1'$ & $x^2 + x + 1$ & неприводимый \\
        $'1' x^3 + '0' x^2 + '0' x + '1'$ & $x^3 + 1$ & $(x+1) \cdot (x^2+x+1)$ \\
        \hline
    \end{tabular}
\end{table}
\exampleend


\subsection{Операции с байтами в AES}

Чтобы определить операции сложения и умножения двух байт, введем сначала представление байта в виде многочлена степени 7 или меньше. Байт
    \[ a =( a_7, a_6, a_5, a_4, a_3, a_2, a_1, a_0) \]
преобразуется в многочлен $a(x)$ с коэффициентами 0 или 1 по правилу
    \[ a(x) = a_{7}x^{7}+a_{6}x^{6}+a_{5}x^{5}+a_{4}x^{4}+a_{3}x^{3}+a_{2}x^{2}+a_{1}x+a_{0}. \]

Далее байт трактуется как элемент конечного поля $\GF{2^8}$, заданного неприводимым многочленом
    \[ m(x) = x^{8}+x^{4}+x^{3}+x +1. \]

Произведение многочленов $a(x)$ и $b(x)$  по модулю многочлена $m(x)$  записывают как
    \[ c(x) = a(x) b(x) \mod m(x). \]
Остаток $c(x)$ представляет собой многочлен степени 7 или меньше. Его коэффициенты $(c_{7}, c_{6}, c_{5}, c_{4}, c_{3}, c_{2}, c_{1}, c_{0})$ образуют байт $c$, который и называется произведением байт $a$ и $b$.

Сложение байт осуществляется по $\oplus$ (исключающее ИЛИ), что является операцией сложения многочленов в двоичном поле.

\emph{Единичным} элементом поля является байт 00000001, или $\mathrm{'01'}$ в шестнадцатеричной записи. \emph{Нулевым} элементом поля является байт 00000000, или $\mathrm{'00'}$ в шестнадцатеричной записи. Одним из \emph{примитивных} элементов поля является байт (0 0 0 0 0 0 1 0), или $\mathrm{'02'}$ в шестнадцатеричной записи. Байты часто записывают в шестнадцатеричной форме, но при математических преобразованиях они должны интерпретироваться как элементы поля $\GF{2^8}$.

Для каждого ненулевого байта $a$ существует обратный байт $b$, такой, что их произведение является единичным байтом:
    \[ a b = 1 \mod m(x). \]
Обратный байт обозначается $b = a^{-1}$.

Для байта $a$, представленного многочленом $a(x)$, нахождение обратного байта $a^{-1}$ сводится к решению уравнения
    \[ m(x) d(x) + b(x) a(x) = 1. \]
Если такое решение найдено, то многочлен $b(x) \mod m(x)$ является представлением обратного байта $a^{-1}$. Обратный элемент (байт) может быть найден с помощью расширенного алгоритма Евклида для многочленов.

\example
Найти байт, обратный байту $a = \mathrm{'83'}$ в шестнадцатеричной записи. Так как $a(x) = x^{7} + x^{6} + 1$, то с помощью расширенного алгоритма Евклида находим
    \[ (x^{8} + x^{4} + x^{3} + x + 1) (x^{4} + x^{3} + x^{2} + x + 1) + (x^{7} + x^{6} + 1) (x^{5} + x^{3}) = 1. \]
Таким образом, обратный элемент поля или обратный байт $\mathrm{'83'}$ равен
    \[ x^{5} + x^{3}, ~ a^{-1} = \mathrm{'00101000'} = \mathrm{'28'}. \]
\exampleend

\example
В алгоритме блокового шифрования AES байты рассматриваются как элементы поля Галуа $\GF{2^8}$. Сложим байты $\mathrm{'57'}$ и $\mathrm{'83'}$. Представляя их многочленами, находим
    \[ (x^6 + x^4 + x^2 + x + 1) + (x^7 + x + 1) = x^7 + x^6 + x^4 + x^2, \]
или в двоичной записи --
    \[ 01010111 \oplus 10000011 = 11010100 = \mathrm{'D4'}. \]
Получили $\mathrm{'57'} + \mathrm{'83'} = \mathrm{'D4'}$.
\exampleend

\example
Выполним в поле $\GF{2^8}$, заданном неприводимым многочленом
    \[ m(x) = x^8 + x^4 + x^3 + x + 1 \]
(из алгоритма AES) операции с байтами: $\mathrm{'FA'} \cdot \mathrm{'A9'} + \mathrm{'E0'}$:
    \[ FA = 11111010, ~ A9 = 10101001, ~ E0 = 11100000, \]
    \[ (x^7 + x^6 + x^5 + x^4 + x^3  +x)(x^7 + x^5 + x^3 + 1) + (x^7 + x^6 + x^5) \mod m(x) = \]
    \[ = x^{14} + x^{13} + x^{10} + x^{8} + x^7 + x^3 + x \mod m(x) = \]
    \[ = (x^{14} + x^{13} + x^{10} + x^{8} + x^7 + x^3 + x) + x^6 \cdot m(x) \mod m(x) = \]
    \[ = x^{13} + x^9 + x^8 + x^6 + x^3 + x \mod m(x) = \]
    \[ = (x^{13} + x^9 + x^8 + x^6 + x^3 + x) + x^5 \cdot m(x) \mod m(x) = \]
    \[ = x^5 + x^3 + x \mod m(x) = \mathrm{'2A'}. \]
\exampleend


\subsection{Операции над вектором из байт в AES}
%\subsection{Многочлены над полем в алгоритме AES}

Поле $\GF{2^{nk}}$ можно задать как расширение степени $nk$ базового поля $\GF{2}$:
    \[ \alpha \in \GF{2^{nk}}, \alpha = \sum\limits_{i=0}^{nk-1} a_i x^i, ~ a_i \in \GF{2} \]
с неприводимым многочленом $r(x)$ степени $nk$ над полем $\GF{2}$,
    \[ r(x) = \sum\limits_{i=0}^{nk} a_i x^i, ~ a_i \in \GF{2}, ~ a_{nk} = 1. \]

Поле $\GF{2^{nk}}$ можно задать и как расширение степени $k$ базового поля $\GF{2^n}$:
    \[ \alpha \in \GF{(2^n}^k), \alpha = \sum\limits_{i=0}^{k-1} a_i x^i, ~ a_i \in \GF{2^n} \]
с неприводимым многочленом $r(x)$ степени $k$ над полем $\GF{2^n}$,
    \[ r(x) = \sum\limits_{i=0}^{k} a_i x^i, ~ a_i \in \GF{2^n}, ~ a_k = 1. \]

\example
В табл. \ref{tab:irreducible-gf8} приведены примеры приводимых и неприводимых многочленов над полем $\GF{2^8}$.
\begin{table}[!ht]
    \centering
    \caption{Примеры многочленов над полем $\GF{2^8}$\label{tab:irreducible-gf8}}
    \begin{tabular}{|c|c|}
        \hline
        Многочлен & Разложение \\
        \hline
        $\mathrm{'01'} x + \mathrm{'00'}$ & неприводимый \\
        $\mathrm{'01'} x + \mathrm{'01'}$ & неприводимый \\
        $\mathrm{'01'} x + \mathrm{'A9'}$ & неприводимый \\
        $\mathrm{'01'} x^2 + \mathrm{'00'} x + \mathrm{'00'}$ & $(\mathrm{'01'} x) \cdot (\mathrm{'01'} x)$ \\
        $\mathrm{'1D'} x^2 + \mathrm{'AF'} x + \mathrm{'52'}$ & $(\mathrm{'41'} x + \mathrm{'0A'}) \cdot (\mathrm{'E3'} x + \mathrm{'5A'})$ \\
        $\mathrm{'01'} x^4 + \mathrm{'01'}$ & $(\mathrm{'01'} x + \mathrm{'01'})^4$ \\
        \hline
    \end{tabular}
\end{table}
\exampleend

В алгоритме AES вектор-столбец $\mathbf{a}$ состоит из четырех байт $a_{0}, a_{1}, a_{2}, a_{3}$. Ему ставится в соответствие многочлен $\mathbf{a}(y)$ от переменной $y$ вида
    \[ \mathbf{a}(y) = a_{3}y^{3}+a_{2}y^{2}+a_{1}y+a_{0}, \]
причем коэффициенты многочлена (байты) интерпретируются как элементы поля $\GF{2^{8}}$. Это значит, что при сложении или умножении двух таких многочленов их коэффициенты складываются и перемножаются, как определено выше.

Многочлены $\mathbf{a}(y)$ и $\mathbf{b}(y)$ умножаются по модулю многочлена
    \[ \mathbf{M}(y)= \mathrm{'01'} y^4 + \mathrm{'01'} = y^4 + 1, ~ \mathrm{'01'} \in \GF{2^8}, \]
    \[ \mathbf{M}(y)= (\mathrm{'01'}, \mathrm{'00'},\mathrm{'00'}, \mathrm{'00'}, \mathrm{'01'}), \]
который \emph{не} является неприводимым над $\GF{2^8}$.
%Следовательно, многочлен $\mathbf{a}(y)$ задает многочлен третьей степени над полем $\GF{2^8}$, но не является элементом поля $\GF{2^{32}}$.

Операция умножения по модулю $\mathbf{M}(y)$  обозначается $\otimes$:
    \[ \mathbf{a}(y) ~ \mathbf{b}(y) \mod \mathbf{M}(y) ~\equiv~ \mathbf{a}(y) \otimes \mathbf{b}(y). \]

Операция <<Перемешивание столбца>> в шифровании AES состоит в умножении многочлена столбца на
    \[ \mathbf{c}(y) = (03, 01, 01, 02) = \mathrm{'03'} y^3 + \mathrm{'01'} y^2 + \mathrm{'01'} y + \mathrm{'02'} \]
по модулю $\mathbf{M}(y)$. Многочлен $\mathbf{c}(y)$ имеет обратный многочлен
    \[ \mathbf{d}(y) = \mathbf{c}^{-1}(y) \mod \mathbf{M}(y) = (\mathrm{0B}, \mathrm{0D}, \mathrm{09}, \mathrm{0E}) = \]
        \[ = \mathrm{'0B'} y^3 + \mathrm{'0D'} y^2 + \mathrm{'09'} y + \mathrm{'0E'}, \]
    \[ \mathbf{c}(y) \otimes \mathbf{d}(y) = (00, 00, 00, 01) = 1. \]
При расшифровании выполняется умножение на $\mathbf{d}(y)$ вместо $\mathbf{c}(y)$.

Так как
    \[ y^j = y^{j \mod 4} \mod y^4+1, \]
где коэффициенты из поля $\GF{2^8}$, то произведение многочленов
    \[ \mathbf{a}(y) = a_{3}y^{3}+ a_{2}y^{2} + a_{1}y + a_{0} \]
и
    \[ \mathbf{b}(y) = b_{3}y^{3} + b_{2}y^{2} + b_{1}y + b_{0}, \]
обозначаемое как
    \[ \mathbf{f}(y) = \mathbf{a}(y) \otimes \mathbf{b}(y) = f_{3}y^{3} + f_{2}y^{2} + f_{1}y + f_{0}, \]
содержит коэффициенты
\[
    \begin{array}{ccccccccc}
        f_{0} & = & a_{0}b_{0} & + & a_{3}b_{1} & + & a_{2}b_{2} & + & a_{1}b_{3}, \\
        f_{1} & = & a_{1}b_{0} & + & a_{0}b_{1} & + & a_{3}b_{2} & + & a_{2}b_{3}, \\
        f_{2} & = & a_{2}b_{0} & + & a_{1}b_{1} & + & a_{0}b_{2} & + & a_{3}b_{3}, \\
        f_{3} & = & a_{3}b_{0} & + & a_{2}b_{1} & + & a_{1}b_{2} & + &  a_{0}b_{3}.
    \end{array}.
\]

Эти соотношения можно переписать также в матричном виде:
\[
    \begin{array}{cccc}
        \left[ \begin{array}{c}
            f_{0} \\
            f_{1} \\
            f_{2} \\
            f_{3}
        \end{array} \right] &  = & \left[\begin{array}{cccc}
            a_{0} & a_{3} & a_{2} & a_{1} \\
            a_{1} & a_{0} & a_{3} & a_{2} \\
            a_{2} & a_{1} & a_{0} & a_{3} \\
            a_{3} & a_{2} & a_{1} & a_{0}
        \end{array}\right] & \left[\begin{array}{c}
            b_{0} \\
            b_{1} \\
            b_{2} \\
            b_{3}
        \end{array} \right]
    \end{array}.
\]

Умножение матриц производится в поле $\GF{2^8}$. Матричное представление полезно, если нужно умножать фиксированный вектор на несколько различных векторов.

\example
Вычислим для $\mathbf{a}(y) = (\mathrm{F2}, \mathrm{7E}, \mathrm{41}, \mathrm{0A})$ произведение $\mathbf{a}(y) \otimes \mathbf{c}(y)$:
\[
    \mathbf{c}(y) = (03, 01, 01, 02),
\] \[
    \mathbf{d}(y) = \mathbf{c}^{-1}(y) \mod \mathbf{M}(y) = (\mathrm{0B}, \mathrm{0D}, \mathrm{09}, \mathrm{0E}).
\] \[
    \mathbf{a}(y) \otimes \mathbf{c}(y) =
    \left[ \begin{array}{cccc}
        \mathrm{0A} & \mathrm{F2} & \mathrm{7E} & \mathrm{41} \\
        \mathrm{41} & \mathrm{0A} & \mathrm{F2} & \mathrm{7E} \\
        \mathrm{7E} & \mathrm{41} & \mathrm{0A} & \mathrm{F2} \\
        \mathrm{F2} & \mathrm{7E} & \mathrm{41} & \mathrm{0A} \\
    \end{array} \right] \cdot
    \left[ \begin{array}{c} \mathrm{02} \\ \mathrm{01} \\ \mathrm{01} \\ \mathrm{03} \end{array} \right] =
\] \[
    \left[ \begin{array}{ccccccc}
        \mathrm{0A} \cdot \mathrm{02} & \oplus & \mathrm{F2} & \oplus & \mathrm{7E} & \oplus & \mathrm{41} \cdot \mathrm{03} \\
        \mathrm{41} \cdot \mathrm{02} & \oplus & \mathrm{0A} & \oplus & \mathrm{F2} & \oplus & \mathrm{7E} \cdot \mathrm{03} \\
        \mathrm{7E} \cdot \mathrm{02} & \oplus & \mathrm{41} & \oplus & \mathrm{0A} & \oplus & \mathrm{F2} \cdot \mathrm{03} \\
        \mathrm{F2} \cdot \mathrm{02} & \oplus & \mathrm{7E} & \oplus & \mathrm{41} & \oplus & \mathrm{0A} \cdot \mathrm{03} \\
    \end{array} \right] =
    =\left[ \begin{array}{c} \mathrm{5B} \\ \mathrm{F8} \\ \mathrm{BA} \\ \mathrm{DE} \end{array} \right];
\] \[
    \begin{array}{l}
        \mathbf{a}(y) \otimes \mathbf{c}(y) = \mathbf{b}(y), \\
        \mathbf{b}(y) \otimes \mathbf{d}(y) = \mathbf{a}(y); \\
    \end{array}
\] \[
    \begin{array}{ccccc}
        (\mathrm{F2}, \mathrm{7E}, \mathrm{41}, \mathrm{0A}) & \otimes & (\mathrm{03}, \mathrm{01}, \mathrm{01}, \mathrm{02}) & = & (\mathrm{DE}, \mathrm{BA}, \mathrm{F8}, \mathrm{5B}), \\
        (\mathrm{DE}, \mathrm{BA}, \mathrm{F8}, \mathrm{5B}) & \otimes & (\mathrm{0B}, \mathrm{0D}, \mathrm{09}, \mathrm{0E}) & = & (\mathrm{F2}, \mathrm{7E}, \mathrm{41}, \mathrm{0A}). \\
    \end{array}
\]
\exampleend


\section{Модульная арифметика}
\selectlanguage{russian}

\subsection{Сложность модульных операций}

Криптосистемы с открытым ключом, как правило, построены в модульной арифметике с длиной модуля от сотни до нескольких тысяч разрядов. Сложность алгоритмов оценивают как количество битовых операций в зависимости от длины. В табл. \ref{tab:mod-binary-complexity} приведены оценки (с точностью до порядка) сложности модульных операций\index{битовая сложность} для простых (или "школьных") алгоритмов вычислений. На самом деле, для реализации арифметики длинных чисел (сотни или тысячи двоичных разрядов) следует применять существенно более эффективные (более "хитрые") алгоритмы вычислений, использующие, например, специальный вид быстрого преобразования Фурье и другие приемы.

\begin{table}[!ht]
    \centering
    \caption{Битовая сложность операций по модулю $n$ длиной $k= \log n$ бит\label{tab:mod-binary-complexity}}
    \begin{tabular}{| p{0.7\textwidth} | c |}
        \hline
        Операция, алгоритм & сложность \\
        \hline
        1. $a \pm b \mod n$ & $O(k)$ \\
        2. $a \cdot b \mod n$ & $O(k^2)$ \\
        3. $\gcd(a, b)$, алгоритм Евклида & $O(k^2)$ \\
        4. $(a,b) \rightarrow (x,y,d) : ax + by = d = \gcd(a,b)$, расширенный алгоритм Евклида & $O(k^2)$ \\
        5. $a^{-1} \mod n$, расширенный алгоритм Евклида & $O(k^2)$ \\
        6. Китайская теорема об остатках & $O(k^2)$ \\
        7. $a^b \mod n$ & $O(k^3)$ \\
        \hline
    \end{tabular}

\end{table}


\subsection{Возведение в степень по модулю}

Метод называется <<возводи в квадрат и перемножай>>. Найдем $a^b \mod n$.
    \[ b = \sum_{i=0}^{k-1} b_i 2^i, \]
    \[ a^b = a^{\sum\limits_{i=0}^{k-1} b_i 2^i} = \prod_{i=0}^{k-1} (a^{{2^i} b_i} \mod n) \mod n. \]
Последовательно вычисляем квадраты
    \[ a_0 = a, ~ a_1 = a_0^2 \mod n, ~ a_2 = a_1^2 \mod n,  \ldots  \]
по модулю $n$ и перемножаем $a_i$, которым соответствует $b_i = 1$. Число возведений в квадрат равно $k-1$ (если $b_{k-1} =1$), а число умножений меньше или равно $k-1$. Возведение в квадрат и умножение можно считать операцией с квадратичной битовой сложностью $O(k^2)$. Поэтому общая битовая сложность возведения в степень -- кубическая
    \[ O(2(k-1)k^2) = O(k^3). \]

\example
\[ \begin{array}{l}
    8^{24} \mod 25 = 8^8 \cdot 8^{16} \mod 25, \\
    8^2 = 14, \\
    8^4 = -4, \\
    8^8 = 16, \\
    8^{16} = 6, \\
    8^{24} = 16 \cdot 6 = -4 \mod 25.
\end{array} \]
\exampleend


\input{euclidean_algorithm}

\input{chinese_remainder_theorem}

\input{pseudo-primes}

\input{groups_of_ec_points_over_finite_fields}

\section[Полиномиальные и экспоненциальные алгоритмы]{Полиномиальные и \\ экспоненциальные алгоритмы}

Данный раздел поясняет обоснованность стойкости криптосистем с открытым ключом и имеет лишь косвенное отношение к дискретной математике.

Машина Тьюринга (МТ) (модель, представляющая любой вычислительный алгоритм) состоит из следующих частей:
\begin{itemize}
    \item неограниченной ленты, разделенной на клетки; в каждой клетке содержится символ из конечного алфавита, содержащего пустой символ blank; если символ ранее не был записан на ленту, то он считается blank;
    \item печатающей головки, которая может считать, записать символ $a_i$ и передвинуть ленту на 1 клетку влево-вправо $d_k$;
    \item конечной таблицы действий
    \[ (q_i, a_j) \rightarrow (q_{i1}, a_{j1}, d_k), \]
где $q$ -- состояние машины.
\end{itemize}

Если таблица переходов однозначна, то машина Тьюринга\index{машина Тьюринга} называется детерминированной. \textbf{Детерминированная} машина Тьюринга может \emph{имитировать} любую существующую детерминированную ЭВМ. Если таблица переходов не однозначна, то есть $(q_i, a_j)$ может переходить по нескольким правилам, то машина \textbf{недетерминированная}. \emph{Квантовый компьютер} является примером недетерминированной машины Тьюринга.

Класс задач $\set{P}$ -- задачи, которые могут быть решены за \emph{полиномиальное} время\index{задача!полиномиальная} на \emph{детерминированной} машине Тьюринга. Пример полиномиальной сложности (количество битовых операций)
    \[ O(k^{\textrm{const}}), \]
где $k$ -- длина входных параметров алгоритма. Операция возведения в степень в модульной арифметике $a^b \mod n$ имеет кубическую сложность $O(k^3)$, где $k$ -- двоичная длина чисел $a,b,n$.

Класс задач $\set{NP}$ -- обобщение класса $\set{P} \subseteq \set{NP}$, включает задачи, которые могут быть решены за \emph{полиномиальное} время на \emph{недетерминированной} машине Тьюринга. Пример сложности задач из $\set{NP}$ -- экспоненциальная сложность\index{задача!экспоненциальная}
    \[ O(\textrm{const}^k). \]
Описанный алгоритм Гельфонда (в разделе криптостойкости системы Эль-Гамаля\index{криптосистема!Эль-Гамаля}) решения задачи дискретного логарифма по нахождению $x$ для заданных $g \mod p$ и $a = g^x \mod p$ имеет сложность $O(e^{k/2})$, где $k$ -- двоичная длина чисел.

В криптографии полиномиальные $\set{P}$ алгоритмы считаются \emph{легкими и вычислимыми} на ЭВМ, которые являются детерминированными машинами Тьюринга. Неполиномиальные (экспоненциальные) $\set{NP}$ алгоритмы считаются \emph{трудными и невычислимыми} на ЭВМ, так как из-за экспоненциального роста сложности всегда можно выбрать такой параметр $k$, что время вычисления станет сравнимым с возрастом Вселенной.

Задача факторизации числа, задача дискретного логарифмирования в группе считаются $\set{NP}$-задачами.

Класс $\set{NP}$-полных задач -- подмножество задач из $\set{NP}$, для которых не известен полиномиальный алгоритм для детерминированной машины Тьюринга, и все задачи могут быть сведены друг к другу за полиномиальное время на \emph{детерминированной} машине Тьюринга. Например, задача об укладке рюкзака является $\set{NP}$-полной.

Стойкость криптосистем с \emph{открытым} ключом, как правило, основана на $\set{NP}$ или $\set{NP}$-полных задачах:
\begin{enumerate}
    \item RSA\index{криптосистема!RSA} -- $\set{NP}$-задача факторизации (строго говоря, на трудности извлечения корня степени $e$ по модулю $n$).
    \item Криптосистемы типа Эль-Гамаля\index{криптосистема!Эль-Гамаля} -- $\set{NP}$-задача дискретного логарифмирования.
\end{enumerate}

\emph{Нерешенной} проблемой является доказательство неравенства
    \[ \set{P} \neq \set{NP}. \]
Именно на гипотезе о том, что для для некоторых задач не существует полиномиальных алгоритмов, и основана стойкость криптосистем с открытым ключом.

\input{coincide-index_method}

%\chapter{Задачи и упражнения}
%
%К \textbf{примерам 1, 2, 3} \textbf{упражнение 1}. Указать способ расшифрования для легального получателя шифротекста и указать способ дешифрования для криптоаналитика, не знающего ключа.
%
%\textbf{Упражнение 2}. Пусть $M_1, M_2, M_3,\ldots,M_s$ -- набор перестановок. Показать, что существует единственная перестановка $M=M_1,M_2,M_3,\ldots M_s$.
%
%\textbf{Упражнение 3}. Вскрыть одиночную ячейку Фейстеля. Для этого задать конкретную функцию $F(K,R)$ и по конкретным значениям $L_{1}$ и $R_{1}$ найти $K$.
%
%\textbf{Упражнение 4}. Разделим последовательность на блоки, каждый из которых содержит 2 бита.
%
%\[\begin{array}{cc} {z_{1} } & {z_{2} } \end{array}|\begin{array}{cc} {z_{3} } & {z_{4} } \end{array}| \ldots |\begin{array}{cc} {z_{N} } & {z_{N+1} } \end{array}\]
%Блок может принимать значения $z_{1} z_{2} =\begin{array}{c} {11} \\ {10} \\ {01} \\ {00} \end{array}$
%Преобразуем последовательность символов:
% если $z_{1} z_{2} =11$ или $z_{1} z_{2} =00$, то пара выбрасывается;
%если $[z_{1} z_{2} =10$, то записываем новый символ $u=1$; если
%$z_{1} z_{2} =01$, то записываем новый символ $u=0$.
%Получаем новую двоичную последовательность.
%
%Показать, что вероятностное распределение символов в новой последовательности является равномерным.
%
%\textbf{Упражнение 5}.Предположим, что криптоаналитик знает, что период генерируемой $M$ -последовательности равен $T=2^{L} -1$. Пусть ему известна часть последовательности длины, меньшей периода: $T_{1}<2^{L} -1$.
% При каком значении $T_{1}$  криптоаналитик может найти многочлен обратной связи.
%
%\textbf{Упражнение 6}. Ответить на вопрос: <<Как подделать ЭП, не зная закрытого ключа?>>
%
%\textbf{Упражнение 7}. При помощи формул Виета найти дискриминант многочлена, представляющего эллиптическую кривую.

\printindex

\chapter*{Литература}
\addcontentsline{toc}{chapter}{Литература}
\begingroup
\renewcommand{\chapter}[2]{}%
%\bibliographystyle{ugost2008s}
%\bibliography{bibliography}
\printbibliography
\endgroup

\end{document}
